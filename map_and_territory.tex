\part{Map and Territory}


\chapter{Predictably Wrong}

\mysection{What Do I Mean By ``Rationality''?}

{
 I mean:}

\begin{enumerate}
\item {
 \textbf{Epistemic rationality}: systematically improving the
accuracy of your beliefs.}

\item {
 \textbf{Instrumental rationality}: systematically achieving your
 values.}
\end{enumerate}

{
 When you open your eyes and look at the room around you,
you'll locate your laptop in relation to the table, and
you'll locate a bookcase in relation to the wall. If
something goes wrong with your eyes, or your brain, then your mental
model might say there's a bookcase where no bookcase
exists, and when you go over to get a book, you'll be
disappointed.}

{
 This is what it's like to have a false belief, a
map of the world that doesn't correspond to the
territory. Epistemic rationality is about building accurate maps
instead. This correspondence between belief and reality is commonly
called ``truth,'' and
I'm happy to call it that.}

{
 Instrumental rationality, on the other hand, is about
\textit{steering} reality---sending the future where you want it to go.
It's the art of choosing actions that lead to outcomes
ranked higher in your preferences. I sometimes call this
``winning.''}

{
 So rationality is about forming true beliefs and making winning
decisions.}

{
 Pursuing ``truth'' here
doesn't mean dismissing uncertain or indirect evidence.
Looking at the room around you and building a mental map of it
isn't different, in principle, from believing that the
Earth has a molten core, or that Julius Caesar was bald. Those
questions, being distant from you in space and time, might seem more
airy and abstract than questions about your bookcase. Yet there are
facts of the matter about the state of the Earth's core
in 2015 CE and about the state of Caesar's head in 50
BCE. These facts may have real effects upon you even if you never find
a way to meet Caesar or the core face-to-face.}

{
 And ``winning'' here need not
come at the expense of others. The project of life can be about
collaboration or self-sacrifice, rather than about competition.
``Your values'' here means
\textit{anything you care about}, including other people. It
isn't restricted to \textit{selfish} values or
\textit{unshared} values.}

{
 When people say ``X is
rational!'' it's usually just a more
strident way of saying ``I think X is
true'' or ``I think X is
good.'' So why have an additional word for
``rational'' as well as
``true'' and
``good''?}

{
 An analogous argument can be given against using
``true.'' There is no need to say
``it is true that snow is white''
when you could just say ``snow is
white.'' What makes the idea of truth useful is that
it allows us to talk about the general features of map-territory
correspondence. ``True models usually produce better
experimental predictions than false models'' is a
useful generalization, and it's not one you can make
without using a concept like
``true'' or
``accurate.''}

{
 Similarly, ``Rational agents make decisions that
maximize the probabilistic expectation of a coherent utility
function'' is the kind of thought that depends on a
concept of (instrumental) rationality, whereas
``It's rational to eat
vegetables'' can probably be replaced with
``It's useful to eat
vegetables'' or
``It's in your interest to eat
vegetables.'' We need a concept like
``rational'' in order to note
general facts about those ways of thinking that systematically produce
truth or value---and the systematic ways in which we fall short of
those standards.}

{
 Sometimes experimental psychologists uncover human reasoning that
seems very strange. For example, someone rates the probability
``Bill plays jazz'' as \textit{less}
than the probability ``Bill is an accountant who plays
jazz.'' This seems like an odd judgment, since any
particular jazz-playing accountant is obviously a jazz player. But to
what higher vantage point do we appeal in saying that the judgment is
\textit{wrong}?}

{
 Experimental psychologists use two gold standards:
\textit{probability theory}, and \textit{decision theory}.}

{
 Probability theory is the set of laws underlying rational belief.
The mathematics of probability describes equally and without
distinction (a) figuring out where your bookcase is, (b) figuring out
the temperature of the Earth's core, and (c) estimating
how many hairs were on Julius Caesar's head.
It's all the same problem of how to process the
evidence and observations to revise
(``update'') one's
beliefs. Similarly, decision theory is the set of laws underlying
rational action, and is equally applicable regardless of what
one's goals and available options are.}

{
 Let ``$P(\text{such-and-such})$'' stand
for ``the probability that such-and-such
happens,'' and $P(A,B)$ for ``the
probability that both $A$ and $B$ happen.'' Since it is a
universal law of probability theory that $P(A) \geq P(A,B)$, the
judgment that $P$(Bill plays jazz) is less than $P$(Bill plays jazz, Bill
is an accountant) is labeled incorrect.}

{
 To keep it technical, you would say that this probability judgment
is \textit{non-Bayesian}. Beliefs and actions that are rational in this
mathematically well-defined sense are called
``Bayesian.''}

{
 Note that the modern concept of rationality is not about reasoning
in words. I gave the example of opening your eyes, looking around you,
and building a mental model of a room containing a bookcase against the
wall. The modern concept of rationality is general enough to include
your eyes and your brain's visual areas as
things-that-map. It includes your wordless intuitions as well. The math
doesn't care whether we use the same English-language
word, ``rational,'' to refer to
Spock and to refer to Bayesianism. The math models good ways of
achieving goals or mapping the world, regardless of whether those ways
fit our preconceptions and stereotypes about what
``rationality'' is supposed to be.}

{
 This does not quite exhaust the problem of what is meant in
practice by ``rationality,'' for two
major reasons:}

{
 First, the Bayesian formalisms in their full form are
computationally intractable on most real-world problems. No one can
\textit{actually} calculate and obey the math, any more than you can
predict the stock market by calculating the movements of quarks.}

{
 This is why there is a whole site called ``Less
Wrong,'' rather than a single page that simply states
the formal axioms and calls it a day. There's a whole
further art to finding the truth and accomplishing value \textit{from
inside a human mind}: we have to learn our own flaws, overcome our
biases, prevent ourselves from self-deceiving, get ourselves into good
emotional shape to confront the truth and do what needs doing, et
cetera, et cetera.}

{
 Second, sometimes the meaning of the math itself is called into
question. The exact rules of probability theory are called into
question by, e.g., anthropic problems in which the number of observers
is uncertain. The exact rules of decision theory are called into
question by, e.g., Newcomblike problems in which other agents may
predict your decision before it happens.\footnote{\textbf{Editor's Note:} For a good introduction
to Newcomb's Problem, see Holt.\footnotemark \  More
generally, you can find definitions and explanations for many of the
terms in this book at the website
wiki.lesswrong.com/wiki/RAZ\_Glossary.\comment{1}}\footnotetext{Jim Holt, ``Thinking Inside the
Boxes,'' \textit{Slate} (2002),
\url{http://www.slate.com/articles/arts/egghead/2002/02/thinkinginside\%5C\_the\%5C\_boxes.single.html}.\comment{2}}}

{
 In cases like these, it is futile to try to settle the problem by
coming up with some new definition of the word
``rational'' and saying,
``Therefore my preferred answer, \textit{by
definition,} is what is meant by the word
`rational.'\,'' This
simply raises the question of why anyone should pay attention to your
definition. I'm not interested in probability theory
because it is the holy word handed down from Laplace.
I'm interested in Bayesian-style belief-updating (with
Occam priors) because I expect that this style of thinking gets us
systematically closer to, you know, \textit{accuracy}, the map that
reflects the territory.}

{
 And then there are questions of how to think that seem not quite
answered by either probability theory or decision theory---like the
question of how to feel about the truth once you have it. Here, again,
trying to define ``rationality'' a
particular way doesn't support an answer, but merely
presumes one.}

{
 I am not here to argue the meaning of a word, not even if that
word is ``rationality.'' The point
of attaching sequences of letters to particular concepts is to let two
people \textit{communicate}{}---to help transport thoughts from one
mind to another. You cannot change reality, or prove the thought, by
manipulating which meanings go with which words.}

{
 So if you understand what concept I am \textit{generally getting
at} with this word ``rationality,''
and with the sub-terms ``epistemic
rationality'' and ``instrumental
rationality,'' we \textit{have communicated}: we have
accomplished everything there is to accomplish by talking about how to
define ``rationality.''
What's left to discuss is not \textit{what meaning} to
attach to the syllables
``ra-tion-al-i-ty'';
what's left to discuss is \textit{what is a good way to
think}.}

{
 If you say, ``It's
(epistemically) rational for me to believe $X$, but the truth is
$Y$,'' then you are probably using the word
``rational'' to mean something other
than what I have in mind. (E.g.,
``rationality'' should be
\textit{consistent under
reflection}{}---``rationally''
looking at the evidence, and
``rationally'' considering how your
mind processes the evidence, shouldn't lead to two
different conclusions.)}

{
 Similarly, if you find yourself saying, ``The
(instrumentally) rational thing for me to do is $X$, but the right thing
for me to do is $Y$,'' then you are almost certainly
using some other meaning for the word
``rational'' or the word
``right.'' I use the term
``rationality''
\textit{normatively}, to pick out desirable patterns of thought.}

{
 In this case---or in any other case where people disagree about
word meanings---you should substitute more specific language in place
of ``rational'':
``The self-benefiting thing to do is to run away, but
I hope I would at least try to drag the child off the railroad
tracks,'' or ``Causal decision
theory as usually formulated says you should two-box on
Newcomb's Problem, but I'd rather have
a million dollars.''}

{
 In fact, I recommend reading back through this essay, replacing
every instance of ``rational'' with
``foozal,'' and seeing if that
changes the connotations of what I'm saying any. If so,
I say: strive not for rationality, but for foozality.}

{
 The word ``rational'' has
potential pitfalls, but there are plenty of \textit{non}{}-borderline
cases where ``rational'' works fine
to communicate what I'm getting at. Likewise
``irrational.'' In these cases
I'm not afraid to use it.}

{
 Yet one should be careful not to \textit{overuse} that word. One
receives no points merely for pronouncing it loudly. If you speak
overmuch of the Way, you will not attain it.}

\myendsectiontext


\bigskip

\mysection{Feeling Rational} 

{
 A popular belief about
``rationality'' is that rationality
opposes all emotion---that all our sadness and all our joy are
automatically anti-logical by virtue of being \textit{feelings.} Yet
strangely enough, I can't find any theorem of
probability theory which proves that I should appear ice-cold and
expressionless. }

{
 So is rationality orthogonal to feeling? No; our emotions arise
from our models of reality. If I believe that my dead brother has been
discovered alive, I will be happy; if I wake up and realize it was a
dream, I will be sad. P.~C.~Hodgell said: ``That which
can be destroyed by the truth should be.'' My
dreaming self's happiness was opposed by truth. My
sadness on waking is rational; there is no truth which destroys it.}

{
 Rationality begins by asking how-the-world-is, but spreads virally
to any other thought which depends on how we think the world is. Your
beliefs about ``how-the-world-is''
can concern anything you think is out there in reality, anything that
either does or does not exist, any member of the class
``things that can make other things
happen.'' If you believe that there is a goblin in
your closet that ties your shoes' laces together, then
this is a belief about how-the-world-is. Your shoes are real---you can
pick them up. If there's something out there that can
reach out and tie your shoelaces together, it must be real too, part of
the vast web of causes and effects we call the
``universe.''}

{
 \textit{Feeling angry at} the goblin who tied your shoelaces
involves a state of mind that is not \textit{just} about
how-the-world-is. Suppose that, as a Buddhist or a lobotomy patient or
just a very phlegmatic person, finding your shoelaces tied together
didn't make you angry. This wouldn't
affect what you expected to see in the world---you'd
still expect to open up your closet and find your shoelaces tied
together. Your anger or calm shouldn't affect your best
guess here, because what happens in your closet does not depend on your
emotional state of mind; though it may take some effort to think that
clearly.}

{
 But the angry feeling is tangled up with a state of mind that
\textit{is} about how-the-world-is; you become angry \textit{because}
you think the goblin tied your shoelaces. The criterion of rationality
spreads virally, from the initial question of whether or not a goblin
tied your shoelaces, to the resulting anger.}

{
 Becoming more rational---arriving at better estimates of
how-the-world-is---can diminish feelings \textit{or intensify them}.
Sometimes we run away from strong feelings by denying the facts, by
flinching away from the view of the world that gave rise to the
powerful emotion. If so, then as you study the skills of rationality
and train yourself not to deny facts, your feelings will become
stronger.}

{
 In my early days I was never quite certain whether it was
\textit{all right} to feel things strongly---whether it was allowed,
whether it was proper. I do not think this confusion arose only from my
youthful misunderstanding of rationality. I have observed similar
troubles in people who do not even aspire to be rationalists; when they
are happy, they wonder if they are really allowed to be happy, and when
they are sad, they are never quite sure whether to run away from the
emotion or not. Since the days of Socrates at least, and probably long
before, the way to appear cultured and sophisticated has been to never
let anyone see you care strongly about anything. It's
\textit{embarrassing} to feel---it's just not done in
polite society. You should see the strange looks I get when people
realize how much I care about rationality. It's not the
unusual subject, I think, but that they're not used to
seeing sane adults who visibly care about \textit{anything.}}

{
 But I know, now, that there's nothing wrong with
feeling strongly. Ever since I adopted the rule of
``That which can be destroyed by the truth should
be,'' I've also come to realize
``That which the truth nourishes should
thrive.'' When something good happens, I am happy,
and there is no confusion in my mind about whether it is rational for
me to be happy. When something terrible happens, I do not flee my
sadness by searching for fake consolations and false silver linings. I
visualize the past and future of humankind, the tens of billions of
deaths over our history, the misery and fear, the search for answers,
the trembling hands reaching upward out of so much blood, what we could
become someday when we make the stars our cities, all that darkness and
all that light---I know that I can never truly understand it, and I
haven't the words to say. Despite all my philosophy I
am still embarrassed to confess strong emotions, and
you're probably uncomfortable hearing them. But I know,
now, that it is rational to feel.}

\myendsectiontext

\mysection{Why Truth? And \ldots}
\label{why_truth}

{
 Some of the comments on \textit{Overcoming Bias} have touched on
the question of why we ought to seek truth. (Thankfully not many have
questioned what truth is.) Our shaping motivation for configuring our
thoughts to rationality, which determines whether a given configuration
is ``good'' or
``bad,'' comes from whyever we
wanted to find truth in the first place. }

{
 It is written:\footnote{Eliezer Yudkowsky, {\em Rationality From AI to Zombies}, pg \pageref{twelve_virtues}} ``The first virtue is
curiosity.'' Curiosity is one reason to seek truth,
and it may not be the only one, but it has a special and admirable
purity. If your motive is curiosity, you will assign priority to
questions according to how the questions, themselves, tickle your
personal aesthetic sense. A trickier challenge, with a greater
probability of failure, may be worth more effort than a simpler one,
just because it is more fun.}

{
 As I noted, people often think of rationality and emotion as
adversaries. Since curiosity is an emotion, I suspect that some people
will object to treating curiosity as a part of rationality. For my
part, I label an emotion as ``not
rational'' if it rests on mistaken beliefs, or
rather, on mistake-producing epistemic conduct: ``If
the iron approaches your face, and you believe it is hot, and it is
cool, the Way opposes your fear. If the iron approaches your face, and
you believe it is cool, and it is hot, the Way opposes your
calm.'' Conversely, then, an emotion that is evoked
by correct beliefs or epistemically rational thinking is a
``rational emotion''; and this has
the advantage of letting us regard calm as an emotional state, rather
than a privileged default.}

{
 When people think of
``emotion'' and
``rationality'' as opposed, I
suspect that they are really thinking of System 1 and System 2---fast
perceptual judgments versus slow deliberative judgments. Deliberative
judgments aren't always true, and perceptual judgments
aren't always false; so it is very important to
distinguish that dichotomy from
``rationality.'' Both systems can
serve the goal of truth, or defeat it, depending on how they are used.}

{
 Besides sheer emotional curiosity, what other motives are there
for desiring truth? Well, you might want to accomplish some specific
real-world goal, like building an airplane, and therefore you need to
know some specific truth about aerodynamics. Or more mundanely, you
want chocolate milk, and therefore you want to know whether the local
grocery has chocolate milk, so you can choose whether to walk there or
somewhere else. If this is the reason you want truth, then the priority
you assign to your questions will reflect the expected utility of their
information---how much the possible answers influence your choices, how
much your choices matter, and how much you expect to find an answer
that changes your choice from its default.}

{
 To seek truth merely for its instrumental value may seem
impure---should we not desire the truth for its own sake?---but such
investigations are extremely important because they create an outside
criterion of verification: if your airplane drops out of the sky, or if
you get to the store and find no chocolate milk, it's a
hint that you did something wrong. You get back feedback on which modes
of thinking work, and which don't. Pure curiosity is a
wonderful thing, but it may not linger too long on verifying its
answers, once the attractive mystery is gone. Curiosity, as a human
emotion, has been around since long before the ancient Greeks. But what
set humanity firmly on the path of Science was noticing that certain
modes of thinking uncovered beliefs that let us \textit{manipulate the
world.} As far as sheer curiosity goes, spinning campfire tales of gods
and heroes satisfied that desire just as well, and no one realized that
anything was wrong with that.}

{
 Are there motives for seeking truth besides curiosity and
pragmatism? The third reason that I can think of is morality: You
believe that to seek the truth is noble and important and worthwhile.
Though such an ideal also attaches an intrinsic value to truth,
it's a very different state of mind from curiosity.
Being curious about what's behind the curtain
doesn't feel the same as believing that you have a
moral duty to look there. In the latter state of mind, you are a lot
more likely to believe that someone \textit{else} should look behind
the curtain, too, or castigate them if they deliberately close their
eyes. For this reason, I would also label as
``morality'' the belief that
truthseeking is pragmatically important \textit{to society}, and
therefore is incumbent as a duty upon all. Your priorities, under this
motivation, will be determined by your ideals about which truths are
most important (not most useful or most intriguing), or about when,
under what circumstances, the duty to seek truth is at its strongest.}

{
 I tend to be suspicious of morality as a motivation for
rationality, \textit{not} because I reject the moral ideal, but because
it invites certain kinds of trouble. It is too easy to acquire, as
learned moral duties, modes of thinking that are dreadful missteps in
the dance. Consider Mr. Spock of \textit{Star Trek}, a naive archetype
of rationality. Spock's emotional state is always set
to ``calm,'' even when wildly
inappropriate. He often gives many significant digits for probabilities
that are grossly uncalibrated. (E.g., ``Captain, if
you steer the Enterprise directly into that black hole, our probability
of surviving is only 2.234\%.'' Yet nine times out of
ten the Enterprise is not destroyed. What kind of tragic fool gives
four significant digits for a figure that is off by two orders of
magnitude?) Yet this popular image is how many people conceive of the
duty to be ``rational''---small
wonder that they do not embrace it wholeheartedly. To make rationality
into a moral duty is to give it all the dreadful degrees of freedom of
an arbitrary tribal custom. People arrive at the wrong answer, and then
indignantly protest that they acted with propriety, rather than
learning from their mistake.}

{
 And yet if we're going to \textit{improve} our
skills of rationality, go beyond the standards of performance set by
hunter-gatherers, we'll need deliberate beliefs about
how to think with propriety. When we write new mental programs for
ourselves, they start out in System 2, the deliberate system, and are
only slowly---if ever---trained into the neural circuitry that
underlies System 1. So if there are certain kinds of thinking that we
find we want to \textit{avoid}{}---like, say, biases---it will end up
represented, within System 2, as an injunction not to think that way; a
professed duty of avoidance.}

{
 If we want the truth, we can most effectively obtain it by
thinking in certain ways, rather than others; these are the techniques
of rationality. And some of the techniques of rationality involve
overcoming a certain class of obstacles, the biases \ldots}

\myendsectiontext

\mysection{\ldots What's a Bias, Again?}

{
 A \textit{bias} is a certain kind of obstacle to our goal of
obtaining truth. (Its character as an
``obstacle'' stems from this goal of
truth.) However, there are many obstacles that are not
``biases.'' }

{
 If we start right out by asking ``What is
bias?,'' it comes at the question in the wrong order.
As the proverb goes, ``There are forty kinds of lunacy
but only one kind of common sense.'' The truth is a
narrow target, a small region of configuration space to hit.
``She loves me, she loves me not''
may be a binary question, but $E = mc^2$ is a tiny dot
in the space of all equations, like a winning lottery ticket in the
space of all lottery tickets. Error is not an exceptional condition; it
is success that is a priori so improbable that it requires an
explanation.}

{
 We don't start out with a moral duty to
``reduce bias,'' because biases are
bad and evil and Just Not Done. This is the sort of thinking someone
might end up with if they acquired a deontological duty of
``rationality'' by social osmosis,
which leads to people trying to execute techniques without appreciating
the reason for them. (Which is bad and evil and Just Not Done,
according to \textit{Surely You're Joking, Mr.
Feynman}, which I read as a kid.)}

{
 Rather, we want to get to the truth, for whatever reason, and we
find various obstacles getting in the way of our goal. These obstacles
are not wholly dissimilar to each other---for example, there are
obstacles that have to do with not having enough computing power
available, or information being expensive. It so happens that a large
group of obstacles seem to have a certain character in common---to
cluster in a region of obstacle-to-truth space---and this cluster has
been labeled ``biases.''}

{
 What is a bias? Can we look at the empirical cluster and find a
compact test for membership? Perhaps we will find that we
can't really give any explanation better than pointing
to a few extensional examples, and hoping the listener understands. If
you are a scientist just beginning to investigate fire, it might be a
lot wiser to point to a campfire and say ``Fire is
that orangey-bright hot stuff over there,'' rather
than saying ``I define fire as an alchemical
transmutation of substances which releases
phlogiston.'' You should not ignore something just
because you can't define it. I can't
quote the equations of General Relativity from memory, but nonetheless
if I walk off a cliff, I'll fall. And we can say the
same of biases---they won't hit any less hard if it
turns out we can't define compactly what a
``bias'' is. So we might point to
conjunction fallacies, to overconfidence, to the availability and
representativeness heuristics, to base rate neglect, and say:
``Stuff like that.''}

{
 With all that said, we seem to label as
``biases'' those obstacles to truth
which are produced, not by the cost of information, nor by limited
computing power, but by the shape of our own mental machinery. Perhaps
the machinery is evolutionarily optimized to purposes that actively
oppose epistemic accuracy; for example, the machinery to win arguments
in adaptive political contexts. Or the selection pressure ran skew to
epistemic accuracy; for example, believing what others believe, to get
along socially. Or, in the classic heuristic-and-bias, the machinery
operates by an identifiable algorithm that does some useful work but
also produces systematic errors: the availability heuristic is not
itself a bias, but it gives rise to identifiable, compactly describable
biases. Our brains are doing something wrong, and after a lot of
experimentation and/or heavy thinking, someone identifies the problem
in a fashion that System 2 can comprehend; then we call it a
``bias.'' Even if we can do no
better for knowing, it is still a failure that arises, in an
identifiable fashion, from a particular kind of cognitive
machinery---not from having too little machinery, but from the
machinery's shape.}

{
 ``Biases'' are distinguished
from errors that arise from cognitive content, such as adopted beliefs,
or adopted moral duties. These we call
``mistakes,'' rather than
``biases,'' and they are much easier
to correct, once we've noticed them for ourselves.
(Though the source of the mistake, or the source of the source of the
mistake, may ultimately be some bias.)}

{
 ``Biases'' are distinguished
from errors that arise from damage to an individual human brain, or
from absorbed cultural mores; biases arise from machinery that is
humanly universal.}

{
 Plato wasn't
``biased'' because he was ignorant
of General Relativity---he had no way to gather that information, his
ignorance did not arise from the shape of his mental machinery. But if
Plato believed that philosophers would make better kings because he
himself was a philosopher---and this belief, in turn, arose because of
a universal adaptive political instinct for self-promotion, and not
because Plato's daddy told him that everyone has a
moral duty to promote their own profession to governorship, or because
Plato sniffed too much glue as a kid---then that was a bias, whether
Plato was ever warned of it or not.}

{
 Biases may not be cheap to correct. They may not even be
correctable. But where we look upon our own mental machinery and see a
causal account of an identifiable class of errors; and when the problem
seems to come from the evolved shape of the machinery, rather from
there being too little machinery, or bad specific content; then we call
that a bias.}

{
 Personally, I see our quest in terms of acquiring personal skills
of rationality, in improving truthfinding technique. The challenge is
to attain the positive goal of truth, not to avoid the negative goal of
failure. Failurespace is wide, infinite errors in infinite variety. It
is difficult to describe so huge a space: ``What is
true of one apple may not be true of another apple; thus more can be
said about a single apple than about all the apples in the
world.'' Success-space is narrower, and therefore
more can be said about it.}

{
 While I am not averse (as you can see) to discussing definitions,
we should remember that is not our primary goal. We are here to pursue
the great human quest for truth: for we have desperate need of the
knowledge, and besides, we're curious. To this end let
us strive to overcome whatever obstacles lie in our way, whether we
call them ``biases'' or not.}

\myendsectiontext

\mysection{Availability}

{
 The \textit{availability heuristic} is judging the frequency or
probability of an event by the ease with which examples of the event
come to mind. }

{
 A famous 1978 study by Lichtenstein, Slovic, Fischhoff, Layman,
and Combs, ``Judged Frequency of Lethal
Events,'' studied errors in quantifying the severity
of risks, or judging which of two dangers occurred more
frequently.\footnote{Sarah Lichtenstein et al., ``Judged Frequency
of Lethal Events,'' \textit{Journal of Experimental
Psychology: Human Learning and Memory} 4, no. 6 (1978): 551--578,
doi:10.1037/0278-7393.4.6.551.\comment{1}} Subjects thought that accidents caused
about as many deaths as disease; thought that homicide was a more
frequent cause of death than suicide. Actually, diseases cause about
sixteen times as many deaths as accidents, and suicide is twice as
frequent as homicide.}

{
 An obvious hypothesis to account for these skewed beliefs is that
murders are more likely to be talked about than suicides---thus,
someone is more likely to recall hearing about a murder than hearing
about a suicide. Accidents are more dramatic than diseases---perhaps
this makes people more likely to remember, or more likely to recall, an
accident. In 1979, a followup study by Combs and Slovic showed that the
skewed probability judgments correlated strongly (0.85 and 0.89) with
skewed reporting frequencies in two newspapers.\footnote{Barbara Combs and Paul Slovic, ``Newspaper
Coverage of Causes of Death,'' \textit{Journalism \&
Mass Communication Quarterly} 56, no. 4 (1979): 837--849,
doi:10.1177/107769907905600420.\comment{2}} This
doesn't disentangle whether murders are more available
to memory because they are more reported-on, or whether newspapers
report more on murders because murders are more vivid (hence also more
remembered). But either way, an availability bias is at work. Selective
reporting is one major source of availability biases. In the ancestral
environment, much of what you knew, you experienced yourself; or you
heard it directly from a fellow tribe-member who had seen it. There was
usually at most one layer of selective reporting between you, and the
event itself. With today's Internet, you may see
reports that have passed through the hands of six bloggers on the way
to you---six successive filters. Compared to our ancestors, we live in
a larger world, in which far more happens, and far less of it reaches
us---a much stronger selection effect, which can create much larger
availability biases.}

{
 In real life, you're unlikely to ever meet Bill
Gates. But thanks to selective reporting by the media, you may be
tempted to compare your life success to his---and suffer hedonic
penalties accordingly. The objective frequency of Bill Gates is
0.00000000015, but you hear about him much more often. Conversely, 19\%
of the planet lives on less than \$1/day, and I doubt that one fifth of
the blog posts you read are written by them.}

{
 Using availability seems to give rise to an absurdity bias; events
that have never happened are not recalled, and hence deemed to have
probability zero. When no flooding has recently occurred (and yet the
probabilities are still fairly calculable), people refuse to buy flood
insurance even when it is heavily subsidized and priced far below an
actuarially fair value. Kunreuther et al.~suggest underreaction to
threats of flooding may arise from ``the inability of
individuals to conceptualize floods that have never occurred\,\ldots Men
on flood plains appear to be very much prisoners of their experience\,\ldots
Recently experienced floods appear to set an upward bound to the
size of loss with which managers believe they ought to be
concerned.''\footnote{Howard Kunreuther, Robin Hogarth, and Jacqueline Meszaros,
``Insurer Ambiguity and Market
Failure,'' \textit{Journal of Risk and Uncertainty} 7
(1 1993): 71--87, doi:10.1007/BF01065315.\comment{3}}}

{
 Burton et al.~report that when dams and levees are built, they
reduce the frequency of floods, and thus apparently create a false
sense of security, leading to reduced precautions.\footnote{Ian Burton, Robert W. Kates, and Gilbert F. White, \textit{The
Environment as Hazard}, 1st ed. (New York: Oxford University Press,
1978).\comment{4}}
While building dams decreases the \textit{frequency} of floods, damage
\textit{per flood} is afterward so much greater that average yearly
damage \textit{increases.} The wise would extrapolate from a memory of
small hazards to the possibility of large hazards. Instead, past
experience of small hazards seems to set a perceived upper bound on
risk. A society well-protected against minor hazards takes no action
against major risks, building on flood plains once the regular minor
floods are eliminated. A society subject to regular minor hazards
treats those minor hazards as an upper bound on the size of the risks,
guarding against regular minor floods but not occasional major floods.}

{
 Memory is not always a good guide to probabilities in the past,
let alone in the future.}

\myendsectiontext


\bigskip

\mysection{Burdensome Details}
\label{burdensome_details}

\begin{quote}
{
 Merely corroborative detail, intended to give artistic
verisimilitude to an otherwise bald and unconvincing narrative\,\ldots}

{\raggedleft
 {}---Pooh-Bah, in Gilbert and Sullivan's
\textit{The Mikado}\footnote{William S. Gilbert and Arthur Sullivan, \textit{The Mikado},
Opera, 1885.\comment{1}}
\par}
\end{quote}


{
 The conjunction fallacy is when humans rate the probability $P(A,B)$
higher than the probability $P(B)$, even though it is a theorem that
$P(A,B) \leq P(B)$. For example, in one experiment in 1981, 68\% of
the subjects ranked it more likely that ``Reagan will
provide federal support for unwed mothers and cut federal support to
local governments'' than that
``Reagan will provide federal support for unwed
mothers.''}

{
 A long series of cleverly designed experiments, which weeded out
alternative hypotheses and nailed down the standard interpretation,
confirmed that conjunction fallacy occurs because we
``substitute judgment of representativeness for
judgment of probability.'' By adding extra details,
you can make an outcome seem \textit{more} characteristic of the
process that generates it. You can make it sound more plausible that
Reagan will support unwed mothers, by \textit{adding} the claim that
Reagan will \textit{also} cut support to local governments. The
implausibility of one claim is compensated by the plausibility of the
other; they ``average out.''}

{
 Which is to say: Adding detail can make a scenario \textsc{sound more
plausible}, even though the event necessarily \textsc{becomes less probable}.}

{
 If so, then, \textit{hypothetically speaking,} we might find
futurists spinning unconscionably plausible and detailed future
histories, or find people swallowing huge packages of unsupported
claims bundled with a few strong-sounding assertions at the center. If
you are presented with the conjunction fallacy in a naked, direct
comparison, then you may succeed on that particular problem by
consciously correcting yourself. But this is only slapping a band-aid
on the problem, not fixing it in general.}

{
 In the 1982 experiment where professional forecasters assigned
systematically higher probabilities to ``Russia
invades Poland, followed by suspension of diplomatic relations between
the USA and the USSR'' than to
``Suspension of diplomatic relations between the USA
and the USSR,'' each experimental group was only
presented with one proposition.\footnote{Tversky and Kahneman, ``Extensional Versus
Intuitive Reasoning.''\comment{2}} What strategy could
these forecasters have followed, as a group, that would have eliminated
the conjunction fallacy, when no individual knew directly about the
comparison? When no individual even knew that the experiment was
\textit{about} the conjunction fallacy? How could they have done better
on their probability judgments?}

{
 Patching one gotcha as a special case doesn't fix
the general problem. The gotcha is the symptom, not the disease.}

{
 What could the forecasters have done to avoid the conjunction
fallacy, without seeing the direct comparison, or even knowing that
anyone was going to test them on the conjunction fallacy? It seems to
me, that they would need to notice the word
``and.'' They would need to be wary
of it---not just wary, but leap back from it. Even without knowing that
researchers were afterward going to test them on the conjunction
fallacy particularly. They would need to notice the conjunction of
\textit{two entire details}, and be \textit{shocked} by the audacity of
anyone asking them to endorse such an insanely complicated prediction.
And they would need to penalize the probability
\textit{substantially}{}---a factor of four, at least, according to the
experimental details.}

{
 It might also have helped the forecasters to think about possible
reasons why the US and Soviet Union would suspend diplomatic relations.
The scenario is not ``The US and Soviet Union suddenly
suspend diplomatic relations for no reason,'' but
``The US and Soviet Union suspend diplomatic relations
for any reason.''}

{
 And the subjects who rated ``Reagan will provide
federal support for unwed mothers and cut federal support to local
governments''? Again, they would need to be shocked
by the word ``and.'' Moreover, they
would need to \textit{add} absurdities---where the absurdity is the log
probability, so you can add it---rather than averaging them. They would
need to think, ``Reagan might or might not cut support
to local governments (1 bit), but it seems very unlikely that he will
support unwed mothers (4 bits). \textit{Total} absurdity: 5
bits.'' Or maybe, ``Reagan
won't support unwed mothers. One strike and
it's out. The other proposition just makes it even
worse.''}

{
 Similarly, consider the six-sided die with four green faces and
two red faces. The subjects had to bet on the sequence (1) \textsc{rgrrr}, (2)
\textsc{grgrrr}, or (3) \textsc{grrrrr} appearing anywhere in twenty rolls of the
dice.\footnote{Amos Tversky and Daniel Kahneman, ``Judgments
of and by Representativeness,'' in \textit{Judgment
Under Uncertainty: Heuristics and Biases}, ed. Daniel Kahneman, Paul
Slovic, and Amos Tversky (New York: Cambridge University Press, 1982),
84--98.\comment{3}} Sixty-five percent of the subjects chose
\textsc{grgrrr}, which is strictly dominated by \textsc{rgrrr}, since any sequence
containing \textsc{grgrrr} also pays off for \textsc{rgrrr}. How could the subjects have
done better? By noticing the inclusion? Perhaps; but that is only a
band-aid, it does not fix the fundamental problem. By explicitly
calculating the probabilities? That would certainly fix the fundamental
problem, but you can't always calculate an exact
probability.}

{
 The subjects lost heuristically by thinking:
``Aha! Sequence 2 has the highest proportion of green
to red! I should bet on Sequence 2!'' To win
heuristically, the subjects would need to think:
``Aha! Sequence 1 is \textit{short}! I should go with
Sequence 1!''}

{
 They would need to feel a stronger \textit{emotional impact} from
Occam's Razor---feel \textit{every} added detail as a
burden, even a single extra roll of the dice.}

{
 Once upon a time, I was speaking to someone who had been
mesmerized by an incautious futurist (one who adds on lots of details
that sound neat). I was trying to explain why I was not likewise
mesmerized by these amazing, incredible theories. So I explained about
the conjunction fallacy, specifically the ``suspending
relations {\textpm} invading Poland'' experiment. And
he said, ``Okay, but what does this have to do
with---'' And I said, ``It is more
probable that universes replicate \textit{for any reason}, than that
they replicate \textit{via black holes because advanced civilizations
manufacture black holes because universes evolve to make them do
it}.'' And he said,
``Oh.''}

{
 Until then, he had not felt these extra details as extra burdens.
Instead they were corroborative detail, lending verisimilitude to the
narrative. Someone presents you with a package of strange ideas,
\textit{one} of which is that universes replicate. Then they present
support \textit{for the assertion that universes replicate.} But this
is not support for the package, though it is all told as one story.}

{
 You have to disentangle the details. You have to hold up every one
independently, and ask, ``How do we know \textit{this}
detail?'' Someone sketches out a picture of
humanity's descent into nanotechnological warfare,
where China refuses to abide by an international control agreement,
followed by an arms race\,\ldots Wait a minute---how do you know it will
be China? Is that a crystal ball in your pocket or are you just happy
to be a futurist? Where are all these details coming from? Where did
\textit{that specific} detail come from?}

{
 For it is written:\footnote{Eliezer Yudkowsky, {\em Rationality From AI to Zombies}, pg \pageref{seventh_virtue}}}

\begin{quote}
{
 \textit{If you can lighten your burden you must do so.}}

{
 \textit{There is no straw that lacks the power to break your
   back.}}
\end{quote}

\myendsectiontext


\bigskip

\mysection{Planning Fallacy}

{
 The Denver International Airport opened 16 months late, at a cost
overrun of \$2 billion. (I've also seen \$3.1 billion
asserted.) The Eurofighter Typhoon, a joint defense project of several
European countries, was delivered 54 months late at a cost of \$19
billion instead of \$7 billion. The Sydney Opera House may be the most
legendary construction overrun of all time, originally estimated to be
completed in 1963 for \$7 million, and finally completed in 1973 for
\$102 million.\footnote{Roger Buehler, Dale Griffin, and Michael Ross,
``Inside the Planning Fallacy: The Causes and
Consequences of Optimistic Time Predictions,'' in
Gilovich, Griffin, and Kahneman, \textit{Heuristics and Biases},
250--270.\comment{1}} }

{
 Are these isolated disasters brought to our attention by selective
availability? Are they symptoms of bureaucracy or government incentive
failures? Yes, very probably. But there's also a
corresponding cognitive bias, replicated in experiments with individual
planners.}

{
 Buehler et al.~asked their students for estimates of when they
(the students) thought they would complete their personal academic
projects.\footnote{Roger Buehler, Dale Griffin, and Michael Ross,
``Exploring the `Planning
Fallacy': Why People Underestimate Their Task
Completion Times,'' \textit{Journal of Personality
and Social Psychology} 67, no. 3 (1994): 366--381,
doi:10.1037/0022-3514.67.3.366; Roger Buehler, Dale Griffin, and
Michael Ross, ``It's About Time:
Optimistic Predictions in Work and Love,''
\textit{European Review of Social Psychology} 6, no. 1 (1995): 1--32,
doi:10.1080/14792779343000112.\comment{2}} Specifically, the researchers asked for
estimated times by which the students thought it was 50\%, 75\%, and
99\% probable their personal projects would be done. Would you care to
guess how many students finished on or before their estimated 50\%,
75\%, and 99\% probability levels?}

\begin{itemize}
\item {
 13\% of subjects finished their project by the time they had
assigned a 50\% probability level;}

\item {
 19\% finished by the time assigned a 75\% probability level;}

\item {
 and only 45\% (less than half!) finished by the time of their 99\%
 probability level.}
\end{itemize}

{
 As Buehler et al.~wrote, ``The results for the
99\% probability level are especially striking: Even when asked to make
a highly conservative forecast, a prediction that they felt virtually
certain that they would fulfill, students' confidence
in their time estimates far exceeded their
accomplishments.''\footnote{Buehler, Griffin, and Ross, ``Inside the
Planning Fallacy.''\comment{3}}}

{
 More generally, this phenomenon is known as the
``planning fallacy.'' The planning
fallacy is that people think they can plan, ha ha.}

{
 A clue to the underlying problem with the planning algorithm was
uncovered by Newby-Clark et al., who found that}

\begin{itemize}
\item {
 Asking subjects for their predictions based on realistic
``best guess'' scenarios; and}

\item {
 Asking subjects for their hoped-for ``best
 case'' scenarios\,\ldots}
\end{itemize}

{
 \ldots produced \textit{indistinguishable}
results.\footnote{Ian R. Newby-Clark et al., ``People Focus on
Optimistic Scenarios and Disregard Pessimistic Scenarios While
Predicting Task Completion Times,'' \textit{Journal
of Experimental Psychology: Applied} 6, no. 3 (2000): 171--182,
doi:10.1037/1076-898X.6.3.171.\comment{4}}}

{
 When people are asked for a
``realistic'' scenario, they
envision everything going exactly as planned, with no
\textit{unexpected} delays or \textit{unforeseen} catastrophes---the
same vision as their ``best case.''}

{
 Reality, it turns out, usually delivers results somewhat worse
than the ``worst case.''}

{
 Unlike most cognitive biases, we know a good debiasing heuristic
for the planning fallacy. It won't work for messes on
the scale of the Denver International Airport, but
it'll work for a lot of personal planning, and even
some small-scale organizational stuff. Just use an
``outside view'' instead of an
``inside view.''}

{
 People tend to generate their predictions by thinking about the
particular, unique features of the task at hand, and constructing a
scenario for how they intend to complete the task---which is just what
we usually think of as \textit{planning.} When you want to get
something done, you have to plan out where, when, how; figure out how
much time and how much resource is required; visualize the steps from
beginning to successful conclusion. All this is the
``inside view,'' and it
doesn't take into account unexpected delays and
unforeseen catastrophes. As we saw before, asking people to visualize
the ``worst case'' still
isn't enough to counteract their optimism---they
don't visualize enough Murphyness.}

{
 The outside view is when you deliberately \textit{avoid} thinking
about the special, unique features of this project, and just ask how
long it took to finish \textit{broadly} similar projects in the past.
This is counterintuitive, since the inside view has so much more
detail---there's a temptation to think that a carefully
tailored prediction, taking into account all available data, will give
better results.}

{
 But experiment has shown that the more detailed
subjects' visualization, the more optimistic (and less
accurate) they become. Buehler et al.~asked an experimental group of
subjects to describe highly specific plans for their Christmas
shopping---where, when, and how.\footnote{Buehler, Griffin, and Ross, ``Inside the
Planning Fallacy.''\comment{5}} On average, this
group expected to finish shopping more than a week before Christmas.
Another group was simply asked when they expected to finish their
Christmas shopping, with an average response of four days. Both groups
finished an average of three days before Christmas.}

{
 Likewise, Buehler et al., reporting on a cross-cultural study,
found that Japanese students expected to finish their essays ten days
before deadline. They actually finished one day before deadline. Asked
when they had previously completed similar tasks, they responded,
``one day before
deadline.''\footnote{Ibid.\comment{6}} This is the power of
the outside view over the inside view.}

{
 A similar finding is that experienced outsiders, who know less of
the details, but who have relevant memory to draw upon, are often much
less optimistic and much more accurate than the actual planners and
implementers.}

{
 So there is a fairly reliable way to fix the planning fallacy, if
you're doing something \textit{broadly} similar to a
reference class of previous projects. Just ask how long similar
projects have taken in the past, without considering \textit{any} of
the special properties of this project. Better yet, ask an experienced
outsider how long similar projects have taken.}

{
 You'll get back an answer that sounds hideously
long, and clearly reflects no understanding of the special reasons why
this particular task will take less time. This answer is true. Deal
with it.}

\myendsectiontext


\bigskip

\mysection{Illusion of Transparency: Why No One Understands You}
\label{illusion_of_transparency}

{
 In hindsight bias, people who know the outcome of a situation
believe the outcome should have been easy to predict in advance.
Knowing the outcome, we reinterpret the situation in light of that
outcome. Even when warned, we can't de-interpret to
empathize with someone who doesn't know what we know. }

{
 Closely related is the \textit{illusion of transparency}: We
always know what \textit{we} mean by our words, and so we expect others
to know it too. Reading our own writing, the intended interpretation
falls easily into place, guided by our knowledge of what we really
meant. It's hard to empathize with someone who must
interpret blindly, guided only by the words.}

{
 June recommends a restaurant to Mark; Mark dines there and
discovers (a) unimpressive food and mediocre service or (b) delicious
food and impeccable service. Then Mark leaves the following message on
June's answering machine: ``June, I
just finished dinner at the restaurant you recommended, and I must say,
it was marvelous, just marvelous.'' Keysar presented
a group of subjects with scenario (a), and 59\% thought that
Mark's message was sarcastic \textit{and that Jane
would perceive the sarcasm}.\footnote{Boaz Keysar, ``The Illusory Transparency of
Intention: Linguistic Perspective Taking in Text,''
\textit{Cognitive Psychology} 26 (2 1994): 165--208,
doi:10.1006/cogp.1994.1006.\comment{1}} Among other subjects,
told scenario (b), only 3\% thought that Jane would perceive
Mark's message as sarcastic. Keysar and Barr seem to
indicate that an actual voice message was played back to the
subjects.\footnote{Keysar and Barr, ``Self-Anchoring in
Conversation.''\comment{2}} Keysar showed that if subjects were told
that the restaurant was horrible \textit{but that Mark wanted to
conceal his response}, they believed June would not perceive sarcasm in
the (same) message:\footnote{Boaz Keysar, ``Language Users as Problem
Solvers: Just What Ambiguity Problem Do They
Solve?,'' in \textit{Social and Cognitive Approaches
to Interpersonal Communication}, ed. Susan R. Fussell and Roger J.
Kreuz (Mahwah, NJ: Lawrence Erlbaum Associates, 1998), 175--200.\comment{3}}}

\begin{quote}
{
 They were just as likely to predict that she would perceive
sarcasm when he attempted to conceal his negative experience as when he
had a positive experience and was truly sincere. So participants took
Mark's \textit{communicative intention} as transparent.
It was as if they assumed that June would perceive whatever intention
Mark wanted her to perceive.\footnote{Keysar and Barr, ``Self-Anchoring in
  Conversation.''\comment{4}}}
\end{quote}

{
 ``The goose hangs high'' is an
archaic English idiom that has passed out of use in modern language.
Keysar and Bly told one group of subjects that ``the
goose hangs high'' meant that the future looks good;
another group of subjects learned that ``the goose
hangs high'' meant the future looks
gloomy.\footnote{Boaz Keysar and Bridget Bly, ``Intuitions of
the Transparency of Idioms: Can One Keep a Secret by Spilling the
Beans?,'' \textit{Journal of Memory and Language} 34
(1 1995): 89--109, doi:10.1006/jmla.1995.1005.\comment{5}} Subjects were then asked which of these two
meanings an \textit{uninformed} listener would be more likely to
attribute to the idiom. Each group thought that listeners would
perceive the meaning presented as
``standard.''}

{
 (Other idioms tested included ``come the uncle
over someone,'' ``to go by the
board,'' and ``to lay out in
lavender.'' Ah, English, such a lovely language.)}

{
 Keysar and Henly tested the calibration of speakers: Would
speakers underestimate, overestimate, or correctly estimate how often
listeners understood them?\footnote{Boaz Keysar and Anne S. Henly,
``Speakers' Overestimation of Their
Effectiveness,'' \textit{Psychological Science} 13 (3
2002): 207--212, doi:10.1111/1467-9280.00439.\comment{6}} Speakers were given
ambiguous sentences (``The man is chasing a woman on a
bicycle.'') and disambiguating pictures (a man
running after a cycling woman), then asked the speakers to utter the
words in front of addressees, then asked speakers to estimate how many
addressees understood the intended meaning. Speakers thought that they
were understood in 72\% of cases and were actually understood in 61\%
of cases. When addressees did not understand, speakers thought they did
in 46\% of cases; when addressees did understand, speakers thought they
did not in only 12\% of cases.}

{
 Additional subjects who \textit{overheard} the explanation showed
no such bias, expecting listeners to understand in only 56\% of cases.}

{
 As Keysar and Barr note, two days before Germany's
attack on Poland, Chamberlain sent a letter intended to make it clear
that Britain would fight if any invasion occurred.\footnote{Keysar and Barr, ``Self-Anchoring in
Conversation.''\comment{7}}
The letter, phrased in polite diplomatese, was heard by Hitler as
conciliatory---and the tanks rolled.}

{
 Be not too quick to blame those who misunderstand your perfectly
clear sentences, spoken or written. Chances are, your words are more
ambiguous than you think.}

\myendsectiontext


\bigskip

\mysection{Expecting Short Inferential Distances}
\label{inferential_distances}

{
 \textit{Homo sapiens}'s environment of
evolutionary adaptedness (a.k.a. EEA or ``ancestral
environment'') consisted of hunter-gatherer bands of
at most 200 people, with no writing. All inherited knowledge was passed
down by speech and memory. }

{
 In a world like that, all background knowledge is universal
knowledge. All information not strictly private is public, period.}

{
 In the ancestral environment, you were unlikely to end up more
than \textit{one inferential step} away from anyone else. When you
discover a new oasis, you don't have to explain to your
fellow tribe members what an oasis is, or why it's a
good idea to drink water, or how to walk. Only you know where the oasis
lies; this is private knowledge. But everyone has the background to
understand your description of the oasis, the concepts needed to think
about water; this is universal knowledge. When you explain things in an
ancestral environment, you almost \textit{never} have to explain your
concepts. At most you have to explain \textit{one} new concept, not two
or more simultaneously.}

{
 In the ancestral environment there were no abstract disciplines
with vast bodies of carefully gathered evidence generalized into
elegant theories transmitted by written books whose conclusions are
\textit{a hundred inferential steps removed} from universally shared
background premises.}

{
 In the ancestral environment, anyone who says something with no
obvious support is a liar or an idiot. You're not
likely to think, ``Hey, maybe this person has
well-supported background knowledge that no one in my band has even
heard of,'' because it was a reliable invariant of
the ancestral environment that this didn't happen.}

{
 Conversely, if you say something blatantly obvious and the other
person doesn't see it, \textit{they're}
the idiot, or they're being deliberately obstinate to
annoy you.}

{
 And to top it off, if someone says something with no obvious
support and \textit{expects} you to believe it---acting all indignant
when you don't---then they must be \textit{crazy.}}

{
 Combined with the illusion of transparency and self-anchoring, I
think this explains a \textit{lot} about the legendary difficulty most
scientists have in communicating with a lay audience---or even
communicating with scientists from other disciplines. When I observe
failures of explanation, I usually see the explainer taking
\textit{one} step back, when they need to take two or more steps back.
Or listeners assume that things should be visible in one step, when
they take two or more steps to explain. Both sides act as if they
expect very short inferential distances from universal knowledge to any
new knowledge.}

{
 A biologist, speaking to a physicist, can justify evolution by
saying it is the simplest explanation. But not everyone on Earth has
been inculcated with that legendary history of science, from Newton to
Einstein, which invests the phrase ``simplest
explanation'' with its awesome import: a Word of
Power, spoken at the birth of theories and carved on their tombstones.
To someone else, ``But it's the
simplest explanation!'' may sound like an interesting
but hardly knockdown argument; it doesn't feel like all
that powerful a tool for comprehending office politics or fixing a
broken car. Obviously the biologist is infatuated with their own ideas,
too arrogant to be open to alternative explanations which sound just as
plausible. (If it sounds plausible to me, it should sound plausible to
any sane member of my band.)}

{
 And from the biologist's perspective, they can
understand how evolution might sound a little odd at first---but when
someone rejects evolution even after the biologist explains that
it's the simplest explanation, well,
it's clear that nonscientists are just idiots and
there's no point in talking to them.}

{
 A clear argument has to lay out an inferential \textit{pathway},
starting from what the audience \textit{already knows or accepts.} If
you don't recurse far enough, you're
just talking to yourself.}

{
 If at any point you make a statement without obvious justification
in arguments you've previously supported, the audience
just thinks you're crazy.}

{
 This also happens when you allow yourself to be seen
\textit{visibly} attaching greater weight to an argument than is
justified in the eyes of the audience \textit{at that time}. For
example, talking as if you think ``simpler
explanation'' is a knockdown argument for evolution
(which it is), rather than a sorta-interesting idea (which it sounds
like to someone who hasn't been raised to revere
Occam's Razor).}

{
 Oh, and you'd better not drop any hints that
\textit{you} think you're working a dozen inferential
steps away from what the audience knows, or that \textit{you} think you
have special background knowledge not available to them. The audience
doesn't know anything about an
evolutionary-psychological argument for a cognitive bias to
underestimate inferential distances leading to traffic jams in
communication. They'll just think
you're condescending.}

{
 And if you think you can explain the concept of
``systematically underestimated inferential
distances'' briefly, in just a few words,
I've got some sad news for you\,\ldots}

\myendsectiontext

\mysection{The Lens That Sees Its Own Flaws}

{
 Light leaves the Sun and strikes your shoelaces and bounces off;
some photons enter the pupils of your eyes and strike your retina; the
energy of the photons triggers neural impulses; the neural impulses are
transmitted to the visual-processing areas of the brain; and there the
optical information is processed and reconstructed into a 3D model that
is recognized as an untied shoelace; and so you believe that your
shoelaces are untied. }

{
 Here is the secret of \textit{deliberate rationality---}this whole
process is not magic, and you can \textit{understand} it. You can
\textit{understand} how you see your shoelaces. You can \textit{think}
about which sort of thinking processes will create beliefs which mirror
reality, and which thinking processes will not.}

{
 Mice can see, but they can't understand seeing.
\textit{You} can understand seeing, and because of that, you can do
things that mice cannot do. Take a moment to marvel at this, for it is
indeed marvelous.}

{
 Mice see, but they don't know they have visual
cortexes, so they can't correct for optical illusions.
A mouse lives in a mental world that includes cats, holes, cheese and
mousetraps---but not mouse brains. Their camera does not take pictures
of its own lens. But we, as humans, can look at a seemingly bizarre
image, and realize that part of what we're seeing is
the lens itself. You don't always have to believe your
own eyes, but you have to realize that you \textit{have} eyes---you
must have distinct mental buckets for the map and the territory, for
the senses and reality. Lest you think this a trivial ability, remember
how rare it is in the animal kingdom.}

{
 The whole idea of Science is, simply, reflective reasoning about a
more reliable process for making the contents of your mind mirror the
contents of the world. It is the sort of thing mice would never invent.
Pondering this business of ``performing replicable
experiments to falsify theories,'' we can see
\textit{why} it works. Science is not a separate magisterium, far away
from real life and the understanding of ordinary mortals. Science is
not something that only applies to the inside of laboratories. Science,
itself, is an understandable process-in-the-world that correlates
brains with reality.}

{
 Science \textit{makes sense}, when you think about it. But mice
can't think about thinking, which is why they
don't have Science. One should not overlook the wonder
of this---or the potential power it bestows on us as individuals, not
just scientific societies.}

{
 Admittedly, understanding the engine of thought may be \textit{a
little more complicated} than understanding a steam engine---but it is
not a \textit{fundamentally} different task.}

{
 Once upon a time, I went to EFNet's \#philosophy
chatroom to ask, ``Do you believe a nuclear war will
occur in the next 20 years? If no, why not?'' One
person who answered the question said he didn't expect
a nuclear war for 100 years, because ``All of the
players involved in decisions regarding nuclear war are not interested
right now.'' ``But why extend that
out for 100 years?'' I asked. ``Pure
hope,'' was his reply.}

{
 Reflecting on this whole thought process, we can see why the
thought of nuclear war makes the person unhappy, and we can see how his
brain therefore rejects the belief. But if you imagine a billion
worlds---Everett branches, or Tegmark
duplicates\footnote{Max Tegmark, ``Parallel
Universes,'' in \textit{Science and Ultimate Reality:
Quantum Theory, Cosmology, and Complexity}, ed. John D. Barrow, Paul C.
W. Davies, and Charles L. Harper Jr. (New York: Cambridge University
Press, 2004), 459--491.\comment{1}}{}---this thought process will not
systematically correlate optimists to branches in which no nuclear war
occurs. (Some clever fellow is bound to say, ``Ah, but
since I have hope, I'll work a little harder at my job,
pump up the global economy, and thus help to prevent countries from
sliding into the angry and hopeless state where nuclear war is a
possibility. So the two events are related after
all.'' At this point, we have to drag in
Bayes's Theorem and measure the relationship
quantitatively. Your optimistic nature cannot have \textit{that} large
an effect on the world; it cannot, of itself, decrease the probability
of nuclear war by 20\%, or however much your optimistic nature shifted
your beliefs. Shifting your beliefs by a large amount, due to an event
that only slightly increases your chance of being right, will still
mess up your mapping.)}

{
 To ask which beliefs make you happy is to turn inward, not
outward---it tells you something about yourself, but it is not evidence
entangled with the environment. I have nothing against happiness, but
it should follow from your picture of the world, rather than tampering
with the mental paintbrushes.}

{
 If you can see this---if you can see that hope is shifting your
\textit{first-order} thoughts by too large a degree---if you can
understand your mind as a mapping engine that has flaws---then you can
apply a reflective correction. The brain is a flawed lens through which
to see reality. This is true of both mouse brains and human brains. But
a human brain is a flawed lens that can understand its own flaws---its
systematic errors, its biases---and apply second-order corrections to
them. This, \textit{in practice,} makes the lens far more powerful. Not
perfect, but far more powerful.}

\myendsectiontext


\bigskip


\chapter{Fake Beliefs}

\mysection{Making Beliefs Pay Rent (in Anticipated Experiences)}
\label{making_beliefs_pay_rent}

{
 Thus begins the ancient parable: }

{
 \textit{If a tree falls in a forest and no one hears it, does it
make a sound? One says, ``Yes it does, for it makes
vibrations in the air.'' Another says,
``No it does not, for there is no auditory processing
in any brain.''}}

{
 Suppose that, after the tree falls, the two walk into the forest
together. Will one expect to see the tree fallen to the right, and the
other expect to see the tree fallen to the left? Suppose that before
the tree falls, the two leave a sound recorder next to the tree. Would
one, playing back the recorder, expect to hear something different from
the other? Suppose they attach an electroencephalograph to any brain in
the world; would one expect to see a different trace than the other?
Though the two argue, one saying
``No,'' and the other saying
``Yes,'' they do not anticipate any
different experiences. The two think they have different models of the
world, but they have no difference with respect to what they expect
will \textit{happen to} them.}

{
 It's tempting to try to eliminate this mistake
class by insisting that the only legitimate kind of belief is an
anticipation of sensory experience. But the world does, in fact,
contain much that is not sensed directly. We don't see
the atoms underlying the brick, but the atoms are in fact there. There
is a floor beneath your feet, but you don't
\textit{experience} the floor directly; you see the light
\textit{reflected} from the floor, or rather, you see what your retina
and visual cortex have processed of that light. To infer the floor from
seeing the floor is to step back into the unseen causes of experience.
It may seem like a very short and direct step, but it is still a step.}

{
 You stand on top of a tall building, next to a grandfather clock
with an hour, minute, and ticking second hand. In your hand is a
bowling ball, and you drop it off the roof. On which tick of the clock
will you hear the crash of the bowling ball hitting the ground?}

{
 To answer precisely, you must use beliefs like
\textit{Earth's gravity is 9.8 meters per second per
second,} and \textit{This building is around 120 meters tall.} These
beliefs are not wordless anticipations of a sensory experience; they
are verbal-ish, propositional. It probably does not exaggerate much to
describe these two beliefs as sentences made out of words. But these
two beliefs have an inferential \textit{consequence} that is a direct
sensory anticipation---if the clock's second hand is on
the 12 numeral when you drop the ball, you anticipate seeing it on the
1 numeral when you hear the crash five seconds later. To anticipate
sensory experiences as precisely as possible, we must process beliefs
that are not anticipations of sensory experience.}

{
 It is a great strength of \textit{Homo sapiens} that we can,
better than any other species in the world, learn to model the unseen.
It is also one of our great weak points. Humans often believe in things
that are not only unseen but unreal.}

{
 The same brain that builds a network of inferred causes behind
sensory experience can also build a network of causes that is not
connected to sensory experience, or poorly connected. Alchemists
believed that phlogiston caused fire---we could oversimply their minds
by drawing a little node labeled
``Phlogiston,'' and an arrow from
this node to their sensory experience of a crackling campfire---but
this belief yielded no advance predictions; the link from phlogiston to
experience was always configured after the experience, rather than
constraining the experience in advance. Or suppose your postmodern
English professor teaches you that the famous writer Wulky Wilkinsen is
actually a ``post-utopian.'' What
does this mean you should expect from his books? Nothing. The belief,
if you can call it that, doesn't connect to sensory
experience at all. But you had better remember the propositional
assertion that ``Wulky Wilkinsen''
has the ``post-utopian'' attribute,
so you can regurgitate it on the upcoming quiz. Likewise if
``post-utopians'' show
``colonial alienation''; if the quiz
asks whether Wulky Wilkinsen shows colonial alienation,
you'd better answer yes. The beliefs are connected to
each other, though still not connected to any anticipated experience.}

{
 We can build up whole networks of beliefs that are connected only
to each other---call these
``floating'' beliefs. It is a
uniquely human flaw among animal species, a perversion of \textit{Homo
sapiens}'s ability to build more general and flexible
belief networks.}

{
 The rationalist virtue of \textit{empiricism} consists of
constantly asking which experiences our beliefs predict---or better
yet, prohibit. Do you believe that phlogiston is the cause of fire?
Then what do you expect to see happen, because of that? Do you believe
that Wulky Wilkinsen is a post-utopian? Then what do you expect to see
because of that? No, not ``colonial
alienation''; \textit{what experience will happen to
you?} Do you believe that if a tree falls in the forest, and no one
hears it, it still makes a sound? Then what experience must therefore
befall you?}

{
 It is even better to ask: what experience \textit{must not} happen
to you? Do you believe that \textit{élan vital} explains the mysterious
aliveness of living beings? Then what does this belief \textit{not}
allow to happen---what would definitely falsify this belief? A null
answer means that your belief does not \textit{constrain} experience;
it permits \textit{anything} to happen to you. It floats.}

{
 When you argue a seemingly factual question, always keep in mind
which difference of anticipation you are arguing about. If you
can't find the difference of anticipation,
you're probably arguing about labels in your belief
network---or even worse, floating beliefs, barnacles on your network.
If you don't know what experiences are implied by Wulky
Wilkinsen being a post-utopian, you can go on arguing forever.}

{
 Above all, don't ask what to believe---ask what to
anticipate. Every question of belief should flow from a question of
anticipation, and that question of anticipation should be the center of
the inquiry. Every guess of belief should begin by flowing to a
specific guess of anticipation, and should continue to pay rent in
future anticipations. If a belief turns deadbeat, evict it.}

\myendsectiontext

\mysection{A Fable of Science and Politics}
\label{a_fable_of_science_and_politics}

{
 In the time of the Roman Empire, civic life was divided between
the Blue and Green factions. The Blues and the Greens murdered each
other in single combats, in ambushes, in group battles, in riots.
Procopius said of the warring factions: ``So there
grows up in them against their fellow men a hostility which has no
cause, and at no time does it cease or disappear, for it gives place
neither to the ties of marriage nor of relationship nor of friendship,
and the case is the same even though those who differ with respect to
these colors be brothers or any other
kin.''\footnote{Procopius, \textit{History of the Wars}, ed. Henry B. Dewing,
vol. 1 (Harvard University Press, 1914).\comment{1}} Edward Gibbon wrote:
``The support of a faction became necessary to every
candidate for civil or ecclesiastical
honors.''\footnote{Edward Gibbon, \textit{The History of the Decline and Fall of
the Roman Empire}, vol. 4 (J. \& J. Harper, 1829).\comment{2}} }

{
 Who were the Blues and the Greens? They were sports fans---the
partisans of the blue and green chariot-racing teams.}

{
 Imagine a future society that flees into a vast underground
network of caverns and seals the entrances. We shall not specify
whether they flee disease, war, or radiation; we shall suppose the
first Undergrounders manage to grow food, find water, recycle air, make
light, and survive, and that their descendants thrive and eventually
form cities. Of the world above, there are only legends written on
scraps of paper; and one of these scraps of paper describes the
\textit{sky}, a vast open space of air above a great unbounded floor.
The sky is cerulean in color, and contains strange floating objects
like enormous tufts of white cotton. But the meaning of the word
``cerulean'' is controversial; some
say that it refers to the color known as
``blue,'' and others that it refers
to the color known as ``green.''}

{
 In the early days of the underground society, the Blues and Greens
contested with open violence; but today, truce prevails---a peace born
of a growing sense of pointlessness. Cultural mores have changed; there
is a large and prosperous middle class that has grown up with effective
law enforcement and become unaccustomed to violence. The schools
provide some sense of historical perspective; how long the battle
between Blues and Greens continued, how many died, how little changed
as a result. Minds have been laid open to the strange new philosophy
that people are people, whether they be Blue or Green.}

{
 The conflict has not vanished. Society is still divided along Blue
and Green lines, and there is a
``Blue'' and a
``Green'' position on almost every
contemporary issue of political or cultural importance. The Blues
advocate taxes on individual incomes, the Greens advocate taxes on
merchant sales; the Blues advocate stricter marriage laws, while the
Greens wish to make it easier to obtain divorces; the Blues take their
support from the heart of city areas, while the more distant farmers
and watersellers tend to be Green; the Blues believe that the Earth is
a huge spherical rock at the center of the universe, the Greens that it
is a huge flat rock circling some other object called a Sun. Not every
Blue or every Green citizen takes the
``Blue'' or
``Green'' position on every issue,
but it would be rare to find a city merchant who believed the sky was
blue, and yet advocated an individual tax and freer marriage laws.}

{
 The Underground is still polarized; an uneasy peace. A few folk
genuinely think that Blues and Greens should be friends, and it is now
common for a Green to patronize a Blue shop, or for a Blue to visit a
Green tavern. Yet from a truce originally born of exhaustion, there is
a quietly growing spirit of tolerance, even friendship.}

{
 One day, the Underground is shaken by a minor earthquake. A
sightseeing party of six is caught in the tremblor while looking at the
ruins of ancient dwellings in the upper caverns. They feel the brief
movement of the rock under their feet, and one of the tourists trips
and scrapes her knee. The party decides to turn back, fearing further
earthquakes. On their way back, one person catches a whiff of something
strange in the air, a scent coming from a long-unused passageway.
Ignoring the well-meant cautions of fellow travellers, the person
borrows a powered lantern and walks into the passageway. The stone
corridor wends upward\,\ldots and upward\,\ldots and finally terminates in
a hole carved out of the world, a place where all stone ends. Distance,
endless distance, stretches away into forever; a gathering space to
hold a thousand cities. Unimaginably far above, too bright to look at
directly, a searing spark casts light over all visible space, the naked
filament of some huge light bulb. In the air, hanging unsupported, are
great incomprehensible tufts of white cotton. And the vast glowing
ceiling above\,\ldots the \textit{color}\,\ldots is\,\ldots}

{
 Now history branches, depending on which member of the sightseeing
party decided to follow the corridor to the surface.}

{
 Aditya the Blue stood under the blue forever, and slowly smiled.
It was not a pleasant smile. There was hatred, and wounded pride; it
recalled every argument she'd ever had with a Green,
every rivalry, every contested promotion.
\textit{``You were right all
along,''} the sky whispered down at her,
\textit{``and now you can prove
it.''} For a moment Aditya stood there, absorbing the
message, glorying in it, and then she turned back to the stone corridor
to tell the world. As Aditya walked, she curled her hand into a
clenched fist. ``The truce,'' she
said, ``is over.''}

{
 Barron the Green stared incomprehendingly at the chaos of colors
for long seconds. Understanding, when it came, drove a pile-driver
punch into the pit of his stomach. Tears started from his eyes. Barron
thought of the Massacre of Cathay, where a Blue army had massacred
every citizen of a Green town, including children; he thought of the
ancient Blue general, Annas Rell, who had declared Greens
``a pit of disease; a pestilence to be
cleansed''; he thought of the glints of hatred
he'd seen in Blue eyes and something inside him
cracked. \textit{``How can you be on their
side?''} Barron screamed at the sky, and then he
began to weep; because he knew, standing under the malevolent blue
glare, that the universe had always been a place of evil.}

{
 Charles the Blue considered the blue ceiling, taken aback. As a
professor in a mixed college, Charles had carefully emphasized that
Blue and Green viewpoints were equally valid and deserving of
tolerance: The sky was a metaphysical construct, and cerulean a color
that could be seen in more than one way. Briefly, Charles wondered
whether a Green, standing in this place, might not see a green ceiling
above; or if perhaps the ceiling would be green at this time tomorrow;
but he couldn't stake the continued survival of
civilization on that. This was merely a natural phenomenon of some
kind, having nothing to do with moral philosophy or society\,\ldots but
one that might be readily misinterpreted, Charles feared. Charles
sighed, and turned to go back into the corridor. Tomorrow he would come
back alone and block off the passageway.}

{
 Daria, once Green, tried to breathe amid the ashes of her world.
\textit{I will not flinch,} Daria told herself, \textit{I will not look
away.} She had been Green all her life, and now she must be Blue. Her
friends, her family, would turn from her. \textit{Speak the truth, even
if your voice trembles,} her father had told her; but her father was
dead now, and her mother would never understand. Daria stared down the
calm blue gaze of the sky, trying to accept it, and finally her
breathing quietened. \textit{I was wrong,} she said to herself
mournfully; \textit{it's not }\textit{so complicated,
after all.} She would find new friends, and perhaps her family would
forgive her\,\ldots or, she wondered with a tinge of hope, rise to this
same test, standing underneath this same sky? ``The
sky is blue,'' Daria said experimentally, and nothing
dire happened to her; but she couldn't bring herself to
smile. Daria the Blue exhaled sadly, and went back into the world,
wondering what she would say.}

{
 Eddin, a Green, looked up at the blue sky and began to laugh
cynically. The course of his world's history came clear
at last; even he couldn't believe
they'd been such fools.
``Stupid,'' Eddin said,
``stupid, \textit{stupid,} and all the time it was
right here.'' Hatred, murders, wars, and all along it
was just a \textit{thing} somewhere, that someone had written about
like they'd write about any other thing. No poetry, no
beauty, nothing that any sane person would ever care about, just one
pointless thing that had been blown out of all proportion. Eddin leaned
against the cave mouth wearily, trying to think of a way to prevent
this information from blowing up the world, and wondering if they
didn't all deserve it.}

{
 Ferris gasped involuntarily, frozen by sheer wonder and delight.
Ferris's eyes darted hungrily about, fastening on each
sight in turn before moving reluctantly to the next; the blue
\textit{sky}, the white \textit{clouds}, the vast unknown
\textit{outside}, full of places and things (and people?) that no
Undergrounder had ever seen. ``Oh, so
\textit{that's} what color it is,''
Ferris said, and went exploring.}

\myendsectiontext


\bigskip

\mysection{Belief in Belief}

{
 Carl Sagan once told a parable of someone who comes to us and
claims: ``There is a dragon in my
garage.'' Fascinating! We reply that we wish to see
this dragon---let us set out at once for the garage!
``But wait,'' the claimant says to
us, ``it is an \textit{invisible}
dragon.'' }

{
 Now as Sagan points out, this doesn't make the
hypothesis unfalsifiable. Perhaps we go to the
claimant's garage, and although we see no dragon, we
hear heavy breathing from no visible source; footprints mysteriously
appear on the ground; and instruments show that something in the garage
is consuming oxygen and breathing out carbon dioxide.}

{
 But now suppose that we say to the claimant,
``Okay, we'll visit the garage and see
if we can hear heavy breathing,'' and the claimant
quickly says no, it's an \textit{inaudible} dragon. We
propose to measure carbon dioxide in the air, and the claimant says the
dragon does not breathe. We propose to toss a bag of flour into the air
to see if it outlines an invisible dragon, and the claimant immediately
says, ``The dragon is permeable to
flour.''}

{
 Carl Sagan used this parable to illustrate the classic moral that
poor hypotheses need to do fast footwork to avoid falsification. But I
tell this parable to make a different point: The claimant must have an
accurate model of the situation \textit{somewhere} in their mind,
because they can anticipate, in advance, \textit{exactly which
experimental results they'll need to excuse.}}

{
 Some philosophers have been much confused by such scenarios,
asking, ``Does the claimant \textit{really} believe
there's a dragon present, or not?''
As if the human brain only had enough disk space to represent one
belief at a time! Real minds are more tangled than that. There are
different types of belief; not all beliefs are direct anticipations.
The claimant clearly does not \textit{anticipate} seeing anything
unusual upon opening the garage door. Otherwise they
wouldn't make advance excuses. It may also be that the
claimant's pool of propositional beliefs contains
\textit{There is a dragon in my garage.} It may seem, to a rationalist,
that these two beliefs should collide and conflict even though they are
of different types. Yet it is a physical fact that you can write
``The sky is green!'' next to a
picture of a blue sky without the paper bursting into flames.}

{
 The rationalist virtue of empiricism is supposed to prevent us
from making this class of mistake. We're supposed to
constantly ask our beliefs which experiences they predict, make them
pay rent in anticipation. But the dragon-claimant's
problem runs deeper, and cannot be cured with such simple advice.
It's not exactly \textit{difficult} to connect belief
in a dragon to anticipated experience of the garage. If you believe
there's a dragon in your garage, then you can expect to
open up the door and see a dragon. If you don't see a
dragon, then that means there's no dragon in your
garage. This is pretty straightforward. You can even try it with your
own garage.}

{
 No, this invisibility business is a symptom of something much
worse.}

{
 Depending on how your childhood went, you may remember a time
period when you first began to doubt Santa Claus's
existence, but you still believed that you were \textit{supposed} to
believe in Santa Claus, so you tried to deny the doubts. As Daniel
Dennett observes, where it is difficult to believe a thing, it is often
much easier to believe that you \textit{ought} to believe it. What does
it mean to believe that the Ultimate Cosmic Sky is both perfectly blue
and perfectly green? The statement is confusing; it's
not even clear what it would \textit{mean} to believe it---what exactly
would \textit{be} believed, if you believed. You can much more easily
believe that it is \textit{proper}, that it is \textit{good} and
\textit{virtuous} and \textit{beneficial}, to believe that the Ultimate
Cosmic Sky is both perfectly blue and perfectly green. Dennett calls
this ``belief in
belief.''\footnote{Daniel C. Dennett, \textit{Breaking the Spell: Religion as a
Natural Phenomenon} (Penguin, 2006).\comment{1}}}

{
 And here things become complicated, as human minds are wont to
do---I think even Dennett oversimplifies how this psychology works in
practice. For one thing, if you believe in belief, you cannot admit to
yourself that you only believe in belief, because it is virtuous to
\textit{believe}, not to believe in belief, and so if you only believe
in belief, instead of believing, you are not virtuous. Nobody will
\textit{admit} to themselves, ``I
don't believe the Ultimate Cosmic Sky is blue and
green, but I believe I ought to believe it''---not
unless they are unusually capable of acknowledging their own lack of
virtue. People don't believe in belief in belief, they
just believe in belief.}

{
 (Those who find this confusing may find it helpful to study
mathematical logic, which trains one to make very sharp distinctions
between the proposition $P$, a proof of $P$, and a proof that $P$ is
provable. There are similarly sharp distinctions between $P$, wanting $P$,
believing $P$, wanting to believe $P$, and believing that you believe $P$.)}

{
 There's different kinds of belief in belief. You
may believe in belief explicitly; you may recite in your deliberate
stream of consciousness the verbal sentence ``It is
virtuous to believe that the Ultimate Cosmic Sky is perfectly blue and
perfectly green.'' (While also believing that you
believe this, unless you are unusually capable of acknowledging your
own lack of virtue.) But there are also less explicit forms of belief
in belief. Maybe the dragon-claimant fears the public ridicule that
they imagine will result if they publicly confess they were wrong
(although, in fact, a rationalist would congratulate them, and others
are more likely to ridicule the claimant if they go on claiming
there's a dragon in their garage). Maybe the
dragon-claimant flinches away from the prospect of admitting to
themselves that there is no dragon, because it conflicts with their
self-image as the glorious discoverer of the dragon, who saw in their
garage what all others had failed to see.}

{
 If all our thoughts were deliberate verbal sentences like
philosophers manipulate, the human mind would be a great deal easier
for humans to understand. Fleeting mental images, unspoken flinches,
desires acted upon without acknowledgement---these account for as much
of ourselves as words.}

{
 While I disagree with Dennett on some details and complications, I
still think that Dennett's notion of \textit{belief in
belief} is the key insight necessary to understand the dragon-claimant.
But we need a wider concept of \textit{belief}, not limited to verbal
sentences. ``Belief'' should include
unspoken anticipation-controllers. ``Belief in
belief'' should include unspoken
cognitive-behavior-guiders. It is not psychologically realistic to say,
``The dragon-claimant does not believe there is a
dragon in their garage; they believe it is beneficial to believe there
is a dragon in their garage.'' But it is realistic to
say the dragon-claimant \textit{anticipates as if} there is no dragon
in their garage, and \textit{makes excuses as if} they believed in the
belief.}

{
 You can possess an ordinary mental picture of your garage, with no
dragons in it, which correctly predicts your experiences on opening the
door, and never once think the verbal phrase \textit{There is no dragon
in my garage.} I even bet it's happened to you---that
when you open your garage door or bedroom door or whatever, and expect
to see no dragons, no such verbal phrase runs through your mind.}

{
 And to flinch away from giving up your belief in the dragon---or
flinch away from giving up your \textit{self-image} as a person who
believes in the dragon---it is not necessary to explicitly think
\textit{I want to believe there's a dragon in my
garage.} It is only necessary to flinch away from the prospect of
admitting you don't believe.}

{
 To correctly anticipate, in advance, which experimental results
shall need to be excused, the dragon-claimant must (a) possess an
accurate anticipation-controlling model somewhere in their mind, and
(b) act cognitively to protect either (b1) their free-floating
propositional belief in the dragon or (b2) their self-image of
believing in the dragon.}

{
 If someone believes in their belief in the dragon, and also
believes in the dragon, the problem is much less severe. They will be
willing to stick their neck out on experimental predictions, and
perhaps even agree to give up the belief if the experimental prediction
is wrong---although belief in belief can still interfere with this, if
the belief itself is not absolutely confident. When someone makes up
excuses \textit{in advance}, it would seem to require that belief and
belief in belief have become unsynchronized.}

\myendsectiontext


\bigskip

\mysection{Bayesian Judo}

{
 You can have some fun with people whose anticipations get out of
sync with what they believe they believe. }

{
 I was once at a dinner party, trying to explain to a man what I
did for a living, when he said: ``I
don't believe Artificial Intelligence is possible
because only God can make a soul.''}

{
 At this point I must have been divinely inspired, because I
instantly responded: ``You mean if I can make an
Artificial Intelligence, it proves your religion is
false?''}

{
 He said, ``What?''}

{
 I said, ``Well, if your religion predicts that I
can't possibly make an Artificial Intelligence, then,
if I make an Artificial Intelligence, it means your religion is false.
Either your religion allows that it might be possible for me to build
an AI; or, if I build an AI, that disproves your
religion.''}

{
 There was a pause, as the one realized he had just made his
hypothesis vulnerable to falsification, and then he said,
``Well, I didn't mean that you
couldn't make an intelligence, just that it
couldn't be emotional in the same way we
are.''}

{
 I said, ``So if I make an Artificial Intelligence
that, without being deliberately preprogrammed with any sort of script,
starts talking about an emotional life that sounds like ours,
\textit{that} means your religion is wrong.''}

{
 He said, ``Well, um, I guess we may have to agree
to disagree on this.''}

{
 I said: ``No, we can't, actually.
There's a theorem of rationality called
Aumann's Agreement Theorem which shows that no two
rationalists can agree to disagree. If two people disagree with each
other, at least one of them must be doing something
wrong.''}

{
 We went back and forth on this briefly. Finally, he said,
``Well, I guess I was really trying to say that I
don't think you can make something
eternal.''}

{
 I said, ``Well, I don't think so
either! I'm glad we were able to reach agreement on
this, as Aumann's Agreement Theorem
requires.'' I stretched out my hand, and he shook it,
and then he wandered away.}

{
 A woman who had stood nearby, listening to the conversation, said
to me gravely, ``That was
beautiful.''}

{
 ``Thank you very much,'' I
said.}

\myendsectiontext

\mysection{Pretending to be Wise}

\begin{quote}
{
 The hottest place in Hell is reserved for those who in time of
crisis remain neutral.}

{\raggedleft
 $\lnot$ Dante Alighieri, famous hell expert\newline
---John F. Kennedy, misquoter
\par}
\end{quote}


{
 It's common to put on a show of
\textit{neutrality} or \textit{suspended judgment} in order to signal
that one is mature, wise, impartial, or just has a superior vantage
point.}

{
 An example would be the case of my parents, who respond to
theological questions like ``Why does ancient Egypt,
which had good records on many other matters, lack any records of Jews
having ever been there?'' with ``Oh,
when I was your age, I also used to ask that sort of question, but now
I've grown out of it.''}

{
 Another example would be the principal who, faced with two
children who were caught fighting on the playground, sternly says:
``It doesn't matter who started the
fight, it only matters who ends it.'' Of course it
matters who started the fight. The principal may not have access to
good \textit{information} about this critical fact, but if so, the
principal should \textit{say} so, not \textit{dismiss the importance}
of who threw the first punch. Let a parent try punching the principal,
and we'll see how far ``It
doesn't matter who started it'' gets
in front of a judge. But to adults it is just \textit{inconvenient}
that children fight, and it matters not at all to their
\textit{convenience} which child started it. It is only
\textit{convenient} that the fight end as rapidly as possible.}

{
 A similar dynamic, I believe, governs the occasions in
international diplomacy where Great Powers sternly tell smaller groups
to stop that fighting \textit{right now}. It doesn't
matter to the Great Power who started it---who provoked, or who
responded disproportionately to provocation---because the Great
Power's ongoing \textit{inconvenience} is only a
function of the ongoing conflict. Oh, can't Israel and
Hamas just get along?}

{
 This I call ``pretending to be
Wise.'' Of course there are many ways to try and
signal wisdom. But trying to signal wisdom by refusing to make
guesses---refusing to sum up evidence---refusing to pass
judgment---refusing to take sides---staying above the fray and looking
down with a lofty and condescending gaze---which is to say, signaling
wisdom by saying and doing nothing---well, that I find particularly
\textit{pretentious.}}

{
 Paolo Freire said, ``Washing
one's hands of the conflict between the powerful and
the powerless means to side with the powerful, not to be
neutral.''\footnote{Paulo Freire, \textit{The Politics of Education: Culture,
Power, and Liberation} (Greenwood Publishing Group, 1985), 122.\comment{1}} A playground is a great
place to be a bully, and a terrible place to be a victim, if the
teachers don't care \textit{who started it.} And
likewise in international politics: A world where the Great Powers
refuse to take sides and only demand immediate truces is a great world
for aggressors and a terrible place for the aggressed. But, of course,
it is a very convenient world in which to be a Great Power or a school
principal.}

{
 So part of this behavior can be chalked up to sheer selfishness on
the part of the Wise.}

{
 But part of it also has to do with signaling a superior vantage
point. After all---what would the \textit{other adults} think of a
principal who actually seemed to be \textit{taking sides} in a fight
between mere \textit{children}? Why, it would lower the
principal's status to a mere \textit{participant in the
fray}!}

{
 Similarly with the revered elder---who might be a CEO, a
prestigious academic, or a founder of a mailing list---whose reputation
for fairness depends on their refusal to pass judgment themselves, when
others are choosing sides. Sides appeal to them for support, but almost
always in vain; for the Wise are revered judges on the condition that
they almost never actually judge---\textit{then} they would just be
another disputant in the fray, no better than any other mere arguer.}

{
 (Oddly, judges in the actual legal system can repeatedly hand down
real verdicts without \textit{automatically} losing their reputation
for impartiality. Maybe because of the understood norm that they
\textit{have} to judge, that it's their job. Or maybe
because judges don't have to repeatedly rule on issues
that have split a tribe on which they depend for their reverence.)}

{
 There \textit{are} cases where it is rational to suspend judgment,
where people leap to judgment only because of their biases. As Michael
Rooney said:\footnote{\url{http://lesswrong.com/lw/he/knowing_about_biases_can_hurt_people/dza}}}

\begin{quote}
{
 The error here is similar to one I see all the time in beginning
philosophy students: when confronted with reasons to be skeptics, they
instead become relativists. That is, when the rational conclusion is to
suspend judgment about an issue, all too many people instead conclude
that any judgment is as plausible as any other.}
\end{quote}

{
 But then how can we avoid the (related but distinct)
pseudo-rationalist behavior of signaling your unbiased impartiality by
falsely claiming that the current balance of evidence is neutral?
``Oh, well, of course you have a lot of passionate
Darwinists out there, but I think the evidence we have
doesn't really enable us to make a definite endorsement
of natural selection over intelligent design.''}

{
 On this point I'd advise remembering that
\textit{neutrality is a definite judgment.} It is not staying
\textit{above} anything. It is putting forth the definite and
particular position that the balance of evidence in a particular case
licenses \textit{only} one summation, which happens to be neutral.
This, too, can be wrong; propounding neutrality is just as attackable
as propounding any particular side.}

{
 Likewise with policy questions. If someone says that both pro-life
and pro-choice sides have good points and that they really should try
to compromise and respect each other more, they are not taking a
position \textit{above} the two standard sides in the abortion debate.
They are putting forth a definite judgment, every bit as particular as
saying ``pro-life!'' or
``pro-choice!''}

{
 If your goal is to improve your general ability to form more
accurate beliefs, it might be useful to avoid focusing on emotionally
charged issues like abortion or the Israeli-Palestinian conflict. But
it's \textit{not} that a rationalist is too mature to
talk about politics. It's \textit{not} that a
rationalist is above this foolish fray in which only mere political
partisans and youthful enthusiasts would stoop to participate.}

{
 As Robin Hanson describes it, the ability to have potentially
divisive conversations is a limited resource. If you can think of ways
to pull the rope sideways, you are justified in expending your limited
resources on relatively less common issues where marginal discussion
offers relatively higher marginal payoffs.}

{
 But then the responsibilities that you deprioritize are a matter
of your limited resources. \textit{Not} a matter of floating high
above, serene and Wise.}

{
 My reply to Paul Graham's comment on Hacker News
seems like a summary worth repeating:}

{
 There's a difference between:}

\begin{itemize}
\item {
 Passing neutral judgment;}

\item {
 Declining to invest marginal resources;}

\item {
 Pretending that either of the above is a mark of deep wisdom,
maturity, and a superior vantage point; with the corresponding
implication that the original sides occupy lower vantage points that
are not importantly different from up there.}
\end{itemize}

\myendsectiontext


\bigskip

\mysection{Religion's Claim to be Non{}-Disprovable}

{
 The earliest account I know of a scientific experiment is,
ironically, the story of Elijah and the priests of Baal. }

{
 The people of Israel are wavering between Jehovah and Baal, so
Elijah announces that he will conduct an experiment to settle
it---quite a novel concept in those days! The priests of Baal will
place their bull on an altar, and Elijah will place
Jehovah's bull on an altar, but neither will be allowed
to start the fire; whichever God is real will call down fire on His
sacrifice. The priests of Baal serve as control group for Elijah---the
same wooden fuel, the same bull, and the same priests making
invocations, but to a false god. Then Elijah pours water on his
altar---ruining the experimental symmetry, but this was back in the
early days---to signify deliberate acceptance of the burden of proof,
like needing a 0.05 significance level. The fire comes down on
Elijah's altar, which is the experimental observation.
The watching people of Israel shout ``The Lord is
God!''---peer review.}

{
 And then the people haul the 450 priests of Baal down to the river
Kishon and slit their throats. This is stern, but necessary. You must
firmly discard the falsified hypothesis, and do so swiftly, before it
can generate excuses to protect itself. If the priests of Baal are
allowed to survive, they will start babbling about how religion is a
separate magisterium which can be neither proven nor disproven.}

{
 Back in the old days, people actually believed their religions
instead of just believing in them. The biblical archaeologists who went
in search of Noah's Ark did not think they were wasting
their time; they anticipated they might become famous. Only after
failing to find confirming evidence---and finding disconfirming
evidence in its place---did religionists execute what William Bartley
called \textit{the retreat to commitment}, ``I believe
because I believe.''}

{
 Back in the old days, there was no concept of
religion's being a separate magisterium. The Old
Testament is a stream-of-consciousness culture dump: history, law,
moral parables, and yes, models of how the universe works. In not one
single passage of the Old Testament will you find anyone talking about
a transcendent wonder at the complexity of the universe. But you will
find plenty of scientific claims, like the universe being created in
six days (which is a metaphor for the Big Bang), or rabbits chewing
their cud. (Which is a metaphor for\,\ldots)}

{
 Back in the old days, saying the local religion
``could not be proven'' would have
gotten you burned at the stake. One of the core beliefs of Orthodox
Judaism is that God appeared at Mount Sinai and said in a thundering
voice, ``Yeah, it's all
true.'' From a Bayesian perspective
that's some darned unambiguous evidence of a
superhumanly powerful entity. (Although it doesn't
prove that the entity is God \textit{per se}, or that the entity is
benevolent---it could be alien teenagers.) The vast majority of
religions in human history---excepting only those invented
\textit{extremely} recently---tell stories of events that would
constitute completely unmistakable evidence if they'd
actually happened. The orthogonality of religion and factual questions
is a \textit{recent} and strictly \textit{Western} concept. The people
who wrote the original scriptures didn't even know the
difference.}

{
 The Roman Empire inherited philosophy from the ancient Greeks;
imposed law and order within its provinces; kept bureaucratic records;
and enforced religious tolerance. The New Testament, created during the
time of the Roman Empire, bears some traces of modernity as a result.
You couldn't invent a story about God completely
obliterating the city of Rome (a la Sodom and Gomorrah), because the
Roman historians would call you on it, and you couldn't
just stone them.}

{
 In contrast, the people who invented the Old Testament stories
could make up pretty much anything they liked. Early Egyptologists were
genuinely shocked to find no trace whatsoever of Hebrew tribes having
ever been in Egypt---they weren't expecting to find a
record of the Ten Plagues, but they expected to find
\textit{something.} As it turned out, they did find something. They
found out that, during the supposed time of the Exodus, Egypt ruled
much of Canaan. That's one \textit{huge} historical
error, but if there are no libraries, nobody can call you on it.}

{
 The Roman Empire did have libraries. Thus, the New Testament
doesn't claim big, showy, large-scale geopolitical
miracles as the Old Testament routinely did. Instead the New Testament
claims smaller miracles which nonetheless fit into the same framework
of evidence. A boy falls down and froths at the mouth; the cause is an
unclean spirit; an unclean spirit could reasonably be expected to flee
from a true prophet, but not to flee from a charlatan; Jesus casts out
the unclean spirit; therefore Jesus is a true prophet and not a
charlatan. This is perfectly ordinary Bayesian reasoning, if you grant
the basic premise that epilepsy is caused by demons (and that the end
of an epileptic fit proves the demon fled).}

{
 Not only did religion used to make claims about factual and
scientific matters, religion used to make claims about
\textit{everything.} Religion laid down a code of law---before
legislative bodies; religion laid down history---before historians and
archaeologists; religion laid down the sexual morals---before
Women's Lib; religion described the forms of
government---before constitutions; and religion answered scientific
questions from biological taxonomy to the formation of stars. The Old
Testament doesn't talk about a sense of wonder at the
complexity of the universe---it was busy laying down the death penalty
for women who wore men's clothing, which was solid and
satisfying religious content of that era. The modern concept of
religion as purely \textit{ethical} derives from every other
area's having been taken over by better institutions.
Ethics is what's \textit{left.}}

{
 Or rather, people \textit{think} ethics is what's
left. Take a culture dump from 2,500 years ago. Over time, humanity
will progress immensely, and pieces of the ancient culture dump will
become ever more glaringly obsolete. Ethics has not been immune to
human progress---for example, we now frown upon such Bible-approved
practices as keeping slaves. Why do people \textit{think} that ethics
is still fair game?}

{
 Intrinsically, there's nothing small about the
ethical problem with slaughtering thousands of innocent first-born male
children to convince an unelected Pharaoh to release slaves who
logically could have been teleported out of the country. It should be
\textit{more} glaring than the comparatively trivial scientific error
of saying that grasshoppers have four legs. And yet, if you say the
Earth is flat, people will look at you like you're
crazy. But if you say the Bible is your source of ethics, women will
not slap you. Most people's concept of rationality is
determined by what they think they can get away with; they think they
can get away with endorsing Bible ethics; and so it only requires a
manageable effort of self-deception for them to overlook the
Bible's moral problems. Everyone has agreed not to
notice the elephant in the living room, and this state of affairs can
sustain itself for a time.}

{
 Maybe someday, humanity will advance further, and anyone who
endorses the Bible as a source of ethics will be treated the same way
as Trent Lott endorsing Strom Thurmond's presidential
campaign. And then it will be said that religion's
``true core'' has always been
genealogy or something.}

{
 The idea that religion is a separate magisterium that
\textit{cannot be proven or disproven} is a Big Lie---a lie which is
repeated over and over again, so that people will say it without
thinking; yet which is, on critical examination, simply false. It is a
wild distortion of how religion happened historically, of how all
scriptures present their beliefs, of what children are told to persuade
them, and of what the majority of religious people on Earth still
believe. You have to admire its sheer brazenness, on a par with
\textit{Oceania has always been at war with Eastasia.} The prosecutor
whips out the bloody axe, and the defendant, momentarily shocked,
thinks quickly and says: ``But you
can't disprove my innocence by mere
evidence---it's a separate
magisterium!''}

{
 And if that doesn't work, grab a piece of paper
and scribble yourself a Get Out of Jail Free card.}

\myendsectiontext

\mysection{Professing and Cheering}

{
 I once attended a panel on the topic, ``Are
science and religion compatible?'' One of the women
on the panel, a pagan, held forth interminably upon how she believed
that the Earth had been created when a giant primordial cow was born
into the primordial abyss, who licked a primordial god into existence,
whose descendants killed a primordial giant and used its corpse to
create the Earth, etc. The tale was long, and detailed, and more absurd
than the Earth being supported on the back of a giant turtle. And the
speaker clearly knew enough science to know this. }

{
 I still find myself struggling for words to describe what I saw as
this woman spoke. She spoke with\,\ldots pride? Self-satisfaction? A
deliberate flaunting of herself?}

{
 The woman went on describing her creation myth for what seemed
like forever, but was probably only five minutes. That strange
pride/satisfaction/flaunting clearly had something to do with her
\textit{knowing} that her beliefs were scientifically outrageous. And
it wasn't that she hated science; as a panelist she
professed that religion and science were compatible. She even talked
about how it was quite understandable that the Vikings talked about a
primordial abyss, given the land in which they lived---explained away
her own religion!---and yet nonetheless insisted this was what she
``believed,'' said with peculiar
satisfaction.}

{
 I'm not sure that Daniel Dennett's
concept of ``belief in belief''
stretches to cover this event. It was weirder than that. She
didn't recite her creation myth with the fanatical
faith of someone who needs to reassure herself. She
didn't act like she expected us, the audience, to be
convinced---or like she needed our belief to validate her.}

{
 Dennett, in addition to suggesting belief in belief, has also
suggested that much of what is called ``religious
belief'' should really be studied as
``religious profession.'' Suppose an
alien anthropologist studied a group of postmodernist English students
who all seemingly \textit{believed} that Wulky Wilkensen was a
post-utopian author. The appropriate question may not be
``Why do the students all believe this strange
belief?'' but ``Why do they all
write this strange sentence on quizzes?'' Even if a
sentence is essentially meaningless, you can still know when you are
supposed to chant the response aloud.}

{
 I think Dennett may be slightly too cynical in suggesting that
religious profession is \textit{just} saying the belief aloud---most
people are honest enough that, if they say a religious statement aloud,
they will also feel obligated to say the verbal sentence into their own
stream of consciousness.}

{
 But even the concept of ``religious
profession'' doesn't seem to cover
the pagan woman's claim to believe in the primordial
cow. If you had to profess a religious belief to satisfy a priest, or
satisfy a co-religionist---heck, to satisfy your own self-image as a
religious person---you would have to \textit{pretend} to believe
\textit{much more convincingly} than this woman was doing. As she
recited her tale of the primordial cow, with that same strange
flaunting pride, she wasn't even \textit{trying} to be
persuasive---wasn't even trying to convince us that she
took her own religion seriously. I think that's the
part that so took me aback. I know people who believe they believe
ridiculous things, but when they profess them, they'll
spend much more effort to convince themselves that they take their
beliefs seriously.}

{
 It finally occurred to me that this woman wasn't
trying to convince us or even convince herself. Her recitation of the
creation story wasn't \textit{about} the creation of
the world at all. Rather, by launching into a five-minute diatribe
about the primordial cow, she was \textit{cheering for paganism}, like
holding up a banner at a football game. A banner saying \textsc{Go Blues}
isn't a statement of fact, or an attempt to persuade;
it doesn't have to be convincing---it's
a cheer.}

{
 That strange flaunting pride\,\ldots it was like she was marching
naked in a gay pride parade. (Not that there's anything
wrong with marching naked in a gay pride parade. Lesbianism is not
something that truth can destroy.) It wasn't just a
cheer, like marching, but an outrageous cheer, like marching
naked---believing that she couldn't be arrested or
criticized, because she was doing it for her pride parade.}

{
 That's why it mattered to her that what she was
saying was beyond ridiculous. If she'd tried to make it
sound more plausible, it would have been like putting on clothes.}

\myendsectiontext

\mysection{Belief as Attire}

{
 I have so far distinguished between belief as
anticipation-controller, belief in belief, professing, and cheering. Of
these, we might call anticipation-controlling beliefs
``proper beliefs'' and the other
forms ``improper beliefs.'' A proper
belief can be wrong or irrational, as when someone genuinely
anticipates that prayer will cure their sick baby. But the other forms
are arguably ``not belief at all.''
}

{
 Yet another form of improper belief is belief as group
identification---as a way of belonging. Robin Hanson uses the excellent
metaphor of wearing unusual clothing, a group uniform like a
priest's vestments or a Jewish skullcap, and so I will
call this ``belief as attire.''}

{
 In terms of humanly realistic psychology, the Muslims who flew
planes into the World Trade Center undoubtedly saw themselves as heroes
defending truth, justice, and the Islamic Way from hideous alien
monsters a la the movie Independence Day. Only a very inexperienced
nerd, the sort of nerd who has no idea how non-nerds see the world,
would say this out loud in an Alabama bar. It is not an American thing
to say. The American thing to say is that the terrorists
``hate our freedom'' and that flying
a plane into a building is a ``cowardly
act.'' You cannot say the phrases
``heroic self-sacrifice'' and
``suicide bomber'' in the same
sentence, even for the sake of accurately describing how the Enemy sees
the world. The very \textit{concept} of the courage and altruism of a
suicide bomber is Enemy attire---you can tell, because the Enemy talks
about it. The cowardice and sociopathy of a suicide bomber is American
attire. There are no quote marks you can use to talk about how the
Enemy sees the world; it would be like dressing up as a Nazi for
Halloween.}

{
 Belief-as-attire may help explain how people can be
\textit{passionate} about improper beliefs. Mere belief in belief, or
religious professing, would have some trouble creating genuine, deep,
powerful emotional effects. Or so I suspect; I confess
I'm not an expert here. But my impression is this:
People who've stopped anticipating-as-if their religion
is true, will go to great lengths to \textit{convince} themselves they
are passionate, and this desperation can be mistaken for passion. But
it's not the same fire they had as a child.}

{
 On the other hand, it is very easy for a human being to genuinely,
passionately, gut-level belong to a group, to cheer for their favorite
sports team. (This is the foundation on which rests the swindle of
``Republicans vs.\ Democrats'' and
analogous false dilemmas in other countries, but that's
a topic for another time.) Identifying with a tribe is a very strong
emotional force. People will die for it. And once you get people to
identify with a tribe, the beliefs which are attire of that tribe will
be spoken with the full passion of belonging to that tribe.}

\myendsectiontext

\mysection{Applause Lights}
\label{applause_lights}

{
 At the Singularity Summit 2007, one of the speakers called for
democratic, multinational development of Artificial Intelligence. So I
stepped up to the microphone and asked:}

\begin{quote}
{
 Suppose that a group of democratic republics form a consortium to
develop AI, and there's a lot of politicking during the
process---some interest groups have unusually large influence, others
get shafted---in other words, the result looks just like the products
of modern democracies. Alternatively, suppose a group of rebel nerds
develops an AI in their basement, and instructs the AI to poll everyone
in the world---dropping cellphones to anyone who
doesn't have them---and do whatever the majority says.
Which of these do you think is more
``democratic,'' and would you feel
safe with either?}
\end{quote}

{
 I wanted to find out whether he believed in the pragmatic adequacy
of the democratic political process, or if he believed in the moral
rightness of voting. But the speaker replied:}

\begin{quote}
{
 The first scenario sounds like an editorial in \textit{Reason}
 magazine, and the second sounds like a Hollywood movie plot.}
\end{quote}

{
 Confused, I asked:}

\begin{quote}
{
 Then what kind of democratic process \textit{did} you have in
 mind?}
\end{quote}

{
 The speaker replied:}

\begin{quote}
{
 Something like the Human Genome Project---that was an
 internationally sponsored research project.}
\end{quote}

{
 I asked:}

\begin{quote}
{
 How would different interest groups resolve their conflicts in a
 structure like the Human Genome Project?}
\end{quote}

{
 And the speaker said:}

\begin{quote}
{
  I don't know.}
\end{quote}

{
 This exchange puts me in mind of a quote from some dictator or
other, who was asked if he had any intentions to move his pet state
toward democracy:}

\begin{quote}
{
 We believe we are already within a democratic system. Some factors
are still missing, like the expression of the people's
will.}
\end{quote}

{
 The substance of a democracy is the specific mechanism that
resolves policy conflicts. If all groups had the same preferred
policies, there would be no need for democracy---we would automatically
cooperate. The resolution process can be a direct majority vote, or an
elected legislature, or even a voter-sensitive behavior of an
Artificial Intelligence, but it has to be \textit{something.} What does
it \textit{mean} to call for a
``democratic'' solution if you
don't have a conflict-resolution mechanism in mind?}

{
 I think it means that you have said the word
``democracy,'' so the audience is
supposed to cheer. It's not so much a propositional
statement, as the equivalent of the
``Applause'' light that tells a
studio audience when to clap.}

{
 This case is remarkable only in that I mistook the applause light
for a policy suggestion, with subsequent embarrassment for all. Most
applause lights are much more blatant, and can be detected by a simple
reversal test. For example, suppose someone says:}

\begin{quote}
{
  We need to balance the risks and opportunities of AI.}
\end{quote}

{
 If you reverse this statement, you get:}

\begin{quote}
{
 We shouldn't balance the risks and opportunities
 of AI.}
\end{quote}

{
 Since the reversal sounds \textit{ab}normal, the unreversed
statement is probably normal, implying it does not convey new
information. There are plenty of legitimate reasons for uttering a
sentence that would be uninformative in isolation.
``We need to balance the risks and opportunities of
AI'' can introduce a discussion topic; it can
emphasize the importance of a specific proposal for balancing; it can
criticize an unbalanced proposal. Linking to a normal assertion can
convey new information to a bounded rationalist---the link itself may
not be obvious. But if \textit{no} specifics follow, the sentence is
probably an applause light.}

{
 I am tempted to give a talk sometime that consists of
\textit{nothing but} applause lights, and see how long it takes for the
audience to start laughing:}

\begin{quote}
{
 I am here to propose to you today that we need to balance the
risks and opportunities of advanced Artificial Intelligence. We should
avoid the risks and, insofar as it is possible, realize the
opportunities. We should not needlessly confront entirely unnecessary
dangers. To achieve these goals, we must plan wisely and rationally. We
should not act in fear and panic, or give in to technophobia; but
neither should we act in blind enthusiasm. We should respect the
interests of all parties with a stake in the Singularity. We must try
to ensure that the benefits of advanced technologies accrue to as many
individuals as possible, rather than being restricted to a few. We must
try to avoid, as much as possible, violent conflicts using these
technologies; and we must prevent massive destructive capability from
falling into the hands of individuals. We should think through these
issues before, not after, it is too late to do anything about them\,\ldots}
\end{quote}

\myendsectiontext

\chapter{Noticing Confusion}

\mysection{Focus Your Uncertainty}

{
 Will bond yields go up, or down, or remain the same? If
you're a TV pundit and your job is to explain the
outcome after the fact, then there's no reason to
worry. No matter \textit{which} of the three possibilities comes true,
you'll be able to explain why the outcome perfectly
fits your pet market theory. There's no reason to think
of these three possibilities as somehow \textit{opposed} to one
another, as \textit{exclusive}, because you'll get full
marks for punditry no matter which outcome occurs. }

{
 But wait! Suppose you're a \textit{novice} TV
pundit, and you aren't experienced enough to make up
plausible explanations on the spot. You need to prepare remarks in
advance for tomorrow's broadcast, and you have limited
time to prepare. In this case, it would be helpful to know
\textit{which} outcome will actually occur---whether bond yields will
go up, down, or remain the same---because then you would only need to
prepare \textit{one} set of excuses.}

{
 Alas, no one can possibly foresee the future. What are you to do?
You certainly can't use
``probabilities.'' We all know from
school that ``probabilities'' are
little numbers that appear next to a word problem, and there
aren't any little numbers here. Worse, you
\textit{feel} uncertain. You don't remember
\textit{feeling} uncertain while you were manipulating the little
numbers in word problems. \textit{College classes teaching math} are
nice clean places, therefore \textit{math itself} can't
apply to life situations that aren't nice and clean.
You wouldn't want to inappropriately transfer thinking
skills from one context to another. Clearly, this is not a matter for
``probabilities.''}

{
 Nonetheless, you only have 100 minutes to prepare your excuses.
You can't spend the entire 100 minutes on
``up,'' and also spend all 100
minutes on ``down,'' and also spend
all 100 minutes on ``same.''
You've got to prioritize somehow.}

{
 If you needed to justify your time expenditure to a review
committee, you would have to spend equal time on each possibility.
Since there are no little numbers written down, you'd
have no documentation to justify spending different amounts of time.
You can hear the reviewers now: \textit{And why, Mr. Finkledinger, did
you spend exactly 42 minutes on excuse \#3? Why not 41 minutes, or 43?
Admit it---you're not being objective!
You're playing subjective favorites!}}

{
 But, you realize with a small flash of relief,
there's no review committee to scold you. This is good,
because there's a major Federal Reserve announcement
tomorrow, and it seems unlikely that bond prices will remain the same.
You don't want to spend 33 precious minutes on an
excuse you don't anticipate needing.}

{
 Your mind keeps drifting to the explanations you use on
television, of why each event plausibly fits your market theory. But it
rapidly becomes clear that plausibility can't help you
here---all three events are plausible. Fittability to your pet market
theory doesn't tell you how to divide your time.
There's an uncrossable gap between your 100 minutes of
time, which are conserved; versus your ability to explain how an
outcome fits your theory, which is unlimited.}

{
 And yet\,\ldots even in your uncertain state of mind, it seems that
you \textit{anticipate} the three events differently; that you
\textit{expect} to need some excuses more than others. And---this is
the fascinating part---when you think of something that makes it seem
\textit{more} likely that bond prices will go up, then you feel
\textit{less} likely to need an excuse for bond prices going down or
remaining the same.}

{
 It even seems like there's a relation between how
much you anticipate each of the three outcomes, and how much time you
want to spend preparing each excuse. Of course the relation
can't actually be quantified. You have 100 minutes to
prepare your speech, but there isn't 100 of anything to
divide up in this anticipation business. (Although you do work out
that, \textit{if} some particular outcome occurs, then your utility
function is logarithmic in time spent preparing the excuse.)}

{
 Still\,\ldots your mind keeps coming back to the idea that
anticipation is limited, unlike excusability, but like time to prepare
excuses. Maybe anticipation should be treated as a \textit{conserved
resource}, like money. Your first impulse is to try to get more
anticipation, but you soon realize that, even if you get more
anticiptaion, you won't have any more time to prepare
your excuses. No, your only course is to \textit{allocate} your
\textit{limited supply} of anticipation as best you can.}

{
 You're pretty sure you weren't
taught anything like that in your statistics courses. They
didn't tell you what to do when you \textit{felt} so
terribly uncertain. They didn't tell you what to do
when there were no little numbers handed to you. Why, even if you tried
to use numbers, you might end up using any sort of numbers at
all---there's no hint what kind of math to use, if you
should be using math! Maybe you'd end up using
\textit{pairs} of numbers, right and left numbers, which
you'd call DS for Dexter-Sinister\,\ldots or who knows
what else? (Though you do have only 100 minutes to spend preparing
excuses.)}

{
 If only there were an art of \textit{focusing your
uncertainty}{}---of \textit{squeezing} as much anticipation as possible
into whichever outcome will \textit{actually happen}!}

{
 But what could we call an art like that? And what would the rules
be like?}

\myendsectiontext

\mysection{What Is Evidence?}
\label{what_is_evidence}

\begin{quote}
{
 The sentence ``snow is white''
is \textit{true} if and only if snow is white.}

{\raggedleft
 {}---Alfred Tarski
\par}

{
 To say of what is, that it is, or of what is not, that it is not,
is \textit{true}.}

{\raggedleft
 {}---Aristotle, \textit{Metaphysics IV}
\par}
\end{quote}


{
 If these two quotes don't seem like a sufficient
definition of ``truth,'' skip ahead
to The Simple Truth (pg.\ \pageref{the_simple_truth}). Here I'm going to talk about
``evidence.'' (I also intend to
discuss beliefs-of-fact, not emotions or morality, as distinguished in
Feeling Rational.)}

{
 Walking along the street, your shoelaces come untied. Shortly
thereafter, for some odd reason, you start \textit{believing} your
shoelaces are untied. Light leaves the Sun and strikes your shoelaces
and bounces off; some photons enter the pupils of your eyes and strike
your retina; the energy of the photons triggers neural impulses; the
neural impulses are transmitted to the visual-processing areas of the
brain; and there the optical information is processed and reconstructed
into a 3D model that is recognized as an untied shoelace. There is a
sequence of events, a chain of cause and effect, within the world and
your brain, by which you end up believing what you believe. The final
outcome of the process is a state of \textit{mind} which mirrors the
state of your actual \textit{shoelaces}.}

{
 What is \textit{evidence}? It is an event entangled, by links of
cause and effect, with whatever you want to know about. If the target
of your inquiry is your shoelaces, for example, then the light entering
your pupils is evidence entangled with your shoelaces. This should not
be confused with the technical sense of
``entanglement'' used in
physics---here I'm just talking about
``entanglement'' in the sense of two
things that end up in correlated states because of the links of cause
and effect between them.}

{
 Not every influence creates the kind of
``entanglement'' required for
evidence. It's no help to have a machine that beeps
when you enter winning lottery numbers, if the machine \textit{also}
beeps when you enter \textit{losing} lottery numbers. The light
reflected from your shoes would not be useful evidence about your
shoelaces, if the photons ended up in the same physical state whether
your shoelaces were tied or untied.}

{
 To say it abstractly: For an event to be \textit{evidence about} a
target of inquiry, it has to happen \textit{differently} in a way
that's entangled with the \textit{different} possible
states of the target. (To say it technically: There has to be Shannon
mutual information between the evidential event and the target of
inquiry, relative to your current state of uncertainty about both of
them.)}

{
 Entanglement can be contagious \textit{when processed correctly},
which is why you need eyes and a brain. If photons reflect off your
shoelaces and hit a rock, the rock won't change much.
The rock won't reflect the shoelaces in any helpful
way; it won't be detectably different depending on
whether your shoelaces were tied or untied. This is why rocks are not
useful witnesses in court. A photographic film will contract
shoelace-entanglement from the incoming photons, so that the photo can
itself act as evidence. If your eyes and brain work correctly,
\textit{you} will become tangled up with your own shoelaces.}

{
 This is why rationalists put such a heavy premium on the
paradoxical-seeming claim that a belief is only really worthwhile if
you could, in principle, be persuaded to believe otherwise. If your
retina ended up in the same state regardless of what light entered it,
you would be blind. Some belief systems, in a rather obvious trick to
reinforce themselves, say that certain beliefs are only really
worthwhile if you believe them \textit{unconditionally---}no matter
what you see, no matter what you think. Your brain is supposed to end
up in the same state regardless. Hence the phrase,
``blind faith.'' If what you believe
doesn't depend on what you see, you've
been blinded as effectively as by poking out your eyeballs.}

{
 If your eyes and brain work correctly, your beliefs will end up
entangled with the facts. \textit{Rational thought produces beliefs
which are themselves evidence.}}

{
 If your tongue speaks truly, your rational beliefs, which are
themselves evidence, can act as evidence for someone else. Entanglement
can be transmitted through chains of cause and effect---and if you
speak, and another hears, that too is cause and effect. When you say
``My shoelaces are untied'' over a
cellphone, you're sharing your entanglement with your
shoelaces with a friend.}

{
 Therefore rational beliefs are contagious, among honest folk who
believe each other to be honest. And it's why a claim
that your beliefs are \textit{not} contagious---that you believe for
private reasons which are not transmissible---is so suspicious. If your
beliefs are entangled with reality, they \textit{should} be contagious
among honest folk.}

{
 If your model of reality suggests that the outputs of your thought
processes should \textit{not} be contagious to others, then your model
says that your beliefs are not themselves evidence, meaning they are
not entangled with reality. You should apply a reflective correction,
and stop believing.}

{
 Indeed, if you \textit{feel}, on a \textit{gut} level, what this
all \textit{means}, you will \textit{automatically} stop believing.
Because ``my belief is not entangled with
reality'' \textit{means} ``my belief
is not accurate.'' As soon as you stop believing
```snow is white' is
true,'' you should (automatically!) stop believing
``snow is white,'' or something is
very wrong.}

{
 So go ahead and explain why the kind of thought processes you use
systematically produce beliefs that mirror reality. Explain why you
think you're \textit{rational.} Why you think that,
using thought processes like the ones you use, minds will end up
believing ``snow is white'' if and
only if snow is white. If you don't believe that the
outputs of your thought processes are entangled with reality, why do
you believe the outputs of your thought processes? It's
the same thing, or it should be.}

\myendsectiontext

\mysection{Scientific Evidence, Legal Evidence, Rational Evidence}

{
 Suppose that your good friend, the police commissioner, tells you
in strictest confidence that the crime kingpin of your city is Wulky
Wilkinsen. As a rationalist, are you licensed to believe this
statement? Put it this way: if you go ahead and insult Wulky,
I'd call you foolhardy. Since it is prudent to act as
if Wulky has a substantially higher-than-default probability of being a
crime boss, the police commissioner's statement must
have been strong Bayesian evidence. }

{
 Our legal system will not imprison Wulky on the basis of the
police commissioner's statement. It is not admissible
as \textit{legal evidence}. Maybe if you locked up every person accused
of being a crime boss by a police commissioner, you'd
\textit{initially} catch a lot of crime bosses, plus some people that a
police commissioner didn't like. Power tends to
corrupt: over time, you'd catch fewer and fewer real
crime bosses (who would go to greater lengths to ensure anonymity) and
more and more innocent victims (unrestrained power attracts corruption
like honey attracts flies).}

{
 This does not mean that the police commissioner's
statement is not rational evidence. It still has a lopsided likelihood
ratio, and you'd still be a fool to insult Wulky. But
on a \textit{social} level, in pursuit of a social goal, we
deliberately define ``legal
evidence'' to include only particular kinds of
evidence, such as the police commissioner's own
observations on the night of April 4th. All legal evidence should
ideally be rational evidence, but not the other way around. We impose
special, strong, additional standards before we anoint rational
evidence as ``legal evidence.''}

{
 As I write this sentence at 8:33 p.m., Pacific time, on August
18th, 2007, I am wearing white socks. As a rationalist, are you
licensed to believe the previous statement? Yes. Could I testify to it
in court? Yes. Is it a \textit{scientific} statement? No, because there
is no experiment you can perform yourself to verify it. Science is made
up of \textit{generalizations} which apply to many particular
instances, so that you can run new real-world experiments which test
the generalization, and thereby verify for yourself that the
generalization is true, without having to trust
anyone's authority. Science is the \textit{publicly
reproducible} knowledge of humankind.}

{
 Like a court system, science as a social process is made up of
fallible humans. We want a protected pool of beliefs that are
\textit{especially} reliable. And we want social rules that encourage
the generation of such knowledge. So we impose special, strong,
additional standards before we canonize rational knowledge as
``scientific knowledge,'' adding it
to the protected belief pool. Is a rationalist licensed to believe in
the historical existence of Alexander the Great? Yes. We have a rough
picture of ancient Greece, untrustworthy but better than maximum
entropy. But we are dependent on authorities such as Plutarch; we
cannot discard Plutarch and verify everything for ourselves. Historical
knowledge is not scientific knowledge.}

{
 Is a rationalist licensed to believe that the Sun will rise on
September 18th, 2007? Yes---not with absolute certainty, but
that's the way to bet. (Pedants: interpret this as the
Earth's rotation and orbit remaining roughly constant
relative to the Sun.) Is this statement, as I write this essay on
August 18th, 2007, a \textit{scientific} belief?}

{
 It may seem perverse to deny the adjective
``scientific'' to statements like
``The Sun will rise on September 18th,
2007.'' If Science could not make predictions about
future events---events which have \textit{not yet} happened---then it
would be useless; it could make no prediction in advance of experiment.
The prediction that the Sun will rise is, definitely, an
\textit{extrapolation} from scientific generalizations. It is based
upon models of the Solar System that you could test for yourself by
experiment.}

{
 But imagine that you're constructing an experiment
to verify prediction \#27, in a new context, of an accepted theory $Q$.
You may not have any concrete reason to suspect the belief is wrong;
you just want to test it in a new context. It seems dangerous to say,
\textit{before} running the experiment, that there is a
``scientific belief'' about the
result. There is a ``conventional
prediction'' or ``theory
$Q$'s prediction.'' But if you already
know the ``scientific belief'' about
the result, why bother to run the experiment?}

{
 You begin to see, I hope, why I identify Science with
\textit{generalizations}, rather than the history of any one
experiment. A historical event happens once; generalizations apply over
many events. History is not reproducible; scientific generalizations
are.}

{
 Is my definition of ``scientific
knowledge'' \textit{true}? That is not a well-formed
question. The special standards we impose upon science are pragmatic
choices. Nowhere upon the stars or the mountains is it written that $p
< 0.05$ shall be the standard for scientific publication. Many
now argue that 0.05 is too weak, and that it would be \textit{useful}
to lower it to 0.01 or 0.001.}

{
 Perhaps future generations, acting on the theory that science is
the \textit{public}, \textit{reproducible} knowledge of humankind, will
only label as ``scientific'' papers
published in an open-access journal. If you charge for access to the
knowledge, is it part of the knowledge of \textit{humankind}? Can we
trust a result if people must pay to criticize it? Is it
\textit{really} science?}

{
 The question ``Is it \textit{really}
science?'' is ill-formed. Is a \$20,000/year
closed-access journal \textit{really} Bayesian evidence? As with the
police commissioner's private assurance that Wulky is
the kingpin, I think we must answer
``Yes.'' But should the
closed-access journal be further canonized as
``science''? Should we allow it into
the special, protected belief pool? For myself, I think science would
be better served by the dictum that only open knowledge counts as
\textit{the public, reproducible knowledge pool of humankind.}}

\myendsectiontext

\mysection{How Much Evidence Does It Take?}



{
 Previously, I defined \textit{evidence} as ``an
event entangled, by links of cause and effect, with whatever you want
to know about,'' and \textit{entangled} as
``happening differently for different possible states
of the target.'' So how much entanglement---how much
evidence---is required to support a belief? }

{
 Let's start with a question simple enough to be
mathematical: How hard would you have to entangle yourself with the
lottery in order to win? Suppose there are seventy balls, drawn without
replacement, and six numbers to match for the win. Then there are
131,115,985 possible winning combinations, hence a randomly selected
ticket would have a 1/131,115,985 probability of winning (0.0000007\%).
To win the lottery, you would need evidence \textit{selective} enough
to visibly favor one combination over 131,115,984 alternatives.}

{
 Suppose there are some tests you can perform which discriminate,
probabilistically, between winning and losing lottery numbers. For
example, you can punch a combination into a little black box that
always beeps if the combination is the winner, and has only a 1/4
(25\%) chance of beeping if the combination is wrong. In Bayesian
terms, we would say the \textit{likelihood ratio} is 4 to 1. This means
that the box is 4 times as likely to beep when we punch in a correct
combination, compared to how likely it is to beep for an incorrect
combination.}

{
 There are still a whole lot of possible combinations. If you punch
in 20 incorrect combinations, the box will beep on 5 of them by sheer
chance (on average). If you punch in all 131,115,985 possible
combinations, then while the box is certain to beep for the one winning
combination, it will also beep for 32,778,996 losing combinations (on
average).}

{
 So this box doesn't let you win the lottery, but
it's better than nothing. If you used the box, your
odds of winning would go from 1 in 131,115,985 to 1 in 32,778,997.
You've made some progress toward finding your target,
the truth, within the huge space of possibilities.}

{
 Suppose you can use another black box to test combinations
\textit{twice}, \textit{independently.} Both boxes are certain to beep
for the winning ticket. But the chance of a box beeping for a losing
combination is 1/4 \textit{independently} for each box; hence the
chance of \textit{both} boxes beeping for a losing combination is 1/16.
We can say that the \textit{cumulative} evidence, of two independent
tests, has a likelihood ratio of 16:1. The number of losing lottery
tickets that pass both tests will be (on average) 8,194,749.}

{
 Since there are 131,115,985 possible lottery tickets, you might
guess that you need evidence whose strength is around 131,115,985 to
1---an event, or series of events, which is 131,115,985 times more
likely to happen for a winning combination than a losing combination.
Actually, this amount of evidence would only be enough to give you an
\textit{even} chance of winning the lottery. Why? Because if you apply
a filter of that power to 131 million losing tickets, there will be, on
average, one losing ticket that passes the filter. The winning ticket
will also pass the filter. So you'll be left with two
tickets that passed the filter, only one of them a winner. Fifty
percent odds of winning, if you can only buy one ticket.}

{
 A better way of viewing the problem: In the beginning, there is 1
winning ticket and 131,115,984 losing tickets, so your odds of winning
are 1:131,115,984. If you use a single box, the odds of it beeping are
1 for a winning ticket and 0.25 for a losing ticket. So we multiply
1:131,115,984 by 1:0.25 and get 1:32,778,996. Adding another box of
evidence multiplies the odds by 1:0.25 again, so now the odds are 1
winning ticket to 8,194,749 losing tickets.}

{
 It is convenient to measure evidence in bits---not like bits on a
hard drive, but mathematician's bits, which are
conceptually different. Mathematician's bits are the
logarithms, base 1/2, of probabilities. For example, if there are four
possible outcomes $A$, $B$, $C$, and $D$, whose probabilities are 50\%, 25\%,
12.5\%, and 12.5\%, and I tell you the outcome was
``$D$,'' then I have transmitted three
bits of information to you, because I informed you of an outcome whose
probability was 1/8.}

{
 It so happens that 131,115,984 is slightly less than 2 to the 27th
power. So 14 boxes or 28 bits of evidence---an event 268,435,456:1
times more likely to happen if the ticket-hypothesis is true than if it
is false---would shift the odds from 1:131,115,984 to
268,435,456:131,115,984, which reduces to 2:1. Odds of 2 to 1 mean two
chances to win for each chance to lose, so the \textit{probability} of
winning with 28 bits of evidence is 2/3. Adding another box, another 2
bits of evidence, would take the odds to 8:1. Adding yet another two
boxes would take the chance of winning to 128:1.}

{
 So if you want to license a \textit{strong belief} that you will
win the lottery---arbitrarily defined as less than a 1\% probability of
being wrong---34 bits of evidence about the winning combination should
do the trick.}

{
 In general, the rules for weighing ``how much
evidence it takes'' follow a similar pattern: The
larger the \textit{space of possibilities} in which the hypothesis
lies, or the more unlikely the hypothesis seems a priori compared to
its neighbors, or the more confident you wish to be, the more evidence
you need.}

{
 You cannot defy the rules; you cannot form accurate beliefs based
on inadequate evidence. Let's say
you've got 10 boxes lined up in a row, and you start
punching combinations into the boxes. You cannot stop on the first
combination that gets beeps from all 10 boxes, saying,
``But the odds of that happening for a losing
combination are a million to one! I'll just ignore
those ivory-tower Bayesian rules and stop here.'' On
average, 131 losing tickets will pass such a test for every winner.
Considering the space of possibilities and the prior improbability, you
jumped to a too-strong conclusion based on insufficient evidence.
That's not a pointless bureaucratic regulation;
it's math.}

{
 Of course, you can still \textit{believe} based on inadequate
evidence, if that is your whim; but you will not be able to believe
\textit{accurately.} It is like trying to drive your car without any
fuel, because you don't believe in the silly-dilly
fuddy-duddy concept that it ought to take fuel to go places. It would
be so much more \textit{fun}, and so much less expensive, if we just
decided to repeal the law that cars need fuel. Isn't it
just obviously better for everyone? Well, you can try, if that is your
whim. You can even shut your eyes and pretend the car is moving. But to
\textit{really} arrive at accurate beliefs requires evidence-fuel, and
the further you want to go, the more fuel you need.}

\myendsectiontext

\mysection{Einstein's Arrogance}
\label{einsteins_arrogance}

{
 In 1919, Sir Arthur Eddington led expeditions to Brazil and to the
island of Principe, aiming to observe solar eclipses and thereby test
an experimental prediction of Einstein's novel theory
of General Relativity. A journalist asked Einstein what he would do if
Eddington's observations failed to match his theory.
Einstein famously replied: ``Then I would feel sorry
for the good Lord. The theory is correct.'' }

{
 It seems like a rather foolhardy statement, defying the trope of
Traditional Rationality that experiment above all is sovereign.
Einstein seems possessed of an arrogance so great that he would refuse
to bend his neck and submit to Nature's answer, as
scientists must do. Who can \textit{know} that the theory is correct,
in advance of experimental test?}

{
 Of course, Einstein did turn out to be right. I try to avoid
criticizing people when they are right. If they genuinely deserve
criticism, I will not need to wait long for an occasion where they are
wrong.}

{
 And Einstein may not have been quite so foolhardy as he sounded\,\ldots}

{
 To assign more than 50\% probability to the correct candidate from
a pool of 100,000,000 possible hypotheses, you need at least 27 bits of
evidence (or thereabouts). You cannot expect to find the correct
candidate without tests that are this strong, because lesser tests will
yield more than one candidate that passes all the tests. If you try to
apply a test that only has a million-to-one chance of a false positive
(\~{}20 bits), you'll end up with a hundred candidates.
Just \textit{finding} the right answer, within a large space of
possibilities, requires a large amount of evidence.}

{
 Traditional Rationality emphasizes justification:
``If you want to convince me of $X$,
you've got to present me with $Y$ amount of
evidence.'' I myself often slip into this phrasing,
whenever I say something like, ``To \textit{justify}
believing in this proposition, at more than 99\% probability, requires
34 bits of evidence.'' Or, ``In
order to assign more than 50\% probability to your hypothesis, you need
27 bits of evidence.'' The Traditional phrasing
implies that you start out with a hunch, or some private line of
reasoning that leads you to a suggested hypothesis, and then you have
to gather ``evidence'' to
\textit{confirm} it---to convince the scientific community, or justify
saying that you \textit{believe} in your hunch.}

{
 But from a Bayesian perspective, you need an amount of evidence
roughly equivalent to the complexity of the hypothesis just to locate
the hypothesis in theory-space. It's not a question of
justifying anything to anyone. If there's a hundred
million alternatives, you need at least 27 bits of evidence just to
focus your attention uniquely on the correct answer.}

{
 This is true even if you call your guess a
``hunch'' or
``intuition.'' Hunchings and
intuitings are real processes in a real brain. If your brain
doesn't have at least 10 bits of genuinely entangled
valid Bayesian evidence to chew on, your brain cannot single out a
correct 10-bit hypothesis for your attention---consciously,
subconsciously, whatever. Subconscious processes can't
find one out of a million targets using only 19 bits of entanglement
any more than conscious processes can. Hunches can be mysterious to the
huncher, but they can't violate the laws of physics.}

{
 You see where this is going: \textit{At the time of first
formulating the hypothesis}{}---the very first time the equations
popped into his head---Einstein must have had, \textit{already in his
possession,} sufficient observational evidence to single out the
complex equations of General Relativity for his unique attention. Or he
couldn't have gotten them \textit{right.}}

{
 Now, how likely is it that Einstein would have \textit{exactly}
enough observational evidence to raise General Relativity to the level
of his attention, but only justify assigning it a 55\% probability?
Suppose General Relativity is a 29.3-bit hypothesis. How likely is it
that Einstein would stumble across \textit{exactly} 29.5 bits of
evidence in the course of his physics reading?}

{
 Not likely! If Einstein had enough observational evidence to
single out the correct equations of General Relativity in the first
place, then he probably had enough evidence to be \textit{damn sure}
that General Relativity was true.}

{
 In fact, since the human brain is not a perfectly efficient
processor of information, Einstein probably had \textit{overwhelmingly
more evidence} than would, in principle, be required for a perfect
Bayesian to assign massive confidence to General Relativity.}

{
 ``Then I would feel sorry for the good Lord; the
theory is correct.'' It doesn't sound
nearly as appalling when you look at it from that perspective. And
remember that General Relativity \textit{was} correct, from all that
vast space of possibilities.}

\myendsectiontext

\mysection{Occam's Razor}
\label{occams_razor}

{
 The more complex an explanation is, the more evidence you need
just to find it in belief-space. (In Traditional Rationality this is
often phrased misleadingly, as ``The more complex a
proposition is, the more evidence is required to argue for
it.'') How can we measure the complexity of an
explanation? How can we determine how much evidence is required? }

{
 Occam's Razor is often phrased as
``The simplest explanation that fits the
facts.'' Robert Heinlein replied that the simplest
explanation is ``The lady down the street is a witch;
she did it.''}

{
 One observes that the length of an English sentence is not a good
way to measure ``complexity.'' And
``fitting'' the facts by merely
\textit{failing to prohibit} them is insufficient.}

{
 Why, exactly, is the length of an English sentence a poor measure
of complexity? Because when you speak a sentence aloud, you are using
\textit{labels} for concepts that the listener shares---the receiver
has already stored the complexity in them. Suppose we abbreviated
Heinlein's whole sentence as
``Tldtsiawsdi!'' so that the entire
explanation can be conveyed in one word; better yet,
we'll give it a short arbitrary label like
``Fnord!'' Does this reduce the
complexity? No, because you have to tell the listener in advance that
``Tldtsiawsdi!'' stands for
``The lady down the street is a witch; she did
it.'' ``Witch,''
itself, is a label for some extraordinary assertions---just because we
all know what it means doesn't mean the concept is
simple.}

{
 An enormous bolt of electricity comes out of the sky and hits
something, and the Norse tribesfolk say, ``Maybe a
really powerful agent was angry and threw a lightning
bolt.'' The human brain is the most complex artifact
in the known universe. If \textit{anger} seems simple,
it's because we don't see all the
neural circuitry that's implementing the emotion.
(Imagine trying to explain why \textit{Saturday Night Live} is funny,
to an alien species with no sense of humor. But don't
feel superior; you yourself have no sense of fnord.) The complexity of
anger, and indeed the complexity of intelligence, was glossed over by
the humans who hypothesized Thor the thunder-agent.}

{
 \textit{To a human,} Maxwell's equations take much
longer to explain than Thor. Humans don't have a
built-in vocabulary for calculus the way we have a built-in vocabulary
for anger. You've got to explain your language, and the
language behind the language, and the very concept of mathematics,
before you can start on electricity.}

{
 And yet it seems that there should be some sense in which
Maxwell's equations are \textit{simpler} than a human
brain, or Thor the thunder-agent.}

{
 There is. It's \textit{enormously} easier (as it
turns out) to write a computer program that simulates
Maxwell's equations, compared to a computer program
that simulates an intelligent emotional mind like Thor.}

{
 The formalism of Solomonoff induction measures the
``complexity of a description'' by
the length of the shortest computer program which produces that
description as an output. To talk about the ``shortest
computer program'' that does something, you need to
specify a space of computer programs, which requires a language and
interpreter. Solomonoff induction uses Turing machines, or rather,
bitstrings that specify Turing machines. What if you
don't like Turing machines? Then
there's only a constant complexity penalty to design
your own universal Turing machine that interprets whatever code you
give it in whatever programming language you like. Different inductive
formalisms are penalized by a worst-case constant factor relative to
each other, corresponding to the size of a universal interpreter for
that formalism.}

{
 In the better (in my humble opinion) versions of Solomonoff
induction, the computer program does not produce a deterministic
prediction, but assigns probabilities to strings. For example, we could
write a program to explain a fair coin by writing a program that
assigns equal probabilities to all $2^N$ strings of
length N. This is Solomonoff induction's approach to
\textit{fitting} the observed data. The higher the probability a
program assigns to the observed data, the better that program
\textit{fits} the data. And probabilities must sum to 1, so for a
program to better ``fit'' one
possibility, it must steal probability mass from some other possibility
which will then ``fit'' much more
poorly. There is no superfair coin that assigns 100\% probability to
heads and 100\% probability to tails.}

{
 How do we trade off the fit to the data, against the complexity of
the program? If you ignore complexity penalties, and think
\textit{only} about fit, then you will always prefer programs that
claim to deterministically predict the data, assign it 100\%
probability. If the coin shows \textsc{htthht}, then the program that claims
that the coin was fixed to show \textsc{htthht} fits the observed data 64 times
better than the program which claims the coin is fair. Conversely, if
you ignore fit, and consider \textit{only} complexity, then the
``fair coin'' hypothesis will always
seem simpler than any other hypothesis. Even if the coin turns up
\textsc{hthhthhhthhhhthhhhht} \ldots Indeed, the fair coin \textit{is} simpler
and it fits this data exactly as well as it fits any other string of 20
coinflips---no more, no less---but we see another hypothesis, seeming
not too complicated, that fits the data much better.}

{
 If you let a program store one more binary bit of information, it
will be able to cut down a space of possibilities by half, and hence
assign twice as much probability to all the points in the remaining
space. This suggests that one bit of program complexity should cost
\textit{at least} a ``factor of two
gain'' in the fit. If you try to design a computer
program that explicitly stores an outcome like \textsc{htthht}, the six bits
that you lose in complexity must destroy all plausibility gained by a
64-fold improvement in fit. Otherwise, you will sooner or later decide
that all fair coins are fixed.}

{
 Unless your program is being smart, and \textit{compressing} the
data, it should do no good just to move one bit from the data into the
program description.}

{
 The way Solomonoff induction works to predict sequences is that
you sum up over all allowed computer programs---if any program is
allowed, Solomonoff induction becomes uncomputable---with each program
having a prior probability of (1/2) to the power of its code length in
bits, and each program is further weighted by its fit to all data
observed so far. This gives you a weighted mixture of experts that can
predict future bits.}

{
 The Minimum Message Length formalism is nearly equivalent to
Solomonoff induction. You send a string describing a code, and then you
send a string describing the data in that code. Whichever explanation
leads to the shortest \textit{total} message is the best. If you think
of the set of allowable codes as a space of computer programs, and the
code description language as a universal machine, then Minimum Message
Length is nearly equivalent to Solomonoff induction. (Nearly, because
it chooses the \textit{shortest} program, rather than summing up over
all programs.)}

{
 This lets us see clearly the problem with using
``The lady down the street is a witch; she did
it'' to explain the pattern in the sequence
0101010101. If you're sending a message to a friend,
trying to describe the sequence you observed, you would have to say:
``The lady down the street is a witch; she made the
sequence come out 0101010101.'' Your accusation of
witchcraft wouldn't let you \textit{shorten} the rest
of the message; you would still have to describe, in full detail, the
data which her witchery caused.}

{
 Witchcraft may fit our observations in the sense of qualitatively
\textit{permitting} them; but this is because witchcraft permits
\textit{everything}, like saying
``Phlogiston!'' So, even after you
say ``witch,'' you still have to
describe all the observed data in full detail. You have not
\textit{compressed the total length of the message describing your
observations} by transmitting the message about witchcraft; you have
simply added a useless prologue, increasing the total length.}

{
 The real sneakiness was concealed in the word
``it'' of ``A witch
did it.'' A witch did \textit{what}?}

{
 Of course, thanks to hindsight bias and anchoring and fake
explanations and fake causality and positive bias and motivated
cognition, it may seem all too obvious that if a woman is a witch, of
\textit{course} she would make the coin come up 0101010101. But
I'll get to that soon enough\ldots}

\myendsectiontext

\mysection{Your Strength as a Rationalist}

{
 The following happened to me in an IRC chatroom, long enough ago
that I was still hanging around in IRC chatrooms. Time has fuzzed the
memory and my report may be imprecise. }

{
 So there I was, in an IRC chatroom, when someone reports that a
friend of his needs medical advice. His friend says that
he's been having sudden chest pains, so he called an
ambulance, and the ambulance showed up, but the paramedics told him it
was nothing, and left, and now the chest pains are getting worse. What
should his friend do?}

{
 I was confused by this story. I remembered reading about homeless
people in New York who would call ambulances just to be taken someplace
warm, and how the paramedics always had to take them to the emergency
room, even on the 27th iteration. Because if they
didn't, the ambulance company could be sued for lots
and lots of money. Likewise, emergency rooms are legally obligated to
treat anyone, regardless of ability to pay. (And the hospital absorbs
the costs, which are enormous, so hospitals are closing their emergency
rooms\,\ldots It makes you wonder what's the point of
having economists if we're just going to ignore them.)
So I didn't quite understand how the described events
could have happened. \textit{Anyone} reporting sudden chest pains
should have been hauled off by an ambulance instantly.}

{
 And this is where I fell down as a rationalist. I remembered
several occasions where my doctor would completely fail to panic at the
report of symptoms that seemed, to me, very alarming. And the Medical
Establishment was always right. Every single time. I had chest pains
myself, at one point, and the doctor patiently explained to me that I
was describing chest muscle pain, not a heart attack. So I said into
the IRC channel, ``Well, if the paramedics told your
friend it was nothing, it must \textit{really be}
nothing---they'd have hauled him off if there was the
tiniest chance of serious trouble.''}

{
 Thus I managed to explain the story within my existing model,
though the fit still felt a little forced\,\ldots}

{
 Later on, the fellow comes back into the IRC chatroom and says his
friend made the whole thing up. Evidently this was not one of his more
reliable friends.}

{
 I should have realized, perhaps, that an unknown acquaintance of
an acquaintance in an IRC channel might be less reliable than a
published journal article. Alas, belief is easier than disbelief; we
believe instinctively, but disbelief requires a conscious
effort.\footnote{Daniel T. Gilbert, Romin W. Tafarodi, and Patrick S. Malone,
``You Can't Not Believe Everything You
Read,'' \textit{Journal of Personality and Social
Psychology} 65 (2 1993): 221--233, doi:10.1037/0022-3514.65.2.221.\comment{1}}}

{
 So instead, by dint of mighty straining, I forced my model of
reality to explain an anomaly that \textit{never actually happened.}
And I \textit{knew} how embarrassing this was. I \textit{knew} that the
usefulness of a model is not what it can explain, but what it
can't. A hypothesis that forbids nothing, permits
everything, and thereby fails to constrain anticipation.}

{
 Your strength as a rationalist is your ability to be more confused
by fiction than by reality. If you are equally good at explaining any
outcome, you have zero knowledge.}

{
 We are all weak, from time to time; the sad part is that I
\textit{could} have been stronger. I had all the information I needed
to arrive at the correct answer, I even \textit{noticed} the problem,
and then I ignored it. My feeling of confusion was a Clue, and I threw
my Clue away.}

{
 I should have paid more attention to that sensation of
\textit{still feels a little forced.} It's one of the
most important feelings a truthseeker can have, a part of your strength
as a rationalist. It is a design flaw in human cognition that this
sensation manifests as a quiet strain in the back of your mind, instead
of a wailing alarm siren and a glowing neon sign reading:}

\begin{center}
  \textsc{Either Your Model Is False Or This Story Is Wrong.}
\end{center}

\myendsectiontext


\bigskip

\mysection{Absence of Evidence Is Evidence of Absence}

{
 From Robyn Dawes's \textit{Rational Choice in an
Uncertain World}:\footnote{Robyn M. Dawes, \textit{Rational Choice in An Uncertain World},
1st ed., ed. Jerome Kagan (San Diego, CA: Harcourt Brace Jovanovich,
1988), 250-251.\comment{1}}}

\begin{quote}
{
 In fact, this post-hoc fitting of evidence to hypothesis was
involved in a most grievous chapter in United States history: the
internment of Japanese-Americans at the beginning of the Second World
War. When California governor Earl Warren testified before a
congressional hearing in San Francisco on February 21, 1942, a
questioner pointed out that there had been no sabotage or any other
type of espionage by the Japanese-Americans up to that time. Warren
responded, ``I take the view that this lack [of
subversive activity] is the most ominous sign in our whole situation.
It convinces me more than perhaps any other factor that the sabotage we
are to get, the Fifth Column activities are to get, are timed just like
Pearl Harbor was timed\,\ldots I believe we are just being lulled into a
false sense of security.''}
\end{quote}

{
 Consider Warren's argument from a Bayesian
perspective. When we see evidence, hypotheses that assigned a
\textit{higher} likelihood to that evidence gain probability at the
expense of hypotheses that assigned a \textit{lower} likelihood to the
evidence. This is a phenomenon of \textit{relative} likelihoods and
\textit{relative} probabilities. You can assign a high likelihood to
the evidence and still lose probability mass to some other hypothesis,
if that other hypothesis assigns a likelihood that is even higher.}

{
 Warren seems to be arguing that, given that we see no sabotage,
this \textit{confirms} that a Fifth Column exists. You could argue that
a Fifth Column \textit{might} delay its sabotage. But the likelihood is
still higher that the \textit{absence} of a Fifth Column would perform
an absence of sabotage.}

{
 Let $E$ stand for the observation of sabotage, and $\lnot E$ for
the observation of no sabotage. The symbol $H_{1}$ stands
for the hypothesis of a Japanese-American Fifth Column, and
$H_{2}$ for the hypothesis that no Fifth Column exists. The
\textit{conditional probability} $P(E|H)$, or
``E given H,'' is how confidently
we'd expect to see the evidence $E$ if we assumed the
hypothesis $H$ were true.}

{
 Whatever the likelihood that a Fifth Column would do no sabotage,
the probability $P(\lnot E|H_{1})$, it
won't be as large as the likelihood that
there's no sabotage \textit{given that
there's no Fifth Column}, the probability
$P(\lnot E|H_{2})$. So observing a lack of
sabotage increases the probability that no Fifth Column exists.}

{
 A lack of sabotage doesn't \textit{prove} that no
Fifth Column exists. Absence of \textit{proof} is not \textit{proof} of
absence. In logic, $(A \Rightarrow B)$, read ``$A$
implies $B$,'' is not equivalent to $(\lnot A
\Rightarrow  \lnot B)$, read ``not-$A$ implies
not-$B$.''}

{
 But in probability theory, absence of \textit{evidence} is always
\textit{evidence} of absence. If $E$ is a binary event and
$P(H|E) > P(H)$, i.e., seeing $E$ increases the
probability of $H$, then $P(H|\lnot E) < P(H)$,
i.e., failure to observe E decreases the probability of H. The
probability $P(H)$ is a weighted mix of $P(H|E)$ and
$P(H|\lnot E)$, and necessarily lies between the two. If
any of this sounds at all confusing, see page \pageref{intuitive_bayesian}, An Intuitive Explanation of
Bayesian Reasoning.}

{
 Under the vast majority of real-life circumstances, a cause may
not reliably produce signs of itself, but the absence of the cause is
even less likely to produce the signs. The absence of an observation
may be strong evidence of absence or very weak evidence of absence,
depending on how likely the cause is to produce the observation. The
absence of an observation that is only weakly permitted (even if the
alternative hypothesis does not allow it at all) is very weak evidence
of absence (though it is evidence nonetheless). This is the fallacy of
``gaps in the fossil
record''---fossils form only rarely; it is futile to
trumpet the absence of a weakly permitted observation when many strong
positive observations have already been recorded. But if there are
\textit{no} positive observations at all, it is time to worry; hence
the Fermi Paradox.}

{
 Your strength as a rationalist is your ability to be more confused
by fiction than by reality; if you are equally good at explaining any
outcome you have zero knowledge. The strength of a model is not what it
\textit{can} explain, but what it \textit{can't}, for
only prohibitions constrain anticipation. If you don't
notice when your model makes the evidence unlikely, you might as well
have no model, and also you might as well have no evidence; no brain
and no eyes.}

\myendsectiontext


\bigskip

\mysection{Conservation of Expected Evidence}
\label{conservation_of_expected_evidence}

{
 Friedrich Spee von Langenfeld, a priest who heard the confessions
of condemned witches, wrote in 1631 the \textit{Cautio Criminalis}
(``prudence in criminal cases''), in
which he bitingly described the decision tree for condemning accused
witches: If the witch had led an evil and improper life, she was
guilty; if she had led a good and proper life, this too was a proof,
for witches dissemble and try to appear especially virtuous. After the
woman was put in prison: if she was afraid, this proved her guilt; if
she was not afraid, this proved her guilt, for witches
characteristically pretend innocence and wear a bold front. Or on
hearing of a denunciation of witchcraft against her, she might seek
flight or remain; if she ran, that proved her guilt; if she remained,
the devil had detained her so she could not get away. }

{
 Spee acted as confessor to many witches; he was thus in a position
to observe \textit{every} branch of the accusation tree, that no matter
\textit{what} the accused witch said or did, it was held as proof
against her. In any individual case, you would only hear one branch of
the dilemma. It is for this reason that scientists write down their
experimental predictions in advance.}

{
 But \textit{you can't have it both ways}{}---as a
matter of probability theory, not mere fairness. The rule that
``absence of evidence \textit{is} evidence of
absence'' is a special case of a more general law,
which I would name Conservation of Expected Evidence: The
\textit{expectation} of the posterior probability, after viewing the
evidence, must equal the prior probability.}

\begin{align*}
 P(H) &= P(H,E) + P(H,\lnot E) \\
 P(H) &= P(H|E) \times P(E) + P(H|\lnot E) \times P(\lnot E)
\end{align*}

{
 \textit{Therefore,} for every expectation of evidence, there is an
equal and opposite expectation of counterevidence.}

{
 If you expect a strong probability of seeing weak evidence in one
direction, it must be balanced by a weak expectation of seeing strong
evidence in the other direction. If you're very
confident in your theory, and therefore anticipate seeing an outcome
that matches your hypothesis, this can only provide a very small
increment to your belief (it is already close to 1); but the unexpected
failure of your prediction would (and must) deal your confidence a huge
blow. On \textit{average}, you must expect to be \textit{exactly} as
confident as when you started out. Equivalently, the mere
\textit{expectation} of encountering evidence---before
you've actually seen it---should not shift your prior
beliefs. (Again, if this is not intuitively obvious, see \pageref{intuitive_bayesian}, An Intuitive
Explanation of Bayesian Reasoning.)}

{
 So if you claim that ``no
sabotage'' is evidence \textit{for} the existence of
a Japanese-American Fifth Column, you must conversely hold that seeing
sabotage would argue \textit{against} a Fifth Column. If you claim that
``a good and proper life'' is
evidence that a woman is a witch, then an evil and improper life must
be evidence that she is not a witch. If you argue that God, to test
humanity's faith, refuses to reveal His existence, then
the miracles described in the Bible must argue against the existence of
God.}

{
 Doesn't quite sound right, does it? Pay attention
to that feeling of \textit{this seems a little forced,} that quiet
strain in the back of your mind. It's important.}

{
 For a true Bayesian, it is impossible to seek evidence that
\textit{confirms} a theory. There is no possible plan you can devise,
no clever strategy, no cunning device, by which you can legitimately
expect your confidence in a fixed proposition to be higher (on
\textit{average}) than before. You can only ever seek evidence to
\textit{test} a theory, not to confirm it.}

{
 This realization can take quite a load off your mind. You need not
worry about how to interpret every possible experimental result to
confirm your theory. You needn't bother planning how to
make \textit{any} given iota of evidence confirm your theory, because
you know that for every expectation of evidence, there is an equal and
oppositive expectation of counterevidence. If you try to weaken the
counterevidence of a possible
``abnormal'' observation, you can
only do it by weakening the support of a
``normal'' observation, to a
precisely equal and opposite degree. It is a zero-sum game. No matter
how you connive, no matter how you argue, no matter how you strategize,
you can't possibly expect the resulting game plan to
shift your beliefs (on average) in a particular direction.}

{
 You might as well sit back and relax while you wait for the
evidence to come in.}

{
 \ldots Human psychology is \textit{so} screwed up.}

\myendsectiontext

\mysection{Hindsight Devalues Science}

{
 This essay is closely based on an excerpt from
Meyers's \textit{Exploring Social
Psychology};\footnote{David G. Meyers, \textit{Exploring Social Psychology} (New
York: McGraw-Hill, 1994), 15--19.\comment{1}} the excerpt is worth reading in its
entirety. }

{
 Cullen Murphy, editor of \textit{The Atlantic}, said that the
social sciences turn up ``no ideas or conclusions that
can't be found in [any] encyclopedia of quotations\,\ldots
Day after day social scientists go out into the world. Day after
day they discover that people's behavior is pretty much
what you'd expect.''}

{
 Of course, the ``expectation''
is all hindsight. (Hindsight bias: Subjects who know the actual answer
to a question assign much higher probabilities they
``would have'' guessed for that
answer, compared to subjects who must guess without knowing the
answer.)}

{
 The historian Arthur Schlesinger, Jr.\ dismissed scientific studies
of World War II soldiers' experiences as
``ponderous demonstrations'' of
common sense. For example:}

\begin{enumerate}
\item {
 Better educated soldiers suffered more adjustment problems than
less educated soldiers. (Intellectuals were less prepared for battle
stresses than street-smart people.)}

\item {
 Southern soldiers coped better with the hot South Sea Island
climate than Northern soldiers. (Southerners are more accustomed to hot
weather.)}

\item {
 White privates were more eager to be promoted to noncommissioned
officers than Black privates. (Years of oppression take a toll on
achievement motivation.)}

\item {
 Southern Blacks preferred Southern to Northern White officers.
(Southern officers were more experienced and skilled in interacting
with Blacks.)}

\item {
 As long as the fighting continued, soldiers were more eager to
return home than after the war ended. (During the fighting, soldiers
knew they were in mortal danger.)}
\end{enumerate}

{
 How many of these findings do you think you \textit{could have}
predicted in advance? Three out of five? Four out of five? Are there
any cases where you would have predicted the opposite---where your
model takes a hit? Take a moment to think before continuing\,\ldots}

{
 ~}

{
 \ldots}

{
 ~}

{
 In this demonstration (from Paul Lazarsfeld by way of Meyers), all
of the findings above are the \textit{opposite} of what was actually
found.\footnote{Paul F. Lazarsfeld, ``The American
Solidier---An Expository Review,'' \textit{Public
Opinion Quarterly} 13, no. 3 (1949): 377--404.\comment{2}} How many times did you think your model took
a hit? How many times did you admit you would have been wrong?
That's how good your model really was. The measure of
your strength as a rationalist is your ability to be more confused by
fiction than by reality.}

{
 Unless, of course, I reversed the results again. What do you
think?}

{
 Do your thought processes at this point, where you \textit{really
don't} know the answer, feel different from the thought
processes you used to rationalize either side of the
``known'' answer?}

{
 Daphna Baratz exposed college students to pairs of supposed
findings, one true (``In prosperous times people spend
a larger portion of their income than during a
recession'') and one the truth's
opposite.\footnote{Daphna Baratz, \textit{How Justified Is the
``Obvious'' Reaction?} (Stanford
University, 1983).\comment{3}} In both sides of the pair, students rated
the supposed finding as what they ``would have
predicted.'' Perfectly standard hindsight bias.}

{
 Which leads people to think they have no need for science, because
they ``could have predicted'' that.}

{
 (Just as you would expect, right?)}

{
 Hindsight will lead us to systematically undervalue the
surprisingness of scientific findings, especially the discoveries we
\textit{understand}{}---the ones that seem real to us, the ones we can
retrofit into our models of the world. If you understand neurology or
physics and read news in that topic, then you probably underestimate
the surprisingness of findings in those fields too. This unfairly
devalues the contribution of the researchers; and worse, will prevent
you from noticing when you are seeing evidence that
doesn't fit what you \textit{really} would have
expected.}

{
 We need to make a conscious effort to be shocked \textit{enough.}}

\myendsectiontext


\bigskip

\chapter{Mysterious Answers}

\mysection{Fake Explanations}

{
 Once upon a time, there was an instructor who taught physics
students. One day the instructor called them into the classroom and
showed them a wide, square plate of metal, next to a hot radiator. The
students each put their hand on the plate and found the side next to
the radiator cool, and the distant side warm. And the instructor said,
\textit{Why do you think this happens?} Some students guessed
convection of air currents, and others guessed strange metals in the
plate. They devised many creative explanations, none stooping so low as
to say ``I don't
know'' or ``This seems
impossible.'' }

{
 And the answer was that before the students entered the room, the
instructor turned the plate around.\footnote{Search for ``heat
conduction.'' Taken from Joachim Verhagen,
\url{http://web.archive.org/web/20060424082937/http://www.nvon.nl:80/scheik/best/diversen/scijokes/scijokes.txt},
archived version, October 27, 2001.
\comment{1}}
}

{
 Consider the student who frantically stammers,
``Eh, maybe because of the heat conduction and
so?'' I ask: Is this answer a proper belief? The
words are easily enough professed---said in a loud, emphatic voice. But
do the words actually control anticipation?}

{
 Ponder that innocent little phrase, ``because
of,'' which comes before ``heat
conduction.'' Ponder some of the \textit{other}
things we could put after it. We could say, for example,
``Because of phlogiston,'' or
``Because of magic.''}

{
 ``Magic!'' you cry.
``That's not a \textit{scientific}
explanation!'' Indeed, the phrases
``because of heat conduction'' and
``because of magic'' are readily
recognized as belonging to different \textit{literary genres.}
``Heat conduction'' is something
that Spock might say on \textit{Star Trek}, whereas
``magic'' would be said by Giles in
\textit{Buffy the Vampire Slayer}.}

{
 However, as Bayesians, we take no notice of literary genres. For
us, the substance of a model is the control it exerts on anticipation.
If you say ``heat conduction,'' what
experience does that lead you to \textit{anticipate}? Under normal
circumstances, it leads you to anticipate that, if you put your hand on
the side of the plate near the radiator, that side will feel warmer
than the opposite side. If ``because of heat
conduction'' can also explain the radiator-adjacent
side feeling \textit{cooler}, then it can explain pretty much
\textit{anything.}}

{
 And as we all know by this point (I do hope), if you are equally
good at explaining any outcome, you have zero knowledge.
``Because of heat conduction,'' used
in such fashion, is a disguised hypothesis of maximum entropy. It is
anticipation-isomorphic to saying
``magic.'' It feels like an
explanation, but it's not.}

{
 Suppose that instead of guessing, we measured the heat of the
metal plate at various points and various times. Seeing a metal plate
next to the radiator, we would ordinarily expect the point temperatures
to satisfy an equilibrium of the diffusion equation with respect to the
boundary conditions imposed by the environment. You might not know the
exact temperature of the first point measured, but after measuring the
first points---I'm not physicist enough to know how
many would be required---you could take an excellent guess at the
rest.}

{
 A true master of the art of using numbers to constrain the
anticipation of material phenomena---a
``physicist''---would take some
measurements and say, ``This plate was in equilibrium
with the environment two and a half minutes ago, turned around, and is
now approaching equilibrium again.''}

{
 The deeper error of the students is not simply that they failed to
constrain anticipation. Their deeper error is that they thought they
were doing physics. They said the phrase ``because
of,'' followed by the sort of words Spock might say
on \textit{Star Trek}, and thought they thereby entered the magisterium
of science.}

{
 Not so. They simply moved their magic from one literary genre to
another.}

\myendsectiontext


\bigskip

\mysection{Guessing the Teacher's Password}

{
 When I was young, I read popular physics books such as Richard
Feynman's \textit{QED: The Strange Theory of Light and
Matter.} I knew that light was waves, sound was waves, matter was
waves. I took pride in my scientific literacy, when I was nine years
old. }

{
 When I was older, and I began to read the \textit{Feynman Lectures
on Physics}, I ran across a gem called ``the wave
equation.'' I could follow the
equation's derivation, but, looking back, I
couldn't see its truth at a glance. So I thought about
the wave equation for three days, on and off, until I saw that it was
embarrassingly obvious. And when I finally understood, I realized that
the whole time I had accepted the honest assurance of physicists that
light was waves, sound was waves, matter was waves, I had not had the
vaguest idea of what the word
``wave'' meant to a physicist.}

{
 There is an instinctive tendency to think that if a physicist says
``light is made of waves,'' and the
teacher says ``What is light made
of?,'' and the student says
``Waves!,'' then the student has
made a true statement. That's only fair, right? We
accept ``waves'' as a correct answer
from the physicist; wouldn't it be unfair to reject it
from the student? Surely, the answer
``Waves!'' is either \textit{true}
or \textit{false}, right?}

{
 Which is one more bad habit to unlearn from school. Words do not
have intrinsic definitions. If I hear the syllables
``bea-ver'' and think of a large
rodent, that is a fact about my own state of mind, not a fact about the
syllables ``bea-ver.'' The sequence
of syllables ``made of waves'' (or
``because of heat conduction'') is
not a \textit{hypothesis}, it is a pattern of vibrations traveling
through the air, or ink on paper. It can \textit{associate} to a
hypothesis in someone's mind, but it is not, of itself,
right or wrong. But in school, the teacher hands you a gold star for
\textit{saying} ``made of waves,''
which must be the correct answer because the teacher heard a physicist
emit the same sound-vibrations. Since verbal behavior (spoken or
written) is what gets the gold star, students begin to think that
verbal behavior has a truth-value. After all, either light is made of
waves, or it isn't, right?}

{
 And this leads into an even worse habit. Suppose the teacher
presents you with a confusing problem involving a metal plate next to a
radiator; the far side feels warmer than the side next to the radiator.
The teacher asks ``Why?'' If you say
``I don't know,''
you have \textit{no} chance of getting a gold star---it
won't even count as class participation. But, during
the current semester, this teacher has used the phrases
``because of heat convection,''
``because of heat conduction,'' and
``because of radiant heat.'' One of
these is probably what the teacher wants. You say,
``Eh, maybe because of heat
conduction?''}

{
 This is not a hypothesis \textit{about} the metal plate. This is
not even a proper belief. It is an attempt to \textit{guess the
teacher's password.}}

{
 Even visualizing the symbols of the diffusion equation (the math
governing heat conduction) doesn't mean
you've formed a hypothesis \textit{about} the metal
plate. This is not school; we are not testing your memory to see if you
can write down the diffusion equation. This is Bayescraft; we are
scoring your anticipations of experience. If you \textit{use} the
diffusion equation, by measuring a few points with a thermometer and
then trying to predict what the thermometer will say on the next
measurement, then it is definitely connected to experience. Even if the
student just visualizes something \textit{flowing}, and therefore holds
a match near the cooler side of the plate to try to measure where the
heat goes, then this mental image of flowing-ness connects to
experience; it controls anticipation.}

{
 If you aren't \textit{using} the diffusion
equation---putting in numbers and getting out results that control your
anticipation of particular experiences---then the connection between
map and territory is severed as though by a knife. What remains is not
a belief, but a verbal behavior.}

{
 In the school system, it's all about verbal
behavior, whether written on paper or spoken aloud. Verbal behavior
gets you a gold star or a failing grade. Part of unlearning this bad
habit is becoming consciously aware of the difference between an
explanation and a password.}

{
 Does this seem too harsh? When you're faced by a
confusing metal plate, can't ``heat
conduction?'' be a first step toward finding the
answer? Maybe, but only if you don't fall into the trap
of thinking that you are looking for a password. What if there is no
teacher to tell you that you failed? Then you may think that
``Light is wakalixes'' is a good
explanation, that ``wakalixes'' is
the correct password. It happened to me when I was nine years old---not
because I was stupid, but because this is what happens \textit{by
default.} This is how human beings think, unless they are trained
\textit{not} to fall into the trap. Humanity stayed stuck in holes like
this for thousands of years.}

{
 Maybe, if we drill students that \textit{words
don't count, only anticipation-controllers,} the
student will \textit{not} get stuck on ``heat
conduction? No? Maybe heat convection? That's not it
either?'' Maybe \textit{then}, thinking the phrase
``heat conduction'' will lead onto a
genuinely helpful path, like:}

\begin{itemize}
\item {
 ``Heat conduction?''}

\item {
 But that's only a phrase---what does it mean?}

\item {
 The diffusion equation?}

\item {
 But those are only symbols---how do I apply them?}

\item {
 What does applying the diffusion equation lead me to anticipate?}

\item {
 It sure doesn't lead me to anticipate that the
side of a metal plate farther away from a radiator would feel warmer.}

\item {
 I notice that I am confused. Maybe the near side just
\textit{feels} cooler, because it's made of more
insulative material and transfers less heat to my hand?
I'll try measuring the temperature\,\ldots}

\item {
 Okay, that wasn't it. Can I try to verify whether
the diffusion equation holds true of this metal plate, at all? Is heat
\textit{flowing} the way it usually does, or is something else going
on?}

\item {
 I could hold a match to the plate and try to measure how heat
 spreads over time\,\ldots}
\end{itemize}

{
 If we are \textit{not} strict about ``Eh, maybe
because of heat conduction?'' being a fake
explanation, the student will very probably get stuck on some
wakalixes-password. \textit{This happens by default: it happened to the
whole human species for thousands of years.}}

\myendsectiontext

\mysection{Science as Attire}

{
 The preview for the \textit{X-Men} movie has a voice-over saying:
 ``In every human being\,\ldots there is the genetic code\,\ldots
 for mutation.'' Apparently you can acquire all
sorts of neat abilities by mutation. The mutant Storm, for example, has
the ability to throw lightning bolts.}

{
 I beg you, dear reader, to consider the biological machinery
necessary to generate electricity; the biological adaptations necessary
to avoid being harmed by electricity; and the cognitive circuitry
required for finely tuned control of lightning bolts. If we actually
observed any organism acquiring these abilities \textit{in one
generation}, as the result of \textit{mutation}, it would outright
falsify the neo-Darwinian model of natural selection. It would be worse
than finding rabbit fossils in the pre-Cambrian. If evolutionary theory
could \textit{actually} stretch to cover Storm, it would be able to
explain anything, and we all know what that would imply.}

{
 The \textit{X-Men} comics use terms like
``evolution,''
``mutation,'' and
``genetic code,'' purely to place
themselves in what they conceive to be the \textit{literary genre} of
science. The part that scares me is wondering how many people,
especially in the media, understand science \textit{only} as a literary
genre.}

{
 I encounter people who very definitely believe in evolution, who
sneer at the folly of creationists. And yet they have no idea of what
the theory of evolutionary biology permits and prohibits.
They'll talk about ``the next step in
the evolution of humanity,'' as if natural selection
got here by following a plan. Or even worse, they'll
talk about something completely outside the domain of evolutionary
biology, like an improved design for computer chips, or corporations
splitting, or humans uploading themselves into computers, and
they'll call \textit{that}
``evolution.'' If evolutionary
biology could cover that, it could cover anything.}

{
 Probably an actual majority of the people who \textit{believe in}
evolution use the phrase ``because of
evolution'' because they want to be part of the
scientific in-crowd---belief as scientific attire, like wearing a lab
coat. If the scientific in-crowd instead used the phrase
``because of intelligent design,''
they would just as cheerfully use that instead---it would make no
difference to their anticipation-controllers. Saying
``because of evolution'' instead of
``because of intelligent design''
does not, \textit{for them,} prohibit Storm. Its only purpose, for
them, is to identify with a tribe.}

{
 I encounter people who are quite willing to entertain the notion
of dumber-than-human Artificial Intelligence, or even mildly
smarter-than-human Artificial Intelligence. Introduce the notion of
strongly superhuman Artificial Intelligence, and
they'll suddenly decide it's
``pseudoscience.''
It's not that they think they have a theory of
intelligence which lets them calculate a theoretical upper bound on the
power of an optimization process. Rather, they associate strongly
superhuman AI to the \textit{literary genre} of apocalyptic literature;
whereas an AI running a small corporation associates to the literary
genre of \textit{Wired} magazine. They aren't speaking
from within a model of cognition. They don't realize
they \textit{need} a model. They don't realize that
science is \textit{about} models. Their devastating critiques consist
purely of \textit{comparisons to apocalyptic literature}, rather than,
say, known laws which prohibit such an outcome. They understand science
\textit{only} as a literary genre, or in-group to belong to. The attire
doesn't look to them like a lab coat; this
isn't the football team they're
cheering for.}

{
 Is there any idea in science that you are \textit{proud} of
believing, though you do not use the belief professionally? You had
best ask yourself which future experiences your belief
\textit{prohibits} from happening to you. That is the sum of what you
have assimilated and made a true part of yourself. Anything else is
probably passwords or attire.}

\myendsectiontext

\mysection{Fake Causality}
\label{fake_causality}

{
 Phlogiston was the eighteenth century's answer to
the Elemental Fire of the Greek alchemists. Ignite wood, and let it
burn. What is the orangey-bright
``fire'' stuff? Why does the wood
transform into ash? To both questions, the eighteenth-century chemists
answered, ``phlogiston.'' }

{
 \ldots and that was it, you see, that was their answer:
``Phlogiston.''}

{
 Phlogiston escaped from burning substances as visible fire. As the
phlogiston escaped, the burning substances lost phlogiston and so
became ash, the ``true material.''
Flames in enclosed containers went out because the air became saturated
with phlogiston, and so could not hold any more. Charcoal left little
residue upon burning because it was nearly pure phlogiston.}

{
 Of course, one didn't use phlogiston theory to
\textit{predict} the outcome of a chemical transformation. You looked
at the result first, then you used phlogiston theory to
\textit{explain} it. It's not that phlogiston theorists
predicted a flame would extinguish in a closed container; rather they
lit a flame in a container, watched it go out, and then said,
``The air must have become saturated with
phlogiston.'' You couldn't even use
phlogiston theory to say what you ought \textit{not} to see; it could
explain everything.}

{
 This was an earlier age of science. For a long time, no one
realized there was a problem. Fake explanations don't
\textit{feel} fake. That's what makes them dangerous.}

{
 Modern research suggests that humans think about cause and effect
using something like the directed acyclic graphs (DAGs) of Bayes nets.
Because it rained, the sidewalk is wet; because the sidewalk is wet, it
is slippery:}

{
 ~}

\mygraphics{images/img44.jpg}

{
 ~}

{
 From this we can infer---or, in a Bayes net, rigorously calculate
in probabilities---that when the sidewalk is slippery, it probably
rained; but if we already know that the sidewalk is wet, learning that
the sidewalk is slippery tells us nothing more about whether it
rained.}

{
 Why is fire hot and bright when it burns?}

{
 ~}

\mygraphics{images/img45.jpg}

{
 ~}

{
 It \textit{feels} like an explanation. It's
\textit{represented} using the same cognitive data format. But the
human mind does not automatically detect when a cause has an
unconstraining arrow to its effect. Worse, thanks to hindsight bias, it
may feel like the cause constrains the effect, when it was merely
fitted to the effect.}

{
 Interestingly, our modern understanding of probabilistic reasoning
about causality can describe precisely what the phlogiston theorists
were doing wrong. One of the primary inspirations for Bayesian networks
was noticing the problem of double-counting evidence if inference
resonates between an effect and a cause. For example,
let's say that I get a bit of unreliable information
that the sidewalk is wet. This should make me think
it's more likely to be raining. But, if
it's more likely to be raining, doesn't
that make it more likely that the sidewalk is wet? And
wouldn't \textit{that} make it more likely that the
sidewalk is slippery? But if the sidewalk is slippery,
it's probably wet; and then I should again raise my
probability that it's raining\,\ldots}

{
 Judea Pearl uses the metaphor of an algorithm for counting
soldiers in a line.\footnote{Judea Pearl, \textit{Probabilistic Reasoning in Intelligent
Systems: Networks of Plausible Inference} (San Mateo, CA: Morgan
Kaufmann, 1988).\comment{1}} Suppose you're
in the line, and you see two soldiers next to you, one in front and one
in back. That's three soldiers, including you. So you
ask the soldier behind you, ``How many soldiers do
\textit{you} see?'' They look around and say,
``Three.'' So that's
a total of six soldiers. This, obviously, is \textit{not} how to do
it.}

{
 A smarter way is to ask the soldier in front of you,
``How many soldiers forward of
you?'' and the soldier in back,
``How many soldiers backward of
you?'' The question ``How many
soldiers forward?'' can be passed on as a message
without confusion. If I'm at the front of the line, I
pass the message ``1 soldier
forward,'' for myself. The person directly in back of
me gets the message ``1 soldier
forward,'' and passes on the message
``2 soldiers forward'' to the
soldier behind them. At the same time, each soldier is also getting the
message ``$N$ soldiers backward'' from
the soldier behind them, and passing it on as ``$N + 1$
soldiers backward'' to the soldier in front of them.
How many soldiers in total? Add the two numbers you receive, plus one
for yourself: that is the total number of soldiers in line.}

{
 The key idea is that every soldier must \textit{separately} track
the two messages, the forward-message and backward-message, and add
them together only at the end. You never add any soldiers from the
backward-message you receive to the forward-message you pass back.
Indeed, the total number of soldiers is never passed as a message---no
one ever says it aloud.}

{
 An analogous principle operates in rigorous probabilistic
reasoning about causality. If you learn something about whether
it's raining, from some source \textit{other} than
observing the sidewalk to be wet, this will send a forward-message from
\framebox{Rain} to \framebox{Sidewalk wet} and raise our expectation of the sidewalk being
wet. If you observe the sidewalk to be wet, this sends a
backward-message to our belief that it is raining, and this message
propagates from \framebox{Rain} to all neighboring nodes \textit{except} the
\framebox{Sidewalk wet} node. We count each piece of evidence exactly once; no
update message ever ``bounces'' back
and forth. The exact algorithm may be found in Judea
Pearl's classic \textit{Probabilistic Reasoning in
Intelligent Systems: Networks of Plausible Inference}.}

{
 So what went wrong in phlogiston theory? When we observe that fire
is hot, the \framebox{Fire} node can send a backward-evidence to the \framebox{Phlogiston}
node, leading us to update our beliefs about phlogiston. But if so, we
can't count this as a successful forward-prediction of
phlogiston theory. The message should go in only one direction, and not
bounce back.}

{
 Alas, human beings do not use a rigorous algorithm for updating
belief networks. We learn about parent nodes from observing children,
and predict child nodes from beliefs about parents. But we
don't keep rigorously separate books for the
backward-message and forward-message. We just remember that phlogiston
is hot, which \textit{causes} fire to be hot. So it seems like
phlogiston theory predicts the hotness of fire. Or, worse, it just
feels like \textit{phlogiston makes the fire hot.}}

{
 Until you notice that no \textit{advance} predictions are being
made, the non-constraining causal node is not labeled
``fake.'' It's
represented the same way as any other node in your belief network. It
feels like a fact, like all the other facts you know:
\textit{Phlogiston makes the fire hot.}}

{
 A properly designed AI would notice the problem instantly. This
wouldn't even require special-purpose code, just
correct bookkeeping of the belief network. (Sadly, we humans
can't rewrite our own code, the way a properly designed
AI could.)}

{
 Speaking of ``hindsight bias''
is just the nontechnical way of saying that humans do not rigorously
separate forward and backward messages, allowing forward messages to be
contaminated by backward ones.}

{
 Those who long ago went down the path of phlogiston were not
trying to be fools. No scientist deliberately wants to get stuck in a
blind alley. Are there any fake explanations in \textit{your} mind? If
there are, I guarantee they're not labeled
``fake explanation,'' so polling
your thoughts for the ``fake''
keyword will not turn them up.}

{
 Thanks to hindsight bias, it's also not enough to
check how well your theory
``predicts'' facts you already know.
You've got to predict for tomorrow, not yesterday.
It's the only way a messy human mind can be guaranteed
of sending a pure forward message.}

\myendsectiontext


\bigskip

\mysection{Semantic Stopsigns}
\label{semantic_stopsigns}

{
 \textit{And the child asked:}}

\begin{quote}
{
 Q: Where did this rock come from?}

{
 A: I chipped it off the big boulder, at the center of the
village.}

{
 Q: Where did the boulder come from?}

{
 A: It probably rolled off the huge mountain that towers over our
village.}

{
 Q: Where did the mountain come from?}

{
 A: The same place as all stone: it is the bones of Ymir, the
primordial giant.}

{
 Q: Where did the primordial giant, Ymir, come from?}

{
 A: From the great abyss, Ginnungagap.}

{
 Q: Where did the great abyss, Ginnungagap, come from?}

{
 A: Never ask that question.}
\end{quote}

{
 Consider the seeming paradox of the First Cause. Science has
traced events back to the Big Bang, but why did the Big Bang happen?
It's all well and good to say that the zero of time
begins at the Big Bang---that there is nothing before the Big Bang in
the ordinary flow of minutes and hours. But saying this presumes our
physical law, which itself appears highly structured; it calls out for
explanation. Where did the physical laws come from? You could say that
we're all a computer simulation, but then the computer
simulation is running on some other world's laws of
physics---where did \textit{those} laws of physics come from?}

{
 At this point, some people say,
``God!''}

{
 What could possibly make anyone, even a highly religious person,
think this even \textit{helped} answer the paradox of the First Cause?
Why wouldn't you automatically ask,
``Where did God come from?'' Saying
``God is uncaused'' or
``God created Himself'' leaves us in
exactly the same position as ``Time began with the Big
Bang.'' We just ask why the whole metasystem exists
in the first place, or why some events but not others are allowed to be
uncaused.}

{
 My purpose here is not to discuss the seeming paradox of the First
Cause, but to ask why anyone would think
``God!''~\textit{could} resolve the
paradox. Saying ``God!''~is a way of
belonging to a tribe, which gives people a motive to say it as often as
possible---some people even say it for questions like
``Why did this hurricane strike New
Orleans?'' Even so, you'd hope people
would notice that on the \textit{particular} puzzle of the First Cause,
saying ``God!''~doesn't help. It doesn't make the
paradox seem any less paradoxical \textit{even if true.} How could
anyone \textit{not} notice this?}

{
 Jonathan Wallace suggested that
``God!''~functions as a
\textit{semantic stopsign}{}---that it isn't a
propositional assertion, so much as a cognitive traffic signal: do not
think past this point. Saying
``God!''~doesn't so
much resolve the paradox, as put up a cognitive traffic signal to halt
the obvious continuation of the question-and-answer chain.}

{
 Of course \textit{you'd} never do that, being a
good and proper atheist, right? But
``God!''~isn't the
\textit{only} semantic stopsign, just the obvious first example.}

{
 The transhuman technologies---molecular nanotechnology, advanced
biotech, genetech, Artificial Intelligence, et cetera---pose tough
policy questions. What kind of role, if any, should a government take
in supervising a parent's choice of genes for their
child? Could parents deliberately choose genes for schizophrenia? If
enhancing a child's intelligence is expensive, should
governments help ensure access, to prevent the emergence of a cognitive
elite? You can propose various institutions to answer these policy
questions---for example, that private charities should provide
financial aid for intelligence enhancement---but the obvious next
question is, ``Will this institution be
effective?'' If we rely on product liability lawsuits
to prevent corporations from building harmful nanotech, will that
really \textit{work}?}

{
 I know someone whose answer to every one of these questions is
``Liberal democracy!''
That's it. That's his answer. If you
ask the obvious question of ``How well have liberal
democracies performed, historically, on problems this
tricky?''~or ``What if liberal
democracy does something stupid?''~then
you're an autocrat, or libertopian, or otherwise a very
very bad person. No one is allowed to question democracy.}

{
 I once called this kind of thinking ``the divine
right of democracy.'' But it is more precise to say
that ``Democracy!''~functioned for
him as a semantic stopsign. If anyone had said to him
``Turn it over to the Coca-Cola
corporation!,'' he would have asked the obvious next
questions: ``Why? What will the Coca-Cola corporation
do about it? Why should we trust them? Have they done well in the past
on equally tricky problems?''}

{
 Or suppose that someone says ``Mexican-Americans
are plotting to remove all the oxygen in Earth's
atmosphere.'' You'd probably ask,
``Why would they do \textit{that}?
Don't Mexican-Americans have to breathe too? Do
Mexican-Americans even function as a unified
conspiracy?'' If you don't ask these
obvious next questions when someone says,
``Corporations are plotting to remove
Earth's oxygen,'' then
``Corporations!''~functions for you
as a semantic stopsign.}

{
 Be careful here not to create a new generic counterargument
against things you don't like---``Oh,
it's just a stopsign!'' No word is a
stopsign of itself; the question is whether a word has that effect on a
particular person. Having strong emotions about something
doesn't qualify it as a stopsign. I'm
not exactly fond of terrorists or fearful of private property; that
doesn't mean
``Terrorists!''~or
``Capitalism!''~are cognitive
traffic signals unto me. (The word
``intelligence'' did once have that
effect on me, though no longer.) What distinguishes a semantic stopsign
is \textit{failure to consider the obvious next question.}}

\myendsectiontext

\mysection{Mysterious Answers to Mysterious Questions}
\label{matmq}

{
 Imagine looking at your hand, and knowing nothing of cells,
nothing of biochemistry, nothing of DNA. You've learned
some anatomy from dissection, so you know your hand contains muscles;
but you don't know why muscles move instead of lying
there like clay. Your hand is just\,\ldots stuff\,\ldots and for some
reason it moves under your direction. Is this not magic?}

\begin{quote}
{
 The animal body does not act as a thermodynamic engine\,\ldots
consciousness teaches every individual that they are, to some extent,
subject to the direction of his will. It appears therefore that
animated creatures have the power of immediately applying to certain
moving particles of matter within their bodies, forces by which the
motions of these particles are directed to produce derived mechanical
effects\,\ldots The influence of animal or vegetable life on matter is
infinitely beyond the range of any scientific inquiry hitherto entered
on. Its power of directing the motions of moving particles, in the
demonstrated daily miracle of our human free-will, and in the growth of
generation after generation of plants from a single seed, are
infinitely different from any possible result of the fortuitous
concurrence of atoms\,\ldots Modern biologists were coming once more to
the acceptance of something and that was a vital principle.}

{\raggedleft
 {}---Lord Kelvin\footnote{Silvanus Phillips Thompson, \textit{The Life of Lord Kelvin}
(American Mathematical Society, 2005).\comment{1}}
\par}
\end{quote}


{
 This was the theory of \textit{vitalism}; that the mysterious
difference between living matter and non-living matter was explained by
an \textit{élan vital} or \textit{vis vitalis}. \textit{élan vital}
infused living matter and caused it to move as consciously directed.
\textit{élan vital} participated in chemical transformations which no
mere non-living particles could undergo---Wöhler's
later synthesis of urea, a component of urine, was a major blow to the
vitalistic theory because it showed that mere \textit{chemistry} could
duplicate a product of biology.}

{
 Calling ``élan vital'' an
explanation, even a fake explanation like phlogiston, is probably
giving it too much credit. It functioned primarily as a
curiosity-stopper. You said ``Why?''~and the answer was ``Élan vital!''}

{
 When you say ``Élan vital!,''
it \textit{feels} like you know why your hand moves. You have a little
causal diagram in your head that says:}

{
 ~}

\mygraphics{images/img48.jpg}


{
 ~}

{
 But actually you know nothing you didn't know
before. You don't know, say, whether your hand will
generate heat or absorb heat, unless you have observed the fact
already; if not, you won't be able to predict it in
advance. Your curiosity feels sated, but it hasn't been
fed. Since you can say ``Why? Élan
vital!'' to any possible observation, it is equally
good at explaining all outcomes, a disguised hypothesis of maximum
entropy, et cetera.}

{
 But the greater lesson lies in the vitalists'
reverence for the \textit{élan vital}, their eagerness to pronounce it
a mystery beyond all science. Meeting the great dragon Unknown, the
vitalists did not draw their swords to do battle, but bowed their necks
in submission. They took pride in their ignorance, made biology into a
\textit{sacred} mystery, and thereby became loath to relinquish their
ignorance when evidence came knocking.}

{
 The Secret of Life was \textit{infinitely beyond the reach of
science!} Not just a \textit{little} beyond, mind you, but
\textit{infinitely} beyond! Lord Kelvin sure did get a tremendous
emotional kick out of \textit{not knowing something.}}

{
 But ignorance exists in the map, not in the territory. If I am
ignorant about a phenomenon, that is a fact about my own state of mind,
not a fact about the phenomenon itself. A phenomenon can \textit{seem}
mysterious to some particular person. There are no phenomena which are
mysterious of themselves. To worship a phenomenon because it seems so
wonderfully mysterious is to worship your own ignorance.}

{
 Vitalism shared with phlogiston the error of \textit{encapsulating
the mystery as a substance.} Fire was mysterious, and the phlogiston
theory encapsulated the mystery in a mysterious substance called
``phlogiston.'' Life was a sacred
mystery, and vitalism encapsulated the sacred mystery in a mysterious
substance called ``élan vital.''
Neither answer helped concentrate the model's
probability density---make some outcomes easier to explain than others.
The ``explanation'' just wrapped up
the question as a small, hard, opaque black ball.}

{
 In a comedy written by Moliére, a physician explains the power of
a soporific by saying that it contains a ``dormitive
potency.'' Same principle. It is a failure of human
psychology that, faced with a mysterious phenomenon, we more readily
postulate mysterious inherent substances than complex underlying
processes.}

{
 But the deeper failure is supposing that an \textit{answer} can be
mysterious. If a phenomenon feels mysterious, that is a fact about our
state of knowledge, not a fact about the phenomenon itself. The
vitalists saw a mysterious gap in their knowledge, and postulated a
mysterious stuff that plugged the gap. In doing so, they mixed up the
map with the territory. All confusion and bewilderment exist in the
mind, not in encapsulated substances.}

{
 This is the ultimate and fully general explanation for why, again
and again in humanity's history, people are shocked to
discover that an incredibly mysterious question has a non-mysterious
answer. Mystery is a property of questions, not answers.}

{
 Therefore I call theories such as vitalism \textit{mysterious
answers to mysterious questions}.}

{
 These are the signs of mysterious answers to mysterious
questions:}

\begin{itemize}
\item {
 First, the explanation acts as a curiosity-stopper rather than an
anticipation-controller.}

\item {
 Second, the hypothesis has no moving parts---the model is not a
specific complex mechanism, but a blankly solid substance or force. The
mysterious substance or mysterious force may be said to be here or
there, to cause this or that; but the reason why the mysterious force
behaves thus is wrapped in a blank unity.}

\item {
 Third, those who proffer the explanation cherish their ignorance;
they speak proudly of how the phenomenon defeats ordinary science or is
unlike merely mundane phenomena.}

\item {
 Fourth, \textit{even after the answer is given, the phenomenon is
still a mystery} and possesses the same quality of wonderful
 inexplicability that it had at the start.}

\end{itemize}

\myendsectiontext


\bigskip

\mysection{The Futility of Emergence}
\label{futility_of_emergence}

{
 The failures of phlogiston and vitalism are historical hindsight.
Dare I step out on a limb, and name some \textit{current} theory which
I deem analogously flawed? }

{
 I name \textit{emergence} or \textit{emergent
phenomena}{}---usually defined as the study of systems whose high-level
behaviors arise or ``emerge'' from
the interaction of many low-level elements. (Wikipedia:
``The way complex systems and patterns arise out of a
multiplicity of relatively simple interactions.'')
Taken literally, that description fits every phenomenon in our universe
above the level of individual quarks, which is part of the problem.
Imagine pointing to a market crash and saying
``It's not a
quark!'' Does that feel like an explanation? No? Then
neither should saying ``It's an
emergent phenomenon!''}

{
 It's the noun
``emergence'' that I protest, rather
than the verb ``emerges from.''
There's nothing wrong with saying ``$X$
emerges from $Y$,'' where $Y$ is some specific, detailed
model with internal moving parts. ``Arises
from'' is another legitimate phrase that means
exactly the same thing: Gravity arises from the curvature of spacetime,
according to the specific mathematical model of General Relativity.
Chemistry arises from interactions between atoms, according to the
specific model of quantum electrodynamics.}

{
 Now suppose I should say that gravity is explained by
``arisence'' or that chemistry is an
``arising phenomenon,'' and claim
that as my explanation.}

{
 The phrase ``emerges from'' is
acceptable, just like ``arises
from'' or ``is caused
by'' are acceptable, if the phrase precedes some
specific model to be judged on its own merits.}

{
 However, this is \textit{not} the way
``emergence'' is commonly used.
``Emergence'' is commonly used as an
explanation in its own right.}

{
 I have lost track of how many times I have heard people say,
``Intelligence is an emergent
phenomenon!'' as if that explained intelligence. This
usage fits all the checklist items for a mysterious answer to a
mysterious question. What do you know, after you have said that
intelligence is ``emergent''? You
can make no new predictions. You do not know anything about the
behavior of real-world minds that you did not know before. It feels
like you believe a new fact, but you don't anticipate
any different outcomes. Your curiosity feels sated, but it has not been
fed. The hypothesis has no moving parts---there's no
detailed internal model to manipulate. Those who proffer the hypothesis
of ``emergence'' confess their
ignorance of the internals, and take pride in it; they contrast the
science of ``emergence'' to other
sciences merely mundane.}

{
 And even after the answer of ``Why?
Emergence!'' is given, \textit{the phenomenon is
still a mystery} and possesses the same sacred impenetrability it had
at the start.}

{
 A fun exercise is to eliminate the adjective
``emergent'' from any sentence in
which it appears, and see if the sentence says anything different:}

\begin{itemize}
\item {
 \textit{Before:} Human intelligence is an emergent product of
neurons firing.}

\item {
 \textit{After:} Human intelligence is a product of neurons
firing.}

\item {
 \textit{Before:} The behavior of the ant colony is the emergent
outcome of the interactions of many individual ants.}

\item {
 \textit{After:} The behavior of the ant colony is the outcome of
the interactions of many individual ants.}

\item {
 \textit{Even better:} A colony is made of ants. We can
successfully predict some aspects of colony behavior using models that
include only individual ants, without any global colony variables,
showing that we understand how those colony behaviors arise from ant
behaviors.}
\end{itemize}

{
 Another fun exercise is to replace the word
``emergent'' with the old word, the
explanation that people had to use before emergence was invented:}

\begin{itemize}
\item {
 \textit{Before:} Life is an emergent phenomenon.}

\item {
 \textit{After:} Life is a magical phenomenon.}

\item {
 \textit{Before:} Human intelligence is an emergent product of
neurons firing.}

\item {
 \textit{After:} Human intelligence is a magical product of neurons
 firing.}
\end{itemize}

{
 Does not each statement convey exactly the same amount of
knowledge about the phenomenon's behavior? Does not
each hypothesis fit exactly the same set of outcomes?}

{
 ``Emergence'' has become very
popular, just as saying ``magic''
used to be very popular.
``Emergence'' has the same deep
appeal to human psychology, for the same reason.
``Emergence'' is such a wonderfully
easy explanation, and it feels good to say it; it gives you a sacred
mystery to worship. Emergence is popular \textit{because} it is the
junk food of curiosity. You can explain anything using emergence, and
so people do just that; for it feels so wonderful to explain things.
Humans are still humans, even if they've taken a few
science classes in college. Once they find a way to escape the shackles
of settled science, they get up to the same shenanigans as their
ancestors---dressed up in the literary genre of
``science,'' but humans are still
humans, and human psychology is still human psychology.}

\myendsectiontext

\mysection{Say Not ``Complexity''}

{
 Once upon a time\,\ldots }

{
 This is a story from when I first met Marcello, with whom I would
later work for a year on AI theory; but at this point I had not yet
accepted him as my apprentice. I knew that he competed at the national
level in mathematical and computing olympiads, which sufficed to
attract my attention for a closer look; but I didn't
know yet if he could learn to think about AI.}

{
 I had asked Marcello to say how he thought an AI might discover
how to solve a Rubik's Cube. Not in a preprogrammed
way, which is trivial, but rather how the AI itself might figure out
the laws of the Rubik universe and reason out how to exploit them. How
would an AI \textit{invent for itself} the concept of an
``operator,'' or
``macro,'' which is the key to
solving the Rubik's Cube?}

{
 At some point in this discussion, Marcello said:
``Well, I think the AI needs complexity to do $X$, and
complexity to do $Y$---''}

{
 And I said, ``Don't say
`\textit{complexity.}'''}

{
 Marcello said, ``Why not?''}

{
 I said, ``Complexity should never be a goal in
itself. You may need to use a particular algorithm that adds some
amount of complexity, but complexity for the sake of complexity just
makes things harder.'' (I was thinking of all the
people whom I had heard advocating that the Internet would
``wake up'' and become an AI when it
became ``sufficiently complex.'')}

{
 And Marcello said, ``But there's
got to be \textit{some} amount of complexity that does
it.''}

{
 I closed my eyes briefly, and tried to think of how to explain it
all in words. To me, saying
``complexity'' simply \textit{felt}
like the wrong move in the AI dance. No one can think fast enough to
deliberate, in words, about each sentence of their stream of
consciousness; for that would require an infinite recursion. We think
in words, but our stream of consciousness is steered below the level of
words, by the trained-in remnants of past insights and harsh experience\,\ldots}

{
 I said, ``Did you read A Technical Explanation of
Technical Explanation?''}

{
 ``Yes,'' said Marcello.}

{
 ``Okay,'' I said.
``Saying `complexity'
doesn't concentrate your probability
mass.''}

{
 ``Oh,'' Marcello said,
``like `emergence.'
Huh. So\,\ldots now I've got to think about how $X$ might
actually happen\,\ldots''}

{
 That was when I thought to myself,
``\textit{Maybe \textbf{this} one is
teachable.}''}

{
 Complexity is not a useless concept. It has mathematical
definitions attached to it, such as Kolmogorov complexity and
Vapnik-Chervonenkis complexity. Even on an intuitive level, complexity
is often worth thinking about---you have to judge the complexity of a
hypothesis and decide if it's ``too
complicated'' given the supporting evidence, or look
at a design and try to make it simpler.}

{
 But concepts are not useful or useless of themselves. Only
\textit{usages} are correct or incorrect. In the step Marcello was
trying to take in the dance, he was trying to explain something for
free, get something for nothing. It is an extremely common misstep, at
least in my field. You can join a discussion on Artificial General
Intelligence and watch people doing the same thing, left and right,
over and over again---constantly skipping over things they
don't understand, without realizing
that's what they're doing.}

{
 In an eyeblink it happens: putting a non-controlling causal node
behind something mysterious, a causal node that feels like an
explanation but isn't. The mistake takes place below
the level of words. It requires no special character flaw; it is how
human beings think by default, how they have thought since the ancient
times.}

{
 What you must avoid is \textit{skipping over the mysterious part};
you must linger at the mystery to confront it directly. There are many
words that can skip over mysteries, and some of them would be
legitimate in other
contexts---``complexity,'' for
example. But the essential mistake is that \textit{skip-over},
regardless of what causal node goes behind it. The skip-over is not a
thought, but a microthought. You have to pay close attention to catch
yourself at it. And when you train yourself to avoid skipping, it will
become a matter of instinct, not verbal reasoning. You have to
\textit{feel} which parts of your map are still blank, and more
importantly, pay attention to that feeling.}

{
 I suspect that in academia there is a huge pressure to sweep
problems under the rug so that you can present a paper with the
appearance of completeness. You'll get more kudos for a
seemingly complete model that includes some ``emergent
phenomena,'' versus an explicitly incomplete map
where the label says ``I got no clue how this part
works'' or ``then a miracle
occurs.'' A journal may not even accept the latter
paper, since who knows but that the unknown steps are really where
everything interesting happens? And yes, it sometimes happens that all
the non-magical parts of your map turn out to also be non-important.
That's the price you sometimes pay, for entering into
terra incognita and trying to solve problems \textit{incrementally.}
But that makes it even \textit{more} important to \textit{know} when
you aren't finished yet. Mostly, people
don't dare to enter terra incognita at all, for the
deadly fear of wasting their time.}

{
 And if you're working on a revolutionary AI
startup, there is an even huger pressure to sweep problems under the
rug; or you will have to admit to yourself that you
don't know how to build an AI yet, and your current
life plans will come crashing down in ruins around your ears. But
perhaps I am over-explaining, since skip-over happens by default in
humans; if you're looking for examples, just watch
people discussing religion or philosophy or spirituality or any science
in which they were not professionally trained.}

{
 Marcello and I developed a convention in our AI work: when we ran
into something we didn't understand, which was often,
we would say ``magic''---as in,
``$X$ magically does $Y$''---to remind
ourselves that \textit{here was an unsolved problem, a gap in our
understanding.} It is far better to say
``magic,'' than
``complexity'' or
``emergence''; the latter words
create an illusion of understanding. Wiser to say
``magic,'' and leave yourself a
placeholder, a reminder of work you will have to do later.}

\myendsectiontext

\mysection{Positive Bias: Look into the Dark}
\label{positive_bias}

{
 I am teaching a class, and I write upon the blackboard three
numbers: 2-4-6. ``I am thinking of a
rule,'' I say, ``which governs
sequences of three numbers. The sequence 2-4-6, as it so happens, obeys
this rule. Each of you will find, on your desk, a pile of index cards.
Write down a sequence of three numbers on a card, and
I'll mark it `Yes' for
fits the rule, or `No' for not fitting
the rule. Then you can write down another set of three numbers and ask
whether it fits again, and so on. When you're confident
that you know the rule, write down the rule on a card. You can test as
many triplets as you like.'' }

{
 Here's the record of one student's
guesses:\newline
}

\begin{equation*}
  \begin{array}{ll}
    4-6-2  & \text{No}\\
    4-6-8  & \text{Yes}\\
    10-12-14 & \text{Yes}
  \end{array}
\end{equation*}

{
 At this point the student wrote down their guess at the rule. What
do \textit{you} think the rule is? Would you have wanted to test
another triplet, and if so, what would it be? Take a moment to think
before continuing. }

{
 The challenge above is based on a classic experiment due to Peter
Wason, the 2-4-6 task. Although subjects given this task typically
expressed high confidence in their guesses, only 21\% of the subjects
successfully guessed the experimenter's real rule, and
replications since then have continued to show success rates of around
20\%.\footnote{Peter Cathcart Wason, ``On the Failure to
Eliminate Hypotheses in a Conceptual Task,''
\textit{Quarterly Journal of Experimental Psychology} 12, no. 3 (1960):
129--140, doi:10.1080/17470216008416717.\comment{1}}}

{
 The study was called ``On the failure to
eliminate hypotheses in a conceptual task.'' Subjects
who attempt the 2-4-6 task usually try to generate \textit{positive}
examples, rather than \textit{negative} examples---they apply the
hypothetical rule to generate a representative instance, and see if it
is labeled ``Yes.''}

{
 Thus, someone who forms the hypothesis ``numbers
increasing by two'' will test the triplet 8-10-12,
hear that it fits, and confidently announce the rule. Someone who forms
the hypothesis X-2X-3X will test the triplet 3-6-9, discover that it
fits, and then announce that rule.}

{
 In every case the actual rule is the same: the three numbers must
be in ascending order.}

{
 But to discover this, you would have to generate triplets that
\textit{shouldn't} fit, such as 20-23-26, and see if
they are labeled ``No.'' Which
people tend not to do, in this experiment. In some cases, subjects
devise, ``test,'' and announce rules
far more complicated than the actual answer.}

{
 This cognitive phenomenon is usually lumped in with
``confirmation bias.'' However, it
seems to me that the phenomenon of trying to test \textit{positive}
rather than \textit{negative} examples, ought to be distinguished from
the phenomenon of trying to preserve the belief you started with.
``Positive bias'' is sometimes used
as a synonym for ``confirmation
bias,'' and fits this particular flaw much better.}

{
 It once seemed that phlogiston theory could explain a flame going
out in an enclosed box (the air became saturated with phlogiston and no
more could be released), but phlogiston theory could just as well have
explained the flame \textit{not} going out. To notice this, you have to
search for negative examples instead of positive examples, look into
zero instead of one; which goes against the grain of what experiment
has shown to be human instinct.}

{
 For by instinct, we human beings only live in half the world.}

{
 One may be lectured on positive bias for days, and yet overlook it
in-the-moment. Positive bias is not something we do as a matter of
logic, or even as a matter of emotional attachment. The 2-4-6 task is
``cold,'' logical, not affectively
``hot.'' And yet the mistake is
sub-verbal, on the level of imagery, of instinctive reactions. Because
the problem doesn't arise from following a deliberate
rule that says ``Only think about positive
examples,'' it can't be solved just
by knowing verbally that ``We ought to think about
both positive and negative examples.'' Which example
automatically pops into your head? You have to learn, wordlessly, to
zag instead of zig. You have to learn to flinch toward the zero,
instead of away from it.}

{
 I have been writing for quite some time now on the notion that the
strength of a hypothesis is what it \textit{can't}
explain, not what it \textit{can}{}---if you are equally good at
explaining any outcome, you have zero knowledge. So to spot an
explanation that isn't helpful, it's
not enough to think of what it does explain very well---you also have
to search for results it \textit{couldn't} explain, and
this is the true strength of the theory.}

{
 So I said all this, and then I challenged the usefulness of
``emergence'' as a concept. One
commenter cited superconductivity and ferromagnetism as examples of
emergence. I replied that non-superconductivity and non-ferromagnetism
were also examples of emergence, which was the problem. But be it far
from me to criticize the commenter! Despite having read extensively on
``confirmation bias,'' I
didn't spot the
``gotcha'' in the 2-4-6 task the
first time I read about it. It's a subverbal
blink-reaction that has to be retrained. I'm still
working on it myself.}

{
 So much of a rationalist's skill is below the
level of words. It makes for challenging work in trying to convey the
Art through words. People will agree with you, but then, in the next
sentence, do something subdeliberative that goes in the opposite
direction. Not that I'm complaining! A major reason
I'm writing this is to observe what my words
\textit{haven't} conveyed.}

{
 Are you searching for positive examples of positive bias right
now, or sparing a fraction of your search on what positive bias should
lead you to \textit{not} see? Did you look toward light or darkness?}

\myendsectiontext


\bigskip

\mysection{Lawful Uncertainty}

{
 In \textit{Rational Choice in an Uncertain World}, Robyn Dawes
describes an experiment by Tversky:\footnote{Dawes, \textit{Rational Choice in An Uncertain World}; Yaacov
Schul and Ruth Mayo, ``Searching for Certainty in an
Uncertain World: The Difficulty of Giving Up the Experiential for the
Rational Mode of Thinking,'' \textit{Journal of
Behavioral Decision Making} 16, no. 2 (2003): 93--106,
doi:10.1002/bdm.434.\comment{1}}\supercomma\footnote{Amos Tversky and Ward Edwards, ``Information
versus Reward in Binary Choices,'' \textit{Journal of
Experimental Psychology} 71, no. 5 (1966): 680--683,
doi:10.1037/h0023123.\comment{2}}}

\begin{quotation}
{
 Many psychological experiments were conducted in the late 1950s
and early 1960s in which subjects were asked to predict the outcome of
an event that had a random component but yet had base-rate
predictability---for example, subjects were asked to predict whether
the next card the experimenter turned over would be red or blue in a
context in which 70\% of the cards were blue, but in which the sequence
of red and blue cards was totally random.}

{
 In such a situation, the strategy that will yield the highest
proportion of success is to predict the more common event. For example,
if 70\% of the cards are blue, then predicting blue on every trial
yields a 70\% success rate.}

{
 What subjects tended to do instead, however, was match
probabilities---that is, predict the more probable event with the
relative frequency with which it occurred. For example, subjects tended
to predict 70\% of the time that the blue card would occur and 30\% of
the time that the red card would occur. Such a strategy yields a 58\%
success rate, because the subjects are correct 70\% of the time when
the blue card occurs (which happens with probability .70) and 30\% of
the time when the red card occurs (which happens with probability .30);
(.70 {\texttimes} .70) + (.30 {\texttimes} .30) = .58.}

{
 In fact, subjects predict the more frequent event with a slightly
higher probability than that with which it occurs, but do not come
close to predicting its occurrence 100\% of the time, even when they
are paid for the accuracy of their predictions\,\ldots For example,
subjects who were paid a nickel for each correct prediction over a
thousand trials\,\ldots predicted [the more common event] 76\% of the
time.}
\end{quotation}

{
 Do not think that this experiment is about a minor flaw in
gambling strategies. It compactly illustrates the most important idea
in all of rationality.}

{
 Subjects just keep guessing red, as if they think they have some
way of predicting the random sequence. Of this experiment Dawes goes on
to say, ``Despite feedback through a thousand trials,
subjects cannot bring themselves to believe that the situation is one
in which they \textit{cannot} predict.''}

{
 But the error must go deeper than that. Even if subjects
\textit{think} they've come up with a hypothesis, they
don't have to \textit{actually bet} on that prediction
in order to test their hypothesis. They can say, ``Now
if \textit{this} hypothesis is correct, the next card will be
red''---and then just bet on blue. They can pick blue
each time, accumulating as many nickels as they can, while mentally
noting their private guesses for any patterns they thought they
spotted. If their predictions come out right, \textit{then} they can
switch to the newly discovered sequence.}

{
 I wouldn't fault a subject for continuing to
invent hypotheses---how could they know the sequence is truly beyond
their ability to predict? But I would fault a subject for
\textit{betting on the guesses}, when this wasn't
necessary to gather information, and literally \textit{hundreds} of
earlier guesses had been disconfirmed.}

{
 Can even a human be \textit{that} overconfident?}

{
 I would suspect that something simpler is going on---that the
all-blue strategy \textit{just didn't occur} to the
subjects.}

{
 People see a mix of mostly blue cards with some red, and suppose
that the optimal betting strategy must be a mix of mostly blue cards
with some red.}

{
 It is a \textit{counterintuitive} idea that, given incomplete
information, \textit{the optimal betting strategy does not resemble a
typical sequence of cards}.}

{
 It is a \textit{counterintuitive} idea that the optimal strategy
is to behave lawfully, even in an environment that has random
elements.}

{
 It seems like your behavior ought to be unpredictable, just like
the environment---but no! \textit{A random key does not open a random
lock just because they are ``both
random.''}}

{
 You don't fight fire with fire; you fight fire
with water. But this thought involves an extra step, a new concept not
directly activated by the problem statement, and so
it's not the first idea that comes to mind.}

{
 In the dilemma of the blue and red cards, our partial knowledge
tells us---on each and every round---that the best bet is blue. This
advice of our partial knowledge is the same on each and every round. If
30\% of the time we go against our partial knowledge and bet on red
instead, then we will do worse thereby---because now
we're being outright stupid, betting on what we know is
the less probable outcome.}

{
 If you bet on red every round, you would do as badly as you could
possibly do; you would be 100\% stupid. If you bet on red 30\% of the
time, faced with 30\% red cards, then you're making
yourself 30\% stupid.}

{
 When your knowledge is incomplete---meaning that the world will
seem to you to have an element of randomness---randomizing your actions
doesn't solve the problem. Randomizing your actions
takes you further from the target, not closer. In a world already
foggy, throwing away your intelligence just makes things worse.}

{
 It is a \textit{counterintuitive} idea that the optimal strategy
can be to \textit{think lawfully, even under conditions of
uncertainty}.}

{
 And so there are not many rationalists, for most who perceive a
chaotic world will try to fight chaos with chaos. You have to take an
extra step, and think of something that doesn't pop
right into your mind, in order to imagine fighting fire with something
that is not itself fire.}

{
 You have heard the unenlightened ones say,
``Rationality works fine for dealing with rational
people, but the world isn't
rational.'' But \textit{faced with an irrational
opponent, throwing away your own reason is not going to help you}.
There are lawful forms of thought that still generate the best
response, even when faced with an opponent who breaks those laws.
Decision theory does \textit{not} burst into flames and die when faced
with an opponent who disobeys decision theory.}

{
 This is no more obvious than the idea of betting all blue, faced
with a sequence of both blue and red cards. But each bet that you make
on red is an expected loss, and so too with every departure from the
Way in your own thinking.}

{
 How many \textit{Star Trek} episodes are thus refuted? How many
theories of AI?}

\myendsectiontext


\bigskip

\mysection{My Wild and Reckless Youth}

{
 It is said that parents do all the things they tell their children
not to do, which is how they know not to do them. }

{
 Long ago, in the unthinkably distant past, I was a devoted
Traditional Rationalist, conceiving myself skilled according to that
kind, yet I knew not the Way of Bayes. When the young Eliezer was
confronted with a mysterious-seeming question, the precepts of
Traditional Rationality did not stop him from devising a Mysterious
Answer. It is, by far, the most embarrassing mistake I made in my life,
and I still wince to think of it.}

{
 What was my mysterious answer to a mysterious question? This I
will not describe, for it would be a long tale and complicated. I was
young, and a mere Traditional Rationalist who knew not the teachings of
Tversky and Kahneman. I knew about Occam's Razor, but
not the conjunction fallacy. I thought I could get away with thinking
complicated thoughts myself, in the literary style of the complicated
thoughts I read in science books, not realizing that correct complexity
is only possible when every step is pinned down overwhelmingly. Today,
one of the chief pieces of advice I give to aspiring young rationalists
is ``Do not attempt long chains of reasoning or
complicated plans.''}

{
 Nothing more than this need be said: Even after I invented my
``answer,'' the phenomenon was still
a mystery unto me, and possessed the same quality of wondrous
impenetrability that it had at the start.}

{
 Make no mistake, that younger Eliezer was not stupid. All the
errors of which the young Eliezer was guilty are still being made today
by respected scientists in respected journals. It would have taken a
subtler skill to protect him than ever he was taught as a Traditional
Rationalist.}

{
 Indeed, the young Eliezer diligently and painstakingly followed
the injunctions of Traditional Rationality in the course of going
astray.}

{
 As a Traditional Rationalist, the young Eliezer was careful to
ensure that his Mysterious Answer made a bold prediction of future
experience. Namely, I expected future neurologists to discover that
neurons were exploiting quantum gravity, a la Sir Roger Penrose. This
required neurons to maintain a certain degree of quantum coherence,
which was something you could look for, and find or not find. Either
you observe that or you don't, right?}

{
 But my hypothesis made no \textit{retrospective} predictions.
According to Traditional Science, retrospective predictions
don't count---so why bother making them? To a Bayesian,
on the other hand, if a hypothesis does not \textit{today} have a
favorable likelihood ratio over ``I
don't know,'' it raises the question
of why you \textit{today} believe anything more complicated than
``I don't know.''
But I knew not the Way of Bayes, so I was not thinking about likelihood
ratios or focusing probability density. I had Made a Falsifiable
Prediction; was this not the Law?}

{
 As a Traditional Rationalist, the young Eliezer was careful not to
believe in magic, mysticism, carbon chauvinism, or anything of that
sort. I proudly professed of my Mysterious Answer,
``It is just physics like all the rest of
physics!'' As if you could save magic from being a
cognitive isomorph of magic, by calling it quantum gravity. But I knew
not the Way of Bayes, and did not see the level on which my idea was
isomorphic to magic. I gave my \textit{allegiance} to physics, but this
did not save me; what does probability theory know of allegiances? I
avoided everything that Traditional Rationality told me was forbidden,
but what was left was still magic.}

{
 Beyond a doubt, my allegiance to Traditional Rationality helped me
get out of the hole I dug myself into. If I hadn't been
a Traditional Rationalist, I would have been \textit{completely}
screwed. But Traditional Rationality still wasn't
enough to get it \textit{right.} It just led me into different mistakes
than the ones it had explicitly forbidden.}

{
 When I think about how my younger self very carefully followed the
rules of Traditional Rationality in the course of getting the answer
\textit{wrong}, it sheds light on the question of why people who call
themselves ``rationalists'' do not
rule the world. You need \textit{one whole hell of a lot} of
rationality before it does anything but lead you into new and
interesting mistakes.}

{
 Traditional Rationality is taught as an art, rather than a
science; you read the biography of famous physicists describing the
lessons life taught them, and you try to do what they tell you to do.
But you haven't lived their lives, and half of what
they're trying to describe is an instinct that has been
trained into them.}

{
 The way Traditional Rationality is designed, it would have been
acceptable for me to spend thirty years on my silly idea, so long as I
succeeded in falsifying it eventually, and was honest with myself about
what my theory predicted, and accepted the disproof when it arrived, et
cetera. This is enough to let the Ratchet of Science click forward, but
it's a little harsh on the people who waste thirty
years of their lives. Traditional Rationality is a walk, not a dance.
It's designed to get you to the truth
\textit{eventually}, and gives you all too much time to smell the
flowers along the way.}

{
 Traditional Rationalists can agree to disagree. Traditional
Rationality doesn't have the \textit{ideal} that
thinking is an exact art in which there is only one correct probability
estimate given the evidence. In Traditional Rationality,
you're allowed to guess, and then test your guess. But
experience has taught me that if you don't
\textit{know}, and you guess, you'll end up being
wrong.}

{
 The Way of Bayes is also an imprecise art, at least the way
I'm holding forth upon it. These essays are still
fumbling attempts to put into words lessons that would be better taught
by experience. But at least there's \textit{underlying}
math, plus experimental evidence from cognitive psychology on how
humans actually think. Maybe that will be enough to cross the
stratospherically high threshold required for a discipline that lets
you actually get it right, instead of just constraining you into
interesting new mistakes.}

\myendsectiontext

\mysection{Failing to Learn from History}

{
 Once upon a time, in my wild and reckless youth, when I knew not
the Way of Bayes, I gave a Mysterious Answer to a mysterious-seeming
question. Many failures occurred in sequence, but one mistake stands
out as most critical: My younger self did not realize that
\textit{solving a mystery should make it feel less confusing.} I was
trying to explain a Mysterious Phenomenon---which to me meant providing
a cause for it, fitting it into an integrated model of reality. Why
should this make the phenomenon less Mysterious, when that is its
nature? I was trying to \textit{explain} the Mysterious Phenomenon, not
render it (by some impossible alchemy) into a mundane phenomenon, a
phenomenon that wouldn't even call out for an unusual
explanation in the first place. }

{
 As a Traditional Rationalist, I knew the historical tales of
astrologers and astronomy, of alchemists and chemistry, of vitalists
and biology. But the Mysterious Phenomenon was not like this. It was
something \textit{new}, something stranger, something more difficult,
something that ordinary science had failed to explain for centuries---}

{
 {}---as if stars and matter and life had not been mysteries for
hundreds of years and thousands of years, from the dawn of human
thought right up until science finally solved them---}

{
 We learn about astronomy and chemistry and biology in school, and
it seems to us that these matters have \textit{always been} the proper
realm of science, that they have \textit{never been} mysterious. When
science dares to challenge a new Great Puzzle, the children of that
generation are skeptical, for they have never seen science explain
something that \textit{feels} mysterious to them. Science is only good
for explaining \textit{scientific} subjects, like stars and matter and
life.}

{
 I thought the lesson of history was that astrologers and
alchemists and vitalists had an innate character flaw, a tendency
toward mysterianism, which led them to come up with mysterious
explanations for non-mysterious subjects. But surely, if a phenomenon
really \textit{was} very weird, a weird explanation might be in order?}

{
 It was only afterward, when I began to see the mundane structure
inside the mystery, that I realized whose shoes I was standing in. Only
then did I realize how reasonable vitalism had seemed \textit{at the
time}, how \textit{surprising} and \textit{embarrassing} had been the
universe's reply of, ``Life is
mundane, and does not need a weird explanation.''}

{
 We read history but we don't \textit{live} it, we
don't \textit{experience} it. If only I had
\textit{personally} postulated astrological mysteries and then
discovered Newtonian mechanics, postulated alchemical mysteries and
then discovered chemistry, postulated vitalistic mysteries and then
discovered biology. I would have thought of my Mysterious Answer and
said to myself: \textit{No way am I falling for that again.}}

\myendsectiontext

\mysection{Making History Available}

{
 There is a habit of thought which I call the \textit{logical
fallacy of generalization from fictional evidence}. Journalists who,
for example, talk about the \textit{Terminator} movies in a report on
AI, do not usually treat \textit{Terminator} as a prophecy or fixed
truth. But the movie is recalled---is available{}---as if it were an
illustrative historical case. As if the journalist had seen it happen
on some other planet, so that it might well happen here. More on this
in Section 7 of ``Cognitive biases potentially
affecting judgment of global
risks.''\footnote{Eliezer Yudkowsky, ``Cognitive Biases
Potentially Affecting Judgment of Global Risks,'' in
\textit{Global Catastrophic Risks}, ed. Nick Bostrom and Milan M.
\'Cirkovi\'c (New York: Oxford University Press, 2008), 91--119.\comment{1}} }

{
 There is an inverse error to generalizing from fictional evidence:
failing to be sufficiently moved by \textit{historical} evidence. The
trouble with generalizing from fictional evidence is that it is
fiction---it never actually happened. It's not drawn
from the same distribution as this, our real universe; fiction differs
from reality in systematic ways. But history \textit{has} happened, and
\textit{should} be available.}

{
 In our ancestral environment, there were no movies; what you saw
with your own eyes was true. Is it any wonder that fictions we see in
lifelike moving pictures have too great an impact on us? Conversely,
things that \textit{really happened}, we encounter as ink on paper;
they happened, but we never \textit{saw} them happen. We
don't remember them happening to us.}

{
 The inverse error is to treat history as mere story, process it
with the same part of your mind that handles the novels you read. You
may say with your lips that it is
``truth,'' rather than
``fiction,'' but that
doesn't mean you are being moved as much as you should
be. Many biases involve being insufficiently moved by dry, abstract
information.}

{
 Once upon a time, I gave a Mysterious Answer to a mysterious
question, not realizing that I was making exactly the same mistake as
astrologers devising mystical explanations for the stars, or alchemists
devising magical properties of matter, or vitalists postulating an
opaque ``élan vital'' to explain all
of biology.}

{
 When I finally realized whose shoes I was standing in, there was a
sudden shock of unexpected connection with the past. I realized that
the invention and destruction of vitalism---which I had only read about
in books---had \textit{actually happened to real people}, who
experienced it much the same way I experienced the invention and
destruction of my own mysterious answer. And I also realized that if I
had actually \textit{experienced} the past---if I had lived through
past scientific revolutions myself, rather than reading about them in
history books---I probably would \textit{not} have made the same
mistake again. I would not have come up with \textit{another}
mysterious answer; the first thousand lessons would have hammered home
the moral.}

{
 So (I thought), to feel sufficiently the force of history, I
should try to approximate the thoughts of an Eliezer who \textit{had}
lived through history---I should try to think as if everything I read
about in history books had actually happened to me. (With appropriate
reweighting for the availability bias of history books---I should
remember being a thousand peasants for every ruler.) I should immerse
myself in history, imagine \textit{living} through eras I only saw as
ink on paper.}

{
 Why should I remember the Wright Brothers' first
flight? I was not there. But as a rationalist, could I dare to
\textit{not} remember, when the event actually happened? Is there so
much difference between seeing an event through your eyes---which is
actually a causal chain involving reflected photons, not a direct
connection---and seeing an event through a history book? Photons and
history books both descend by causal chains from the event itself.}

{
 I had to overcome the false amnesia of being born at a particular
time. I had to recall---make available---\textit{all} the memories, not
just the memories which, by mere coincidence, belonged to myself and my
own era.}

{
 The Earth became older, of a sudden.}

{
 To my former memory, the United States had always existed---there
was never a time when there was no United States. I had not remembered,
until that time, how the Roman Empire rose, and brought peace and
order, and lasted through so many centuries, until I forgot that things
had ever been otherwise; and yet the Empire fell, and barbarians
overran my city, and the learning that I had possessed was lost. The
modern world became more fragile to my eyes; it was not the first
modern world.}

{
 So many mistakes, made over and over and \textit{over} again,
because I did not remember making them, in every era I never lived\,\ldots}

{
 And to think, people sometimes wonder if overcoming bias is
important.}

{
 Don't you remember how many times your biases have
killed you? You don't? I've noticed
that sudden amnesia often follows a fatal mistake. But take it from me,
it happened. I remember; I wasn't there.}

{
 So the next time you doubt the strangeness of the future, remember
how you were born in a hunter-gatherer tribe ten thousand years ago,
when no one knew of Science at all. Remember how you were shocked, to
the depths of your being, when Science explained the great and terrible
sacred mysteries that you once revered so highly. Remember how you once
believed that you could fly by eating the right mushrooms, and then you
accepted with disappointment that you would never fly, and then you
flew. Remember how you had always thought that slavery was right and
proper, and then you changed your mind. Don't imagine
how you \textit{could} have predicted the change, for that is amnesia.
\textit{Remember} that, in fact, you did not guess. Remember how,
century after century, the world changed in ways you did not guess.}

{
 Maybe then you will be less shocked by what happens next.}

\myendsectiontext


\bigskip

\mysection{Explain/Worship/Ignore?}
\label{explain_worship_ignore}

{
 As our tribe wanders through the grasslands, searching for fruit
trees and prey, it happens every now and then that water pours down
from the sky. }

{
 ``Why does water sometimes fall from the
sky?'' I ask the bearded wise man of our tribe.}

{
 He thinks for a moment, this question having never occurred to him
before, and then says, ``From time to time, the sky
spirits battle, and when they do, their blood drips from the
sky.''}

{
 ``Where do the sky spirits come
from?'' I ask.}

{
 His voice drops to a whisper. ``From the before
time. From the long long ago.''}

{
 When it rains, and you don't know why, you have
several options. First, you could simply not ask why---not follow up on
the question, or never think of the question in the first place. This
is the Ignore command, which the bearded wise man originally selected.
Second, you could try to devise some sort of explanation, the Explain
command, as the bearded man did in response to your first question.
Third, you could enjoy the sensation of mysteriousness---the Worship
command.}

{
 Now, as you are bound to notice from this story, each time you
select Explain, the best-case scenario is that you get an explanation,
such as ``sky spirits.'' But then
this explanation itself is subject to the same dilemma---Explain,
Worship, or Ignore? Each time you hit Explain, science grinds for a
while, returns an explanation, and then another dialog box pops up. As
good rationalists, we feel duty-bound to keep hitting Explain, but it
seems like a road that has no end.}

{
 You hit Explain for life, and get chemistry; you hit Explain for
chemistry, and get atoms; you hit Explain for atoms, and get electrons
and nuclei; you hit Explain for nuclei, and get quantum chromodynamics
and quarks; you hit Explain for how the quarks got there, and get back
the Big Bang\,\ldots}

{
 We can hit Explain for the Big Bang, and wait while science grinds
through its process, and maybe someday it will return a perfectly good
explanation. But then that will just bring up another dialog box. So,
if we continue long enough, we must come to a \textit{special} dialog
box, a \textit{new} option, an Explanation That Needs No Explanation, a
place where the chain ends---and this, maybe, is the only explanation
worth knowing.}

{
 There---I just hit Worship.}

{
 Never forget that there are many more ways to worship something
than lighting candles around an altar.}

{
 If I'd said, ``Huh, that does
seem paradoxical. I wonder how the apparent paradox is
resolved?'' then I would have hit Explain, which does
sometimes take a while to produce an answer.}

{
 And if the whole issue seems to you unimportant, or irrelevant, or
if you'd rather put off thinking about it until
tomorrow, than you have hit Ignore.}

{
 Select your option wisely.}

\myendsectiontext

\mysection{``Science'' as Curiosity{}-Stopper}

{
 Imagine that I, in full view of live television cameras, raised my
hands and chanted \textit{abracadabra} and caused a brilliant light to
be born, flaring in empty space beyond my outstretched hands. Imagine
that I committed this act of blatant, unmistakeable sorcery under the
full supervision of James Randi and all skeptical armies. Most people,
I think, would be \textit{fairly curious} as to what was going on. }

{
 But now suppose instead that I don't go on
television. I do not wish to share the power, nor the truth behind it.
I want to keep my sorcery secret. And yet I also want to cast my spells
whenever and wherever I please. I want to cast my brilliant flare of
light so that I can read a book on the train---without anyone becoming
curious. Is there a spell that stops curiosity?}

{
 Yes indeed! Whenever anyone asks ``How did you do
that?,'' I just say
``Science!''}

{
 It's not a real explanation, so much as a
curiosity-stopper. It doesn't tell you whether the
light will brighten or fade, change color in hue or saturation, and it
certainly doesn't tell you how to make a similar light
yourself. You don't actually \textit{know} anything
more than you knew before I said the magic word. But you turn away,
satisfied that nothing unusual is going on.}

{
 Better yet, the same trick works with a standard light switch.}

{
 Flip a switch and a light bulb turns on. Why?}

{
 In school, one is taught that the password to the light bulb is
``Electricity!'' By now, I hope,
you're wary of marking the light bulb
``understood'' on such a basis. Does
saying ``Electricity!'' let you do
calculations that will control your anticipation of experience? There
is, at the least, a great deal more to learn. (Physicists should ignore
this paragraph and substitute a problem in evolutionary theory, where
the substance of the theory is again in calculations that few people
know how to perform.)}

{
 If you thought the light bulb was \textit{scientifically
inexplicable}, it would seize the \textit{entirety} of your attention.
You would drop whatever else you were doing, and focus on that light
bulb.}

{
 But what does the phrase ``scientifically
explicable'' mean? It means that someone
\textit{else} knows how the light bulb works. When you are told the
light bulb is ``scientifically
explicable,'' you don't know more
than you knew earlier; you don't know whether the light
bulb will brighten or fade. But because someone \textit{else} knows, it
devalues the knowledge in your eyes. You become less curious.}

{
 Someone is bound to say, ``If the light bulb were
unknown to science, you could gain fame and fortune by investigating
it.'' But I'm not talking about
greed. I'm not talking about career ambition.
I'm talking about the raw emotion of curiosity---the
feeling of being intrigued. Why should \textit{your} curiosity be
diminished because someone \textit{else}, not you, knows how the light
bulb works? Is this not spite? It's not enough for
\textit{you} to know; other people must also be ignorant, or you
won't be happy?}

{
 There are goods that knowledge may serve besides curiosity, such
as the social utility of technology. For these instrumental goods, it
matters whether some other entity in local space already knows. But for
my own curiosity, why should it matter?}

{
 Besides, consider the consequences if you permit
``Someone else knows the answer'' to
function as a curiosity-stopper. One day you walk into your living room
and see a giant green elephant, seemingly hovering in midair,
surrounded by an aura of silver light.}

{
 ``What the heck?'' you say.}

{
 And a voice comes from above the elephant, saying,}

\begin{center}
  \textsc{Somebody already knows why this elephant is here.}
\end{center}

{
 ``Oh,'' you say,
``in that case, never mind,'' and
walk on to the kitchen.}

{
 I don't know the grand unified theory for this
universe's laws of physics. I also
don't know much about human anatomy with the exception
of the brain. I couldn't point out on my body where my
kidneys are, and I can't recall offhand what my liver
does. (I am not proud of this. Alas, with all the math I need to study,
I'm not likely to learn anatomy anytime soon.)}

{
 Should I, so far as \textit{curiosity} is concerned, be more
intrigued by my ignorance of the ultimate laws of physics, than the
fact that I don't know much about what goes on inside
my own body?}

{
 If I raised my hands and cast a light spell, you would be
intrigued. Should you be any \textit{less} intrigued by the very fact
that I raised my hands? When you raise your arm and wave a hand around,
this act of will is coordinated by (among other brain areas) your
cerebellum. I bet you don't know how the cerebellum
works. \textit{I} know a little---though only the gross details, not
enough to perform calculations\,\ldots but so what? What does that
matter, if \textit{you} don't know? Why should there be
a double standard of curiosity for sorcery and hand motions?}

{
 Look at yourself in the mirror. Do you know what
you're looking at? Do you know what looks out from
behind your eyes? Do you know what you are? Some of that answer,
Science knows, and some of it Science does not. But why should that
distinction matter to your curiosity, if \textit{you}
don't know?}

{
 Do you know how your knees work? Do you know how your shoes were
made? Do you know why your computer monitor glows? Do you know why
water is wet?}

{
 The world around you is full of puzzles. Prioritize, if you must.
But do not complain that cruel Science has emptied the world of
mystery. With reasoning such as that, I could get you to overlook an
elephant in your living room.}

\myendsectiontext

\mysection{Truly Part of You}
\label{truly_part_of_you}

{
 A classic paper by Drew McDermott, ``Artificial
Intelligence Meets Natural Stupidity,'' criticized AI
programs that would try to represent notions like \textit{happiness is
a state of mind} using a semantic network:\footnote{Drew McDermott, ``Artificial Intelligence
Meets Natural Stupidity,'' \textit{SIGART
Newsletter}, no. 57 (1976): 4--9, doi:10.1145/1045339.1045340.\comment{1}}}

\begin{center}
 \texttt{HAPPINESS -{}-{}-IS-A-{}-{}-{\textgreater} STATE-OF-MIND}
\end{center}

{
 And of course there's nothing \textit{inside} the
\texttt{HAPPINESS} node; it's just a naked \textsc{lisp} token with a
suggestive English name.}

{
 So, McDermott says, ``A good test for the
disciplined programmer is to try using gensyms in key places and see if
he still admires his system. For example, if \texttt{STATE-OF-MIND} is renamed
\texttt{G1073} \ldots'' then we would have \texttt{IS-A(HAPPINESS,
G1073)} ``which looks much more
dubious.''}

{
 Or as I would slightly rephrase the idea: If you substituted
randomized symbols for \textit{all} the suggestive English names, you
would be completely unable to figure out what \texttt{G1071(G1072, G1073)}
meant. Was the AI program meant to represent hamburgers? Apples?
Happiness? Who knows? \textit{If you delete the suggestive English
names, they don't grow back.}}

{
 Suppose a physicist tells you that ``Light is
waves,'' and you \textit{believe} the physicist. You
now have a little network in your head that says:}

\begin{center}
\texttt{IS-A(LIGHT, WAVES)}.
\end{center}

{
 If someone asks you ``What is light made
of?'' you'll be able to say
``Waves!'' }

{
 As McDermott says, ``The whole problem is getting
the hearer to notice what it has been told. Not
`understand,' but
`notice.'\,'' Suppose
that instead the physicist told you, ``Light is made
of little curvy things.'' (Not true, by the way.)
Would you \textit{notice} any difference of anticipated experience?}

{
 How can you realize that you shouldn't trust your
seeming knowledge that ``light is
waves''? One test you could apply is asking,
``Could I \textit{regenerate} this knowledge if it
were somehow deleted from my mind?''}

{
 This is similar in spirit to scrambling the names of suggestively
named \textsc{lisp} tokens in your AI program, and seeing if someone else can
figure out what they allegedly
``refer'' to. It's
also similar in spirit to observing that an Artificial Arithmetician
programmed to record and play back}

\begin{center}
\texttt{Plus-Of(Seven, Six) = Thirteen}
\end{center}

{
 can't regenerate the knowledge if you delete it
from memory, until another human re-enters it in the database. Just as
if you forgot that ``light is
waves,'' you couldn't get back the
knowledge except the same way you got the knowledge to begin with---by
asking a physicist. You couldn't generate the knowledge
for yourself, the way that physicists originally generated it. }

{
 The same experiences that lead us to formulate a belief, connect
that belief to other knowledge and sensory input and motor output. If
you see a beaver chewing a log, then you know what this
thing-that-chews-through-logs looks like, and you will be able to
recognize it on future occasions whether it is called a
``beaver'' or not. But if you
acquire your beliefs about beavers by someone else telling you facts
about ``beavers,'' you may not be
able to recognize a beaver when you see one.}

{
 This is the terrible danger of trying to \textit{tell} an
Artificial Intelligence facts that it could not learn for itself. It is
also the terrible danger of trying to \textit{tell} someone about
physics that they cannot verify for themselves. For what physicists
mean by ``wave'' is not
``little squiggly thing'' but a
purely mathematical concept.}

{
 As Davidson observes, if you believe that
``beavers'' live in deserts, are
pure white in color, and weigh 300 pounds when adult, then you do not
have any beliefs \textit{about} beavers, true or false. Your belief
about ``beavers'' is not right
enough to be wrong.\footnote{Richard Rorty, ``Out of the Matrix: How the
Late Philosopher Donald Davidson Showed That Reality
Can't Be an Illusion,'' \textit{The
Boston Globe} (October 2003).\comment{2}} If you don't
have enough experience to regenerate beliefs when they are deleted,
then do you have enough experience to connect that belief to anything
at all? Wittgenstein: ``A wheel that can be turned
though nothing else moves with it, is not part of the
mechanism.''}

{
 Almost as soon as I started reading about AI---even before I read
McDermott---I realized it would be \textit{a really good idea} to
always ask myself: ``How would I regenerate this
knowledge if it were deleted from my mind?''}

{
 The deeper the deletion, the stricter the test. If all proofs of
the Pythagorean Theorem were deleted from my mind, could I re-prove it?
I think so. If all knowledge of the Pythagorean Theorem were deleted
from my mind, would I notice the Pythagorean Theorem to re-prove?
That's harder to boast, without putting it to the test;
but if you handed me a right triangle with sides of length 3 and 4, and
told me that the length of the hypotenuse was calculable, I think I
would be able to calculate it, if I still knew all the rest of my
math.}

{
 What about the notion of \textit{mathematical proof}? If no one
had ever told it to me, would I be able to reinvent \textit{that} on
the basis of other beliefs I possess? There was a time when humanity
did not have such a concept. Someone must have invented it. What was it
that they noticed? Would I notice if I saw something equally novel and
equally important? Would I be able to think that far outside the box?}

{
 How much of your knowledge could you regenerate? From how deep a
deletion? It's not just a test to cast out
insufficiently connected beliefs. It's a way of
absorbing \textit{a fountain of knowledge, not just one fact.}}

{
 A shepherd builds a counting system that works by throwing a
pebble into a bucket whenever a sheep leaves the fold, and taking a
pebble out whenever a sheep returns. If you, the apprentice, do not
understand this system---if it is magic that works for no apparent
reason---then you will not know what to do if you accidentally drop an
extra pebble into the bucket. That which you cannot make yourself, you
cannot \textit{remake} when the situation calls for it. You cannot go
back to the source, tweak one of the parameter settings, and regenerate
the output, without the source. If ``two plus four
equals six'' is a brute fact unto you, and then one
of the elements changes to ``five,''
how are you to know that ``two plus five equals
seven'' when you were simply \textit{told} that
``two plus four equals six''?}

{
 If you see a small plant that drops a seed whenever a bird passes
it, it will not occur to you that you can use this plant to partially
automate the sheep-counter. Though you learned something that the
original maker would use to improve on their invention, you
can't go back to the source and re-create it.}

{
 When you contain the source of a thought, that thought can change
along with you as you acquire new knowledge and new skills. When you
contain the source of a thought, it becomes truly a part of you and
grows along with you.}

{
 Strive to make yourself the source of every thought worth
thinking. If the thought originally came from outside, make sure it
comes from inside as well. Continually ask yourself:
``How would I regenerate the thought if it were
deleted?'' When you have an answer, imagine
\textit{that} knowledge being deleted as well. And when you find a
fountain, see what else it can pour.}

\myendsectiontext


\bigskip

\mysectionnn{Interlude: The Simple Truth}
\label{the_simple_truth}

\begin{quote}
{
 I remember this paper I wrote on existentialism. My teacher gave
it back with an F. She'd underlined true and truth
wherever it appeared in the essay, probably about twenty times, with a
question mark beside each. She wanted to know what I meant by truth.}

{\raggedleft
 {}---Danielle Egan, journalist
\par}
\end{quote}


{
 This essay is meant to restore a naive view of truth.}

{
 Someone says to you: ``My miracle snake oil can
rid you of lung cancer in just three weeks.'' You
reply: ``Didn't a clinical study show
this claim to be untrue?'' The one returns:
``This notion of
`truth' is quite naive; what do you mean
by `true'?''}

{
 Many people, so questioned, don't know how to
answer in exquisitely rigorous detail. Nonetheless they would not be
wise to abandon the concept of
``truth.'' There was a time when no
one knew the equations of gravity in exquisitely rigorous detail, yet
if you walked off a cliff, you would fall.}

{
 Often I have seen---especially on Internet mailing lists---that
amidst other conversation, someone says ``$X$ is
true,'' and then an argument breaks out over the use
of the word ``true.'' This essay is
\textit{not} meant as an encyclopedic reference for that argument.
Rather, I hope the arguers will read this essay, and then go back to
whatever they were discussing before someone questioned the nature of
truth.}

{
 In this essay I pose questions. If you see what seems like a
really obvious answer, it's probably the answer I
intend. The obvious choice isn't \textit{always} the
best choice, but sometimes, by golly, it \textit{is}. I
don't stop looking as soon I find an obvious answer,
but if I go on looking, and the obvious-seeming answer \textit{still}
seems obvious, I don't feel guilty about keeping it.
Oh, sure, everyone \textit{thinks} two plus two is four, everyone
\textit{says} two plus two is four, and in the mere mundane drudgery of
everyday life everyone \textit{behaves} as if two plus two is four, but
what does two plus two \textit{really, ultimately} equal? As near as I
can figure, four. It's still four even if I intone the
question in a solemn, portentous tone of voice. Too simple, you say?
Maybe, on this occasion, life doesn't \textit{need} to
be complicated. Wouldn't that be refreshing?}

{
 If you are one of those fortunate folk to whom the question seems
trivial at the outset, I hope it still seems trivial at the finish. If
you find yourself stumped by deep and meaningful questions, remember
that if you know exactly how a system works, and could build one
yourself out of buckets and pebbles, it should not be a mystery to
you.}

{
 If confusion threatens when you interpret a metaphor as a
metaphor, try taking everything \textit{completely literally.}}

\hr

{
 Imagine that in an era before recorded history or formal
mathematics, I am a shepherd and I have trouble tracking my sheep. My
sheep sleep in an enclosure, a fold; and the enclosure is high enough
to guard my sheep from wolves that roam by night. Each day I must
release my sheep from the fold to pasture and graze; each night I must
find my sheep and return them to the fold. If a sheep is left outside,
I will find its body the next morning, killed and half-eaten by wolves.
But it is so discouraging, to scour the fields for hours, looking for
one last sheep, when I know that probably all the sheep are in the
fold. Sometimes I give up early, and usually I get away with it; but
around a tenth of the time there is a dead sheep the next morning.}

{
 If only there were some way to divine whether sheep are still
grazing, without the inconvenience of looking! I try several methods: I
toss the divination sticks of my tribe; I train my psychic powers to
locate sheep through clairvoyance; I search carefully for reasons to
believe all the sheep are in the fold. It makes no difference. Around a
tenth of the times I turn in early, I find a dead sheep the next
morning. Perhaps I realize that my methods aren't
working, and perhaps I carefully excuse each failure; but my dilemma is
still the same. I can spend an hour searching every possible nook and
cranny, when most of the time there are no remaining sheep; or I can go
to sleep early and lose, on the average, one-tenth of a sheep.}

{
 Late one afternoon I feel especially tired. I toss the divination
sticks and the divination sticks say that all the sheep have returned.
I visualize each nook and cranny, and I don't imagine
scrying any sheep. I'm still not confident enough, so I
look inside the fold and it seems like there are a lot of sheep, and I
review my earlier efforts and decide that I was especially diligent.
This dissipates my anxiety, and I go to sleep. The next morning I
discover \textit{two} dead sheep. Something inside me snaps, and I
begin thinking creatively.}

{
 That day, loud hammering noises come from the gate of the
sheepfold's enclosure.}

{
 The next morning, I open the gate of the enclosure only a little
way, and as each sheep passes out of the enclosure, I drop a pebble
into a bucket nailed up next to the door. In the afternoon, as each
returning sheep passes by, I take one pebble out of the bucket. When
there are no pebbles left in the bucket, I can stop searching and turn
in for the night. It is a \textit{brilliant} notion. It will
revolutionize shepherding.}

{
 That was the theory. In practice, it took considerable refinement
before the method worked reliably. Several times I searched for hours
and didn't find any sheep, and the next morning there
were no stragglers. On each of these occasions it required deep thought
to figure out where my bucket system had failed. On returning from one
fruitless search, I thought back and realized that the bucket already
contained pebbles when I started; this, it turned out, was a bad idea.
Another time I randomly tossed pebbles into the bucket, to amuse
myself, between the morning and the afternoon; this too was a bad idea,
as I realized after searching for a few hours. But I practiced my
pebblecraft, and became a reasonably proficient pebblecrafter.}

{
 One afternoon, a man richly attired in white robes, leafy laurels,
sandals, and business suit trudges in along the sandy trail that leads
to my pastures.}

{
 ``Can I help you?'' I inquire.}

{
 The man takes a badge from his coat and flips it open, proving
beyond the shadow of a doubt that he is Markos Sophisticus Maximus, a
delegate from the Senate of Rum. (One might wonder whether another
could steal the badge; but so great is the power of these badges that
if any other were to use them, they would in that instant be
\textit{transformed} into Markos.)}

{
 ``Call me Mark,'' he says.
``I'm here to confiscate the magic
pebbles, in the name of the Senate; artifacts of such great power must
not fall into ignorant hands.''}

{
 ``That bleedin'
apprentice,'' I grouse under my breath,
``he's been yakkin' to
the villagers again.'' Then I look at
Mark's stern face, and sigh. ``They
aren't magic pebbles,'' I say aloud.
``Just ordinary stones I picked up from the
ground.''}

{
 A flicker of confusion crosses Mark's face, then
he brightens again. ``I'm here for the
magic bucket!'' he declares.}

{
 ``It's not a magic
bucket,'' I say wearily. ``I used to
keep dirty socks in it.''}

{
 Mark's face is puzzled. ``Then
where is the magic?'' he demands.}

{
 An interesting question. ``It's
hard to explain,'' I say.}

{
 My current apprentice, Autrey, attracted by the commotion, wanders
over and volunteers his explanation:
``It's the level of pebbles in the
bucket,'' Autrey says.
``There's a magic level of pebbles,
and you have to get the level just right, or it doesn't
work. If you throw in more pebbles, or take some out, the bucket
won't be at the magic level anymore. Right now, the
magic level is,'' Autrey peers into the bucket,
``about one-third full.''}

{
 ``I see!'' Mark says excitedly.
From his back pocket Mark takes out his own bucket, and a heap of
pebbles. Then he grabs a few handfuls of pebbles, and stuffs them into
the bucket. Then Mark looks into the bucket, noting how many pebbles
are there. ``There we go,'' Mark
says, ``the magic level of this bucket is half full.
Like that?''}

{
 ``No!'' Autrey says sharply.
``Half full is not the magic level. The magic level is
about one-third. Half full is definitely unmagic. Furthermore,
you're using the wrong bucket.''}

{
 Mark turns to me, puzzled. ``I thought you said
the bucket wasn't magic?''}

{
 ``It's not,'' I
say. A sheep passes out through the gate, and I toss another pebble
into the bucket. ``Besides, I'm
watching the sheep. Talk to Autrey.''}

{
 Mark dubiously eyes the pebble I tossed in, but decides to
temporarily shelve the question. Mark turns to Autrey and draws himself
up haughtily. ``It's a free
country,'' Mark says, ``under the
benevolent dictatorship of the Senate, of course. I can drop whichever
pebbles I like into whatever bucket I like.''}

{
 Autrey considers this. ``No you
can't,'' he says finally,
``there won't be any
magic.''}

{
 ``Look,'' says Mark patiently,
``I watched you carefully. You looked in your bucket,
checked the level of pebbles, and called that the magic level. I did
exactly the same thing.''}

{
 ``That's not how it
works,'' says Autrey.}

{
 ``Oh, I see,'' says Mark,
``It's not the level of pebbles in
\textit{my} bucket that's magic, it's
the level of pebbles in \textit{your} bucket. Is that what you claim?
What makes your bucket so much better than mine,
huh?''}

{
 ``Well,'' says Autrey,
``if we were to empty your bucket, and then pour all
the pebbles from my bucket into your bucket, then your bucket would
have the magic level. There's also a procedure we can
use to check if your bucket has the magic level, if we know that my
bucket has the magic level; we call that a bucket compare
operation.''}

{
 Another sheep passes, and I toss in another pebble.}

{
 ``He just tossed in another
pebble!'' Mark says. ``And I suppose
you claim the new level is also magic? I could toss pebbles into your
bucket until the level was the same as mine, and then our buckets would
agree. You're just comparing my bucket to your bucket
to determine whether \textit{you} think the level is
`magic' or not. Well, I think
\textit{your} bucket isn't magic, because it
doesn't have the same level of pebbles as mine. So
there!''}

{
 ``Wait,'' says Autrey,
``you don't
understand---''}

{
 ``By `magic
level,' you mean simply the level of pebbles in your
own bucket. And when I say `magic
level,' I mean the level of pebbles in my bucket. Thus
you look at my bucket and say it `isn't
magic,' but the word
`magic' means different things to
different people. You need to specify \textit{whose} magic it is. You
should say that my bucket doesn't have
`Autrey's magic level,'
and I say that your bucket doesn't have
`Mark's magic level.'
That way, the apparent contradiction goes away.''}

{
 ``But---'' says Autrey
helplessly.}

{
 ``Different people can have different buckets
with different levels of pebbles, which proves this business about
`magic' is completely arbitrary and
subjective.''}

{
 ``Mark,'' I say,
``did anyone tell you what these pebbles
\textit{do}?''}

{
 ``\textit{Do?}'' says Mark.
``I thought they were just magic.''}

{
 ``If the pebbles didn't do
anything,'' says Autrey, ``our ISO
9000 process efficiency auditor would eliminate the procedure from our
daily work.''}

{
 ``What's your
auditor's name?''}

{
 ``Darwin,'' says Autrey.}

{
 ``Hm,'' says Mark.
``Charles does have a reputation as a strict auditor.
So do the pebbles bless the flocks, and cause the increase of
sheep?''}

{
 ``No,'' I say.
``The virtue of the pebbles is this; if we look into
the bucket and see the bucket is empty of pebbles, we know the pastures
are likewise empty of sheep. If we do not use the bucket, we must
search and search until dark, lest one last sheep remain. Or if we stop
our work early, then sometimes the next morning we find a dead sheep,
for the wolves savage any sheep left outside. If we look in the bucket,
we know when all the sheep are home, and we can retire without
fear.''}

{
 Mark considers this. ``That sounds rather
implausible,'' he says eventually.
``Did you consider using divination sticks? Divination
sticks are infallible, or at least, anyone who says they are fallible
is burned at the stake. This is an extremely painful way to die; it
follows that divination sticks are infallible.''}

{
 ``You're welcome to use
divination sticks if you like,'' I say.}

{
 ``Oh, good heavens, of course
not,'' says Mark. ``They work
infallibly, with absolute perfection on every occasion, as befits such
blessed instruments; but what if there were a dead sheep the next
morning? I only use the divination sticks when there is no possibility
of their being proven wrong. Otherwise I might be burned alive. So how
does your magic bucket work?''}

{
 How does the bucket work\,\ldots ? I'd better start
with the simplest possible case.
``Well,'' I say,
``suppose the pastures are empty, and the bucket
isn't empty. Then we'll waste hours
looking for a sheep that isn't there. And if there are
sheep in the pastures, but the bucket is empty, then Autrey and I will
turn in too early, and we'll find dead sheep the next
morning. So an empty bucket is magical if and only if the pastures are
empty---''}

{
 ``Hold on,'' says Autrey.
``That sounds like a vacuous tautology to me.
Aren't an empty bucket and empty pastures obviously the
same thing?''}

{
 ``It's not
vacuous,'' I say.
``Here's an analogy: The logician
Alfred Tarski once said that the assertion `Snow is
white' is true if and only if snow is white. If you can
understand that, you should be able to see why an empty bucket is
magical if and only if the pastures are empty of
sheep.''}

{
 ``Hold on,'' says Mark.
``These are \textit{buckets}. They
don't have anything to do with \textit{sheep}. Buckets
and sheep are obviously completely different. There's
no way the sheep can ever interact with the
bucket.''}

{
 ``Then where do \textit{you} think the magic
comes from?'' inquires Autrey.}

{
 Mark considers. ``You said you could compare two
buckets to check if they had the same level\,\ldots I can see how buckets
can interact with buckets. Maybe when you get a large collection of
buckets, and they all have the same level,
\textit{that's} what generates the magic.
I'll call that the coherentist theory of magic
buckets.''}

{
 ``Interesting,'' says Autrey.
``I know that my master is working on a system with
multiple buckets---he says it might work better because of
`redundancy' and `error
correction.' That sounds like coherentism to
me.''}

{
 ``They're not quite the
same---'' I start to say.}

{
 ``Let's test the coherentism
theory of magic,'' says Autrey. ``I
can see you've got five more buckets in your back
pocket. I'll hand you the bucket we're
using, and then you can fill up your other buckets to the same
level---''}

{
 Mark recoils in horror. ``Stop! These buckets
have been passed down in my family for generations, and
they've always had the same level! If I accept your
bucket, my bucket collection will become less coherent, and the magic
will go away!''}

{
 ``But your \textit{current} buckets
don't have anything to do with the
sheep!'' protests Autrey.}

{
 Mark looks exasperated. ``Look,
I've explained before, there's
obviously no way that sheep can interact with buckets. Buckets can only
interact with other buckets.''}

{
 ``I toss in a pebble whenever a sheep
passes,'' I point out.}

{
 ``When a sheep passes, you toss in a
pebble?'' Mark says. ``What does
that have to do with anything?''}

{
 ``It's an interaction between the
sheep and the pebbles,'' I reply.}

{
 ``No, it's an interaction between
the pebbles and \textit{you},'' Mark says.
``The magic doesn't come from the
sheep, it comes from \textit{you}. Mere sheep are obviously nonmagical.
The magic has to come from \textit{somewhere}, on the way to the
bucket.''}

{
 I point at a wooden mechanism perched on the gate.
``Do you see that flap of cloth hanging down from that
wooden contraption? We're still fiddling with that---it
doesn't work reliably---but when sheep pass through,
they disturb the cloth. When the cloth moves aside, a pebble drops out
of a reservoir and falls into the bucket. That way, Autrey and I
won't have to toss in the pebbles
ourselves.''}

{
 Mark furrows his brow. ``I don't
quite follow you\,\ldots is the \textit{cloth}
magical?''}

{
 I shrug. ``I ordered it online from a company
called Natural Selections. The fabric is called Sensory
Modality.'' I pause, seeing the incredulous
expressions of Mark and Autrey. ``I admit the names
are a bit New Agey. The point is that a passing sheep triggers a chain
of cause and effect that ends with a pebble in the bucket.
\textit{Afterward} you can compare the bucket to other buckets, and so
on.''}

{
 ``I still don't get
it,'' Mark says. ``You
can't fit a sheep into a bucket. Only pebbles go in
buckets, and it's obvious that pebbles only interact
with other pebbles.''}

{
 ``The sheep interact with things that interact
with pebbles\,\ldots'' I search for an analogy.
``Suppose you look down at your shoelaces. A photon
leaves the Sun; then travels down through Earth's
atmosphere; then bounces off your shoelaces; then passes through the
pupil of your eye; then strikes the retina; then is absorbed by a rod
or a cone. The photon's energy makes the attached
neuron fire, which causes other neurons to fire. A neural activation
pattern in your visual cortex can interact with your beliefs about your
shoelaces, since beliefs about shoelaces also exist in neural
substrate. If you can understand that, you should be able to see how a
passing sheep causes a pebble to enter the bucket.''}

{
 ``At exactly \textit{which} point in the process
does the pebble become magic?'' says Mark.}

{
 ``It\,\ldots um\,\ldots'' Now
\textit{I'm} starting to get confused. I shake my head
to clear away cobwebs. This all seemed simple enough when I woke up
this morning, and the pebble-and-bucket system hasn't
gotten any more complicated since then. ``This is a
lot easier to understand if you remember that the \textit{point} of the
system is to keep track of sheep.''}

{
 Mark sighs sadly. ``Never mind\,\ldots
it's obvious you don't know. Maybe all
pebbles are magical to start with, even before they enter the bucket.
We could call that position panpebblism.''}

{
 ``Ha!'' Autrey says, scorn rich
in his voice. ``Mere wishful thinking! Not all pebbles
are created equal. The pebbles in \textit{your} bucket are \textit{not}
magical. They're only lumps of
stone!''}

{
 Mark's face turns stern.
``Now,'' he cries,
``now you see the danger of the road you walk! Once
you say that some people's pebbles are magical and some
are not, your pride will consume you! You will think yourself superior
to all others, and so fall! Many throughout history have tortured and
murdered because they thought their own pebbles
supreme!'' A tinge of condescension enters
Mark's voice. ``Worshipping a level of
pebbles as `magical' implies that
there's an absolute pebble level in a Supreme Bucket.
Nobody believes in a Supreme Bucket these days.''}

{
 ``One,'' I say.
``Sheep are not absolute pebbles. Two, I
don't think my bucket actually contains the sheep.
Three, I don't worship my bucket level as perfect---I
adjust it sometimes---and I do that \textit{because} I care about the
sheep.''}

{
 ``Besides,'' says Autrey,
``someone who believes that possessing absolute
pebbles \textit{would} license torture and murder, is making a mistake
that has nothing to do with buckets. You're solving the
wrong problem.''}

{
 Mark calms himself down. ``I suppose I
can't expect any better from mere shepherds. You
probably believe that snow is white, don't
you.''}

{
 ``Um\,\ldots yes?'' says Autrey.}

{
 ``It doesn't bother you that
\textit{Joseph Stalin} believed that snow is
white?''}

{
 ``Um\,\ldots no?'' says Autrey.}

{
 Mark gazes incredulously at Autrey, and finally shrugs.
``Let's suppose, purely for the sake
of argument, that your pebbles are magical and mine
aren't. Can you tell me what the difference
is?''}

{
 ``My pebbles \textit{represent} the
sheep!'' Autrey says triumphantly.
``\textit{Your} pebbles don't have the
representativeness property, so they won't work. They
are empty of meaning. Just look at them. There's no
aura of semantic content; they are merely pebbles. You need a bucket
with special causal powers.''}

{
 ``Ah!'' Mark says.
``Special causal powers, instead of
magic.''}

{
 ``Exactly,'' says Autrey.
``I'm not superstitious. Postulating
magic, in this day and age, would be unacceptable to the international
shepherding community. We have found that postulating magic simply
doesn't work as an explanation for shepherding
phenomena. So when I see something I don't understand,
and I want to explain it using a model with no internal detail that
makes no predictions even in retrospect, I postulate special causal
powers. If that doesn't work, I'll move
on to calling it an emergent phenomenon.''}

{
 ``What kind of special powers does the bucket
have?'' asks Mark.}

{
 ``Hm,'' says Autrey.
``Maybe this bucket is imbued with an
\textit{about-ness} relation to the pastures. That would explain why it
worked---when the bucket is empty, it \textit{means} the pastures are
empty.''}

{
 ``Where did you find this
bucket?'' says Mark. ``And how did
you realize it had an about-ness relation to the
pastures?''}

{
 ``It's an \textit{ordinary
bucket},'' I say. ``I used to climb
trees with it\,\ldots I don't think this question
\textit{needs} to be difficult.''}

{
 ``I'm talking to
Autrey,'' says Mark.}

{
 ``You have to bind the bucket to the pastures,
and the pebbles to the sheep, using a magical ritual---pardon me, an
emergent process with special causal powers---that my master
discovered,'' Autrey explains.}

{
 Autrey then attempts to describe the ritual, with Mark nodding
along in sage comprehension.}

{
 ``You have to throw in a pebble \textit{every}
time a sheep leaves through the gate?'' says Mark.
``Take out a pebble \textit{every} time a sheep
returns?''}

{
 Autrey nods. ``Yeah.''}

{
 ``That must be really hard,''
Mark says sympathetically.}

{
 Autrey brightens, soaking up Mark's sympathy like
rain. ``Exactly!'' says Autrey.
``It's \textit{extremely} hard on your
emotions. When the bucket has held its level for a while, you\,\ldots
tend to get attached to that level.''}

{
 A sheep passes then, leaving through the gate. Autrey sees; he
stoops, picks up a pebble, holds it aloft in the air.
``Behold!'' Autrey proclaims.
``A sheep has passed! I must now toss a pebble into
this bucket, my dear bucket, and destroy that fond level which has held
for so long---'' Another sheep passes. Autrey, caught
up in his drama, misses it; so I plunk a pebble into the bucket. Autrey
is still speaking: ``---for that is the supreme test
of the shepherd, to throw in the pebble, be it ever so agonizing, be
the old level ever so precious. Indeed, only the best of shepherds can
meet a requirement so stern---''}

{
 ``Autrey,'' I say,
``if you want to be a great shepherd someday, learn to
shut up and throw in the pebble. No fuss. No drama. Just do
it.''}

{
 ``And this ritual,'' says Mark,
``it binds the pebbles to the sheep by the magical
laws of Sympathy and Contagion, like a voodoo
doll.''}

{
 Autrey winces and looks around. ``Please!
Don't call it Sympathy and Contagion. We shepherds are
an anti-superstitious folk. Use the word
`intentionality,' or something like
that.''}

{
 ``Can I look at a pebble?''
says Mark.}

{
 ``Sure,'' I say. I take one of
the pebbles out of the bucket, and toss it to Mark. Then I reach to the
ground, pick up another pebble, and drop it into the bucket.}

{
 Autrey looks at me, puzzled.
``Didn't you just mess it
up?''}

{
 I shrug. ``I don't think so.
We'll know I messed it up if there's a
dead sheep next morning, or if we search for a few hours and
don't find any sheep.''}

{
 ``But---'' Autrey says.}

{
 ``I taught you everything \textit{you} know, but
I haven't taught you everything \textit{I}
know,'' I say.}

{
 Mark is examining the pebble, staring at it intently. He holds his
hand over the pebble and mutters a few words, then shakes his head.
``I don't sense any magical
power,'' he says. ``Pardon me. I
don't sense any intentionality.''}

{
 ``A pebble only has intentionality if
it's inside a ma---an emergent
bucket,'' says Autrey. ``Otherwise
it's just a mere pebble.''}

{
 ``Not a problem,'' I say. I
take a pebble out of the bucket, and toss it away. Then I walk over to
where Mark stands, tap his hand holding a pebble, and say:
``I declare this hand to be part of the magic
bucket!'' Then I resume my post at the gates.}

{
 Autrey laughs. ``Now you're just
being gratuitously evil.''}

{
 I nod, for this is indeed the case.}

{
 ``Is that really going to work,
though?'' says Autrey.}

{
 I nod again, hoping that I'm right.
I've done this before with two buckets, and in
principle, there should be no difference between Mark's
hand and a bucket. Even if Mark's hand is imbued with
the \textit{élan vital} that distinguishes live matter from dead
matter, the trick should work as well as if Mark were a marble statue.}

{
  Mark is looking at his hand, a bit unnerved. ``So\,\ldots
  the pebble has intentionality again, now?''}

{
 ``Yep,'' I say.
``Don't add any more pebbles to your
hand, or throw away the one you have, or you'll break
the ritual.''}

{
 Mark nods solemnly. Then he resumes inspecting the pebble.
``I understand now how your flocks grew so
great,'' Mark says. ``With the power
of this bucket, you could keep on tossing pebbles, and the sheep would
keep returning from the fields. You could start with just a few sheep,
let them leave, then fill the bucket to the brim before they returned.
And if tending so many sheep grew tedious, you could let them all
leave, then empty almost all the pebbles from the bucket, so that only
a few returned\,\ldots increasing the flocks again when it came time for
shearing\,\ldots dear heavens, man! Do you realize the sheer
\textit{power} of this ritual you've discovered? I can
only imagine the implications; humankind might leap ahead a
decade---no, a century!''}

{
 ``It doesn't work that
way,'' I say. ``If you add a pebble
when a sheep hasn't left, or remove a pebble when a
sheep hasn't come in, that breaks the ritual. The power
does not linger in the pebbles, but vanishes all at once, like a soap
bubble popping.''}

{
 Mark's face is terribly disappointed.
``Are you sure?''}

{
 I nod. ``I tried that and it
didn't work.''}

{
 Mark sighs heavily. ``And this\,\ldots
\textit{math}\,\ldots seemed so powerful and useful until then\,\ldots Oh,
well. So much for human progress.''}

{
 ``Mark, it was a \textit{brilliant}
idea,'' Autrey says encouragingly.
``The notion didn't occur to me, and
yet it's so obvious\,\ldots it would save an
\textit{enormous} amount of effort\,\ldots there \textit{must} be a way
to salvage your plan! We could try different buckets, looking for one
that would keep the magical pow---the intentionality in the pebbles,
even without the ritual. Or try other pebbles. Maybe our pebbles just
have the wrong properties to have \textit{inherent} intentionality.
What if we tried it using stones carved to resemble tiny sheep? Or just
write `sheep' on the pebbles; that might
be enough.''}

{
 ``Not going to work,'' I
predict dryly.}

{
 Autrey continues. ``Maybe we need organic
pebbles, instead of silicon pebbles\,\ldots or maybe we need to use
expensive gemstones. The price of gemstones doubles every eighteen
months, so you could buy a handful of cheap gemstones now, and wait,
and in twenty years they'd be really
expensive.''}

{
 ``You tried adding pebbles to create more sheep,
and it didn't work?'' Mark asks me.
``What exactly did you do?''}

{
 ``I took a handful of dollar bills. Then I hid
the dollar bills under a fold of my blanket, one by one; each time I
hid another bill, I took another paperclip from a box, making a small
heap. I was careful not to keep track in my head, so that all I knew
was that there were `many' dollar bills,
and `many' paperclips. Then when all the
bills were hidden under my blanket, I added a single additional
paperclip to the heap, the equivalent of tossing an extra pebble into
the bucket. Then I started taking dollar bills from under the fold, and
putting the paperclips back into the box. When I finished, a single
paperclip was left over.''}

{
 ``What does that result mean?''
asks Autrey.}

{
 ``It means the trick didn't work.
Once I broke ritual by that single misstep, the power did not linger,
but vanished instantly; the heap of paperclips and the pile of dollar
bills no longer went empty at the same time.''}

{
 ``You \textit{actually} tried
this?'' asks Mark.}

{
 ``Yes,'' I say,
``I actually performed the experiment, to verify that
the outcome matched my theoretical prediction. I have a sentimental
fondness for the scientific method, even when it seems absurd. Besides,
what if I'd been wrong?''}

{
 ``If it \textit{had} worked,''
says Mark, ``you would have been guilty of
counterfeiting! Imagine if everyone did that; the economy would
collapse! Everyone would have billions of dollars of currency, yet
there would be nothing for money to buy!''}

{
 ``Not at all,'' I reply.
``By that same logic whereby adding another paperclip
to the heap creates another dollar bill, creating another dollar bill
would create an additional dollar's worth of goods and
services.''}

{
 Mark shakes his head. ``Counterfeiting is still a
crime\,\ldots You should not have tried.''}

{
 ``I was \textit{reasonably} confident I would
fail.''}

{
 ``Aha!'' says Mark.
``You \textit{expected} to fail! You
didn't \textit{believe} you could do
it!''}

{
 ``Indeed,'' I admit.
``You have guessed my expectations with stunning
accuracy.''}

{
 ``Well, that's the
problem,'' Mark says briskly.
``Magic is fueled by belief and willpower. If you
don't believe you can do it, you can't.
You need to change your belief about the experimental result; that will
change the result itself.''}

{
 ``Funny,'' I say nostalgically,
``that's what Autrey said when I told
him about the pebble-and-bucket method. That it was too ridiculous for
him to believe, so it wouldn't work for
him.''}

{
 ``How did you persuade him?''
inquires Mark.}

{
 ``I told him to shut up and follow
instructions,'' I say, ``and when
the method worked, Autrey started believing in it.''}

{
 Mark frowns, puzzled. ``That makes no sense. It
doesn't resolve the essential chicken-and-egg
dilemma.''}

{
 ``Sure it does. The bucket method works whether
or not you believe in it.''}

{
 ``That's
\textit{absurd!}'' sputters Mark.
``I don't believe in magic that works
whether or not you believe in it!''}

{
 ``I said that too,'' chimes in
Autrey. ``Apparently I was wrong.''}

{
  Mark screws up his face in concentration. ``But\,\ldots
  if you didn't believe in magic that works whether
or not you believe in it, then why did the bucket method work when you
didn't believe in it? Did you believe in magic that
works whether or not you believe in it whether or not you believe in
magic that works whether or not you believe in it?''}

{
  ``I don't\,\ldots \textit{think} so\,\ldots''
  says Autrey doubtfully.}

{
 ``Then if you didn't believe in
magic that works whether or not you\,\ldots hold on a second, I need to
work this out with paper and pencil---'' Mark
scribbles frantically, looks skeptically at the result, turns the piece
of paper upside down, then gives up. ``Never
mind,'' says Mark. ``Magic is
difficult enough for me to comprehend; metamagic is out of my
depth.''}

{
 ``Mark, I don't think you
understand the art of bucketcraft,'' I say.
``It's not about using pebbles to
control sheep. It's about making sheep control pebbles.
In this art, it is not necessary to begin by believing the art will
work. Rather, first the art works, then one comes to believe that it
works.''}

{
 ``Or so you believe,'' says
Mark.}

{
 ``So I believe,'' I reply,
``\textit{because} it happens to be a fact. The
correspondence between reality and my beliefs comes from reality
controlling my beliefs, not the other way around.''}

{
 Another sheep passes, causing me to toss in another pebble.}

{
 ``Ah! Now we come to the root of the
problem,'' says Mark.
``What's this so-called
`reality' business? I understand what it
means for a hypothesis to be elegant, or falsifiable, or compatible
with the evidence. It sounds to me like calling a belief
`true' or
`real' or
`actual' is merely the difference
between saying you believe something, and saying you really really
believe something.''}

\label{reality_defined}
{
 I pause. ``Well\,\ldots'' I say
slowly. ``Frankly, I'm not entirely
sure myself where this `reality'
business comes from. I can't create my own reality in
the lab, so I must not understand it yet. But occasionally I believe
strongly that something is going to happen, and then something else
happens instead. I need a name for whatever-it-is that determines my
experimental results, so I call it
`reality'. This
`reality' is somehow separate from even
my very best hypotheses. Even when I have a simple hypothesis, strongly
supported by all the evidence I know, sometimes I'm
still surprised. So I need different names for the thingies that
determine my predictions and the thingy that determines my experimental
results. I call the former thingies
`belief,' and the latter thingy
`reality.'\,''}

{
 Mark snorts. ``I don't even know
why I bother listening to this obvious nonsense. Whatever you say about
this so-called `reality,' it is merely
another belief. Even your belief that reality precedes your beliefs is
a belief. It follows, as a logical inevitability, that reality does not
exist; only beliefs exist.''}

{
 ``Hold on,'' says Autrey,
``could you repeat that last part? You lost me with
that sharp swerve there in the middle.''}

{
 ``No matter what you say about reality,
it's just another belief,'' explains
Mark. ``It follows with crushing necessity that there
is no reality, only beliefs.''}

{
 ``I see,'' I say.
``The same way that no matter what you eat, you need
to eat it with your mouth. It follows that there is no food, only
mouths.''}

{
 ``Precisely,'' says Mark.
``Everything that you eat has to be in your mouth. How
can there be food that exists outside your mouth? The thought is
nonsense, proving that `food' is an
incoherent notion. That's why we're all
starving to death; there's no
food.''}

{
 Autrey looks down at his stomach. ``But
I'm \textit{not} starving to
death.''}

{
 ``\textit{Aha!}'' shouts Mark
triumphantly. ``And how did you utter that very
objection? With your \textit{mouth}, my friend! With your
\textit{mouth}! What better demonstration could you ask that there is
no food?''}

{
 ``\textit{What's this about
starvation?}'' demands a harsh, rasping voice from
directly behind us. Autrey and I stay calm, having gone through this
before. Mark leaps a foot in the air, startled almost out of his wits.}

{
 Inspector Darwin smiles tightly, pleased at achieving surprise,
and makes a small tick on his clipboard.}

{
 ``Just a metaphor!'' Mark says
quickly. ``You don't need to take away
my mouth, or anything like that---''}

{
 ``\textit{Why} do you need a \textit{mouth} if
there is no \textit{food}?'' demands Darwin angrily.
``\textit{Never mind.} I have no \textit{time} for
this \textit{foolishness}. I am here to inspect the
\textit{sheep.}''}

{
 ``Flock's thriving,
sir,'' I say. ``No dead sheep since
January.''}

{
 ``\textit{Excellent.} I award you 0.12 units of
\textit{fitness}. Now what is this \textit{person} doing here? Is he a
necessary part of the \textit{operations}?''}

{
 ``As far as I can see, he would be of more use to
the human species if hung off a hot-air balloon as
ballast,'' I say.}

{
 ``Ouch,'' says Autrey mildly.}

{
 ``I do not \textit{care} about the \textit{human
species}. Let him speak for \textit{himself}.''}

{
 Mark draws himself up haughtily. ``This mere
\textit{shepherd},'' he says, gesturing at me,
``has claimed that there is such a thing as reality.
This offends me, for I know with deep and abiding certainty that there
is no truth. The concept of `truth' is
merely a stratagem for people to impose their own beliefs on others.
Every culture has a different `truth,'
and no culture's `truth'
is superior to any other. This that I have said holds at all times in
all places, and I insist that you agree.''}

{
 ``Hold on a second,'' says
Autrey. ``If nothing is true, why should I believe you
when you say that nothing is true?''}

{
 ``I didn't say that nothing is
true---'' says Mark.}

{
 ``Yes, you did,'' interjects
Autrey, ``I heard you.''}

{
 ``---I said that
`truth' is an excuse used by some
cultures to enforce their beliefs on others. So when you say something
is `true,' you mean only that it would
be advantageous to your own social group to have it
believed.''}

{
 ``And this that you have
said,'' I say, ``is it
true?''}

{
 ``Absolutely, positively
true!'' says Mark emphatically.
``People create their own
realities.''}

{
 ``Hold on,'' says Autrey,
sounding puzzled again, ``saying that people create
their own realities is, logically, a completely separate issue from
saying that there is no truth, a state of affairs I cannot even imagine
coherently, perhaps because you still have not explained how exactly it
is supposed to work---''}

{
 ``There you go again,'' says
Mark exasperatedly, ``trying to apply your Western
concepts of logic, rationality, reason, coherence, and
self-consistency.''}

{
 ``Great,'' mutters Autrey,
``now I need to add a \textit{third} subject heading,
to keep track of this entirely separate and distinct
claim---''}

{
 ``It's not
separate,'' says Mark. ``Look,
you're taking the wrong attitude by treating my
statements as hypotheses, and carefully deriving their consequences.
You need to think of them as fully general excuses, which I apply when
anyone says something I don't like.
It's not so much a model of how the universe works, as
a Get Out of Jail Free card. The \textit{key} is to apply the excuse
\textit{selectively}. When I say that there is no such thing as truth,
that applies only to \textit{your} claim that the magic bucket works
whether or not I believe in it. It does \textit{not} apply to
\textit{my} claim that there is no such thing as
truth.''}

{
 ``Um\,\ldots why not?'' inquires
Autrey.}

{
 Mark heaves a patient sigh. ``Autrey, do you
think you're the first person to think of that
question? To ask us how our own beliefs can be meaningful if all
beliefs are meaningless? That's the same thing many
students say when they encounter this philosophy, which,
I'll have you know, has many adherents and an extensive
literature.''}

{
 ``So what's the
answer?'' says Autrey.}

{
 ``We named it the `reflexivity
problem,'\,'' explains Mark.}

{
 ``But what's the
\textit{answer}?'' persists Autrey.}

{
 Mark smiles condescendingly. ``Believe me,
Autrey, you're not the first person to think of such a
simple question. There's no point in presenting it to
us as a triumphant refutation.''}

{
 ``But what's the \textit{actual
answer?}''}

{
 ``Now, I'd like to move on to the
issue of how logic kills cute baby seals---''}

{
 ``\textit{You} are wasting
\textit{time},'' snaps Inspector Darwin.}

{
 ``Not to mention, losing track of
sheep,'' I say, tossing in another pebble.}

{
 Inspector Darwin looks at the two arguers, both apparently
unwilling to give up their positions.
``Listen,'' Darwin says, more kindly
now, ``I have a simple notion for resolving your
dispute. \textit{You} say,'' says Darwin, pointing to
Mark, ``that people's beliefs alter
their personal realities. And \textit{you} fervently
believe,'' his finger swivels to point at Autrey,
``that Mark's beliefs
\textit{can't} alter reality. So let Mark believe
really hard that he can fly, and then step off a cliff. Mark shall see
himself fly away like a bird, and Autrey shall see him plummet down and
go splat, and you shall both be happy.''}

{
 We all pause, considering this.}

{
 ``It \textit{sounds} reasonable\,\ldots'' Mark says finally.}

{
 ``There's a cliff right
there,'' observes Inspector Darwin.}

{
 Autrey is wearing a look of intense concentration. Finally he
shouts: ``Wait! If that were true, we would all have
long since departed into our own private universes, in which case the
other people here are only figments of your
imagination---there's no point in trying to prove
anything to us---''}

{
 A long dwindling scream comes from the nearby cliff, followed by a
dull and lonely splat. Inspector Darwin flips his clipboard to the page
that shows the current gene pool and pencils in a slightly lower
frequency for Mark's alleles.}

{
 Autrey looks slightly sick. ``Was that really
necessary?''}

{
 ``\textit{Necessary?}'' says
Inspector Darwin, sounding puzzled. ``It just
\textit{happened}\,\ldots I don't quite understand your
question.''}

{
 Autrey and I turn back to our bucket. It's time to
bring in the sheep. You wouldn't want to forget about
that part. Otherwise what would be the point?}

\myendsectiontext



